%%---------------------------------------------------------------------------%%
%% Monte Carlo Synthetic Acceleration Paper
%% Submitted to Journal of Computational Physics 2012
%%---------------------------------------------------------------------------%%
%%
%% Copyright 2007, 2008, 2009 Elsevier Ltd
%%
%% This file is part of the 'Elsarticle Bundle'.
%% ---------------------------------------------
%%
%% It may be distributed under the conditions of the LaTeX Project Public
%% License, either version 1.2 of this license or (at your option) any
%% later version.  The latest version of this license is in
%%    http://www.latex-project.org/lppl.txt
%% and version 1.2 or later is part of all distributions of LaTeX
%% version 1999/12/01 or later.
%%
%% The list of all files belonging to the 'Elsarticle Bundle' is
%% given in the file `manifest.txt'.
%%

%%---------------------------------------------------------------------------%%
%% Document Options
%%---------------------------------------------------------------------------%%
%% Template article for Elsevier's document class `elsarticle'
%% with numbered style bibliographic references
%% SP 2008/03/01
%%
%%
%%
%% $Id: elsarticle-template-num.tex 4 2009-10-24 08:22:58Z rishi $
%%
%%
\documentclass[preprint,12pt]{elsarticle}

%% Use the option review to obtain double line spacing
%% \documentclass[preprint,review,12pt]{elsarticle}

%% Use the options 1p,twocolumn; 3p; 3p,twocolumn; 5p; or 5p,twocolumn
%% for a journal layout:
%% \documentclass[final,1p,times]{elsarticle}
%% \documentclass[final,1p,times,twocolumn]{elsarticle}
%% \documentclass[final,3p,times]{elsarticle}
%% \documentclass[final,3p,times,twocolumn]{elsarticle}
%% \documentclass[final,5p,times]{elsarticle}
%% \documentclass[final,5p,times,twocolumn]{elsarticle}

%% if you use PostScript figures in your article
%% use the graphics package for simple commands
%% \usepackage{graphics}
%% or use the graphicx package for more complicated commands
%% \usepackage{graphicx}
%% or use the epsfig package if you prefer to use the old commands
%% \usepackage{epsfig}

%% The amssymb package provides various useful mathematical symbols
\usepackage{amssymb}
%% The amsthm package provides extended theorem environments
%% \usepackage{amsthm}

%% The lineno packages adds line numbers. Start line numbering with
%% \begin{linenumbers}, end it with \end{linenumbers}. Or switch it on
%% for the whole article with \linenumbers after \end{frontmatter}.
%% \usepackage{lineno}

%% natbib.sty is loaded by default. However, natbib options can be
%% provided with \biboptions{...} command. Following options are
%% valid:

%%   round  -  round parentheses are used (default)
%%   square -  square brackets are used   [option]
%%   curly  -  curly braces are used      {option}
%%   angle  -  angle brackets are used    <option>
%%   semicolon  -  multiple citations separated by semi-colon
%%   colon  - same as semicolon, an earlier confusion
%%   comma  -  separated by comma
%%   numbers-  selects numerical citations
%%   super  -  numerical citations as superscripts
%%   sort   -  sorts multiple citations according to order in ref. list
%%   sort&compress   -  like sort, but also compresses numerical citations
%%   compress - compresses without sorting
%%
%% \biboptions{comma,round}

% \biboptions{}

\journal{Computational Physics}

%%---------------------------------------------------------------------------%%
%% Math Variables
%%---------------------------------------------------------------------------%%
\newcommand{\email}[1]{$\langle$#1@lanl.gov$\rangle$}

\newcommand{\vA}{\ensuremath{\ve{A}}}
\newcommand{\vb}{\ensuremath{\ve{b}}}
\newcommand{\vx}{\ensuremath{\ve{x}}}
\newcommand{\vr}{\ensuremath{\ve{r}}}
\newcommand{\vI}{\ensuremath{\ve{I}}}
\newcommand{\vH}{\ensuremath{\ve{H}}}
\newcommand{\vP}{\ensuremath{\ve{P}}}

\newcommand{\Cv}{\ensuremath{C_{v}}}
\newcommand{\ros}{\ensuremath{\sigma_{\scriptscriptstyle\mathrm{R}}}}
\newcommand{\bOmega}{\ensuremath{\mathbf{\Omega}}}

\newcommand{\dt}{\ensuremath{\Delta t}}
\newcommand{\sign}{\ensuremath{\tilde{\sigma}^n}}
\newcommand{\qn}{\ensuremath{q^n}}
\newcommand{\Tn}{\ensuremath{T^n}}
\newcommand{\Dn}{\ensuremath{D^n}}
\newcommand{\phin}{\ensuremath{\phi^n}}
\newcommand{\Di}{\ensuremath{\Delta_i}}
\newcommand{\Dj}{\ensuremath{\Delta_j}}
\newcommand{\Dk}{\ensuremath{\Delta_k}}

\newcommand{\sigT}{\ensuremath{\sigma_{T\,ijk}}}
\newcommand{\sigm}{\ensuremath{\sigma^{-}}}
\newcommand{\sigp}{\ensuremath{\sigma^{+}}}

\newcommand{\bphi}{\ensuremath{\boldsymbol{\phi}}}

\DeclareMathOperator{\diag}{diag}

%%---------------------------------------------------------------------------%%
\begin{document}

\begin{frontmatter}

  %% Title, authors and addresses

  %% use the tnoteref command within \title for footnotes;
  %% use the tnotetext command for the associated footnote;
  %% use the fnref command within \author or \address for footnotes;
  %% use the fntext command for the associated footnote;
  %% use the corref command within \author for corresponding author footnotes;
  %% use the cortext command for the associated footnote;
  %% use the ead command for the email address,
  %% and the form \ead[url] for the home page:
  %%
  %% \title{Title\tnoteref{label1}}
  %% \tnotetext[label1]{}
  %% \author{Name\corref{cor1}\fnref{label2}}
  %% \ead{email address}
  %% \ead[url]{home page}
  %% \fntext[label2]{}
  %% \cortext[cor1]{}
  %% \address{Address\fnref{label3}}
  %% \fntext[label3]{}

  \title{A Monte Carlo Synthetic Acceleration Method for Solving the
    Thermal Radiation Diffusion Equation}

  %% use optional labels to link authors explicitly to addresses:
  %% \author[label1,label2]{<author name>}
  %% \address[label1]{<address>}
  %% \address[label2]{<address>}

  \author{}

  \address{}

  %%---------------------------------------------------------------------------%%
  \begin{abstract}

    We present a synthetic-acceleration based Monte Carlo method for
    solving the 1T thermal radiation diffusion equations.  We show
    that this method can be an effective solver for sparse matrix
    systems.  We also discuss its general applicability to broader
    classes of problems.

  \end{abstract}

  %%---------------------------------------------------------------------------%%
  \begin{keyword}
    %% keywords here, in the form: keyword \sep keyword

    %% MSC codes here, in the form: \MSC code \sep code
    %% or \MSC[2008] code \sep code (2000 is the default)

  \end{keyword}

\end{frontmatter}

%%
%% Start line numbering here if you want
%%
% \linenumbers

%% main text
%%---------------------------------------------------------------------------%%
\section{Introduction}
\label{sec:introduction}

In Ref.~\cite{evans_2003}, we developed a residual Monte Carlo method
that was an application of Halton's Sequential Monte Carlo
method~\cite{halton_1962,halton_1994} to the 1D equilibrium (1T)
thermal radiation diffusion equation.  At that time, Halton's method
was not widely known in computational physics.  While extending our
method to 3D, we have discovered that the Sequential Monte Carlo
method is actually a variant of iterative refinement cast as a
residual method.  Using Monte Carlo to accelerate fixed-point
(Richardson) iteration, our new method surpasses our previous 1D
method and allows for extension to more general solution techniques.

The purpose of this study is to demonstrate an efficient Monte Carlo
solution method for 3D, time-dependent discrete systems.  In this
note, we extend our previous work in the following manner:
\begin{itemize}
\item We develop a rigorous synthetic-acceleration Monte Carlo method
  for solving sparse matrix systems.
\item We apply this solver to the 1T thermal radiation diffusion
  equation in 3D.
\end{itemize}
In \S~\ref{sec:monte-carlo-matrix} we review Monte Carlo methods for
solving matrix systems. In \S~\ref{sec:iter-refin-monte} we discuss
our new synthetic-acceleration scheme and connect with previous work
by Halton~\cite{halton_1994} and others~\cite{evans_2003}.  The 1T
radiation diffusion equation and its discrete form will be derived in
\S~\ref{sec:discr-form-radi}.  We discuss using Synthetic-Accleration
Monte Carlo to solve the discrete radiation diffusion equation in
\S~\ref{sec:solut-radi-diff}.  Results from two 3D test problems are
given in \S~\ref{sec:results}.  We finish with conclusions and ideas
for future work in \S~\ref{sec:conclusions}.

%%---------------------------------------------------------------------------%%
\section{Monte Carlo Matrix Solution Methods}
\label{sec:monte-carlo-matrix}

Consider the following matrix equation:
\begin{equation}
  \vA\vx = \vb\:,
  \label{eq:Ax=b}
\end{equation}
which can be written,
\begin{equation}
  \begin{split}
    \vx &= (\vI - \vA)\vx + \vb\\
    &= \vH\vx + \vb\:.
  \end{split}
  \label{eq:point_iteration}
\end{equation}
Here, $\vH = (\vI - \vA)$ is the \latin{iteration matrix}.  When the
spectral radius, $\rho(\vH)$, of the iteration matrix satisfies
\begin{equation}
  \rho(\vH) < 1\:,
\end{equation}
we can expand $\vA^{-1}$ using the \latin{Neumann Series},
\begin{equation}
  \vA^{-1} = (\vI - \vH)^{-1} =  \sum_{k=0}^{\infty} \vH^k\:.
\end{equation} 
Thus, when $\rho(\vH) < 1$ we can recast the solution vector $\vx$ as
a series,
\begin{equation}
  \begin{split}
    \vx &= (\vI - \vH)^{-1}\vb\\
    &= \vb + \vH\vb + \vH^2\vb + \vH^3\vb + \ldots\:.
    \label{eq:neumann_series}
  \end{split}
\end{equation}
With this knowledge in hand, we can write an iterative method that
solves Eq.~(\ref{eq:Ax=b}) (Richardson's Iteration),
\begin{equation}
  \vx^{k+1} = \vH\vx^k + \vb\:.
  \label{eq:richardson}
\end{equation}
Equation~(\ref{eq:richardson}) will converge when $\rho(\vH) < 1\ 
\forall\ \vb \in \mathcal{R}^{N}$~\cite{kelley_1995}.

\subsection{Direct Method}
\label{sec:direct-method}

Now we consider a Monte Carlo method that can be used to estimate the solution
to Eq.~(\ref{eq:Ax=b})~\cite{hammersley_1964}.  First, rewrite
Eq.~(\ref{eq:neumann_series}) in order to calculate a component of
$\vx$,
\begin{equation}
  \begin{split}
    x_i &= (\vb)_i + (\vH\vb)_i + (\vH^2\vb)_i + (\vH^3\vb)_i +
    \ldots\\
    &= \sum_{k=0}^{\infty}\sum_{i_1}^{N}\sum_{i_2}^{N}\ldots
    \sum_{i_k}^{N}h_{i,i_1}h_{i_1,i_2}\ldots h_{i_{k-1},i_k}b_{i_k}\:.
  \end{split}
  \label{eq:neumann_series_i}
\end{equation}
This series can be interpreted as a series of transitions from
$i_{m-1}\rightarrow i_m$ that can be simulated by a random walk.
Let $X$ be a random variable sampled from a random walk with $k$
events that initiates in state $i$,
\begin{equation}
  \begin{split}
    X(i_0 = i) &= \sum_{m=0}^{k}W_m b_{i_m}\\
    &= \sum_{m=0}^{k}w_{i,i_1}w_{i_1,i_2}\ldots
    w_{i_{m-1},i_m}b_{i_m}\:.
  \end{split}
  \label{eq:estimator_xi}
\end{equation}
Here, the particle weight on the $m^\text{th}$ step is denoted $W_m$,
and each particle starts with unit weight. When the particle
transitions between states, for example $i\rightarrow j$, the weight
is multiplied by the factor
\begin{equation}
  w_{ij} = \frac{h_{ij}}{p_{ij}}\:.
  \label{eq:weight}
\end{equation}
Then, the expected value of $X$ is
\begin{equation}
  \begin{split}
    E[X(i_0=i)] &= \sum_{\nu}P_\nu X_\nu\\
    &= \sum_{k=0}^{\infty}\sum_{i_1}^{N}\sum_{i_2}^{N}\ldots
    \sum_{i_k}^{N}p_{i,i_1}p_{i_1,i_2}\ldots p_{i_{k-1},i_k}
    w_{i,i_1}w_{i_1,i_2}\dots w_{i_{k-1},i_k}b_{i_k}\\
    &= x_i\:,
  \end{split}
  \label{eq:expectation_xi}
\end{equation}
where $\nu$ denotes a particular random walk permutation.
Therefore, the estimator in Eq.~(\ref{eq:estimator_xi}) is an unbiased
estimator of the components of $\vx$ provided $\rho(\vH) < 1$.

We are left to define the transition probabilities, $p_{ij}$. The most
straightforward approach is to set
\begin{equation}
  p_{ij} = \frac{|h_{ij}|}{\sum_{j}|h_{ij}|}\:.
  \label{eq:probability}
\end{equation}
Each row of the transition probability matrix, $\vP$, represents a
discrete probability density function for selecting a new state $j$,
given that the current state is $i$.  Random walks can be terminated
in two ways: the matrix can be augmented with a terminating event
equation that describes the probability that a particle ends its walk,
or the random walk can be terminated by weight cutoff.  Generally, we
choose to terminate random walks using a weight cutoff as opposed to
augmenting the matrix.

\subsection{Adjoint Method}
\label{sec:adjoint-method}

An alternative approach to the one just described is to calculate
contributions to every component of $\vx$ during the random walk.  In
the mathematical literature this is referred to as an adjoint method;
however, to transport practitioners this approach more closely
resembles a traditional forward method.  In this method, the weight
change from state $i\rightarrow j$ is
\begin{equation}
  w_{ij} = \frac{h_{ji}}{p_{ij}}\:.
  \label{eq:adjoint-weight}
\end{equation}
The transition probabilities may be calculated as
\begin{equation}
  p_{ij} = \frac{|h_{ji}|}{\sum_{j}|h_{ji}|}\:.
  \label{eq:adjoint-probability}
\end{equation}
Note that the indices are reversed so that the probabilities are
normalized over a column, as opposed to the forward method in which
the probabilities are normalized over a row.  This is equivalent to
forming the Neumann series in reverse order.  Correspondingly, the
estimator for this method is
\begin{equation}
  \begin{split}
    X &= \sum_{m=0}^{k}W_m\delta_{i_m,i}\\
    &= \sum_{m=0}^{k}\hat{b}_{i_0}w_{i_0,i_1}w_{i_1,i_2}\ldots
                                  w_{i_{m-1},i_m}\delta_{i_m,i}\:.
  \end{split}
  \label{eq:adjoint-tally}
\end{equation}
Here, $\hat{b}_{i_0}$ is the sampled source in cell/state $i_0$.  The
Kronecker delta implies that tallies are only made in the cell/state
where the random walk currently resides.  This is the common approach
in standard Monte Carlo transport simulations.  We will investigate
this approach as applied to the 1T thermal radiation diffusion
problem in \S~\ref{sec:solut-radi-diff}.

Similar to the Direct Method, the Adjoint Method random walk process
requires a terminating condition.  In all of the work that follows we
utilize a relative weight cutoff.  The weight cutoff is defined as a
fraction of the starting weight such that
\begin{equation}
  W_f = W_c\hat{b}_{i_0}\:,
  \label{eq:weight_cutoff}
\end{equation}
where $W_f$ is the terminating weight of the random walk, $W_c$ is the
input relative weight cutoff, and $\hat{b}_{i_0}$ is the initial
weight of the source particle.  The random walk ends on the
$m^\text{th}$ step if $W_m < W_f$.

%%---------------------------------------------------------------------------%%
\section{Synthetic-Acceleration Monte Carlo}
\label{sec:iter-refin-monte}

The methods presented in \S~\ref{sec:monte-carlo-matrix} are
characterized by slow convergence rates.
Halton~\cite{halton_1962,halton_1994} proposed a staged residual
scheme called Sequential Monte Carlo to speed up the convergence of
these methods.  A variant of this scheme has been successfully applied
to the 1D nonlinear thermal radiation diffusion equation in
Ref.~\cite{evans_2003}.

Concisely, the Sequential Monte Carlo method solves
Eq.~(\ref{eq:Ax=b}) using the adjoint solution technique described in
\S~\ref{sec:adjoint-method}.  The following iteration scheme is
applied:
\begin{subequations}
  \begin{gather}
    \vr^{l+1} = \vb - \vA\vx^{l}\:, \\
    \vA\delta\vx^{l+1} = \vr^{l+1}\:, \label{eq:residual_solve}\\
    \vx^{l+1} = \vx^{l} + \delta\vx^{l+1}\:.
  \end{gather}
\end{subequations}
In this scheme, the adjoint method is used to estimate the solution to
Eq.~(\ref{eq:residual_solve}), and the residual is iterated to
convergence.  This iteration sequence is closely related to iterative
refinement; the exception being that there is no update to to
$x^{l+1}$ between residual iterations.

We propose a modification of this scheme that uses Monte Carlo as a
synthetic-acceleration scheme for fixed-point iteration. The
Fixed-Point Monte Carlo Synthetic-Acceleration (MCSA) method is
defined as follows:
\begin{subequations}
  \begin{gather}
    \vx^{l+1/2} = (\vI - \vA)\vx^l + \vb\:, \\
    %%
    \vr^{l+1/2} = \vb - \vA\vx^{l+1/2}\:,\\
    %%
    \hat{\vA}\delta\vx^{l+1/2} = \vr^{l+1/2}\:,
    \label{eq:MCSA-residual_solve}\\ 
    %%
    \vx^{l+1}=\vx^{l+1/2}+\delta\vx^{l+1/2}\:.
  \end{gather}
  \label{eq:general-MCSA}
\end{subequations}
The hat on $\vA$ in Eq.~(\ref{eq:MCSA-residual_solve}) indicates that
the Monte Carlo solution only approximately inverts this operator.
However, $\hat{\vA}$ is the same operator as $\vA$;
$\hat{\vA}\equiv\vA$.  Thus, we have defined a scheme in which the
initial estimate of $\vx$ in each iteration is updated using a single
fixed-point iteration\footnote{Although, a Krylov method could be used
  here.}.  The residual is calculated and is used as a source to
estimate the error, $\delta\vx^{l+1/2}$, by solving
Eq.~(\ref{eq:MCSA-residual_solve}) via the Monte Carlo adjoint method.
The error is used to calculate an updated iterate of the solution
vector, $\vx^{l+1}$.  The entire sequence is iterated to convergence
based on the following stopping criterion~\cite{kelley_1995},
\begin{equation}
  \|\vr\|_{\infty} < \epsilon\cdot\|\vb\|_\infty\:.
  \label{eq:stopping-criteria}
\end{equation}

%%---------------------------------------------------------------------------%%
\section{Discrete Form of the Radiation Diffusion Equation}
\label{sec:discr-form-radi}

\subsection{Equilibrium Diffusion Model}

The general form of the equilibrium diffusion equation is
\cite{morel_1996}
\begin{equation} 
  (\rho\Cv + 4aT^{3})\pder{T}{t} -
  \nabla\cdot{\Bigl(\frac{4acT^{3}}{3\ros}\Bigr)\grad{T}} = Q\:,
  \label{eq:eq-diff-T}
\end{equation}
where $\Cv\equiv\Cv(\ve{r},T(\ve{r},t))$
$[\text{Jerks}\cdot\text{g}^{-1}\cdot\text{keV}^{-1}]$ is the specific
heat capacity of the material, $\rho\equiv\rho(\ve{r})$
$[\text{g}\cdot\text{cm}^{-3}]$ is the density of the material,
$a=0.01372$ $[\text{Jerks}\cdot\text{cm}^{-3}\cdot\text{keV}^{-4}]$ is
the radiation constant, $c=299.79$ $[\text{cm}\cdot\text{sh}^{-1}]$ is
the vacuum light speed, and $T\equiv T(\ve{r},t)$ $[\text{keV}]$ is
the temperature that characterizes both the radiation and the
material.  The opacity, $\ros\equiv\ros$(\ve{r},T(\ve{r},t))
$[\text{cm}^{-1}]$, is the Rosseland Mean opacity.  The source,
$Q\equiv Q(\ve{r},t)$
$[\text{Jerks}\cdot\text{cm}^{-3}\cdot\text{sh}^{-1}]$, is assumed to
be independent of $T$.

We chose to use the angle-energy integrated radiation intensity,
$\phi\equiv\phi(\ve{r},t)$
$[\text{Jerks}\cdot\text{cm}^{-2}\cdot\text{sh}^{-1}]$, as the primary
state variable.  The intensity is defined,
\begin{equation}
  \label{eq:lte-phi}
  \begin{split}
    \phi(\ve{r},t) &=
    \int_{4\pi}d\bOmega\int_{\nu}d\nu\:\psi(\ve{r},\nu,\hat{\bOmega},t) \\
    &= \int_{\nu}d\nu\: 4\pi B(\nu,T) \\
    &= acT^{4}\:,
  \end{split}
\end{equation}
where $\psi$ is the angular radiation intensity.  The Planck function
(or Planckian) is defined as,
\begin{equation}
  B(\nu,T)=\frac{2h\nu^{3}}{c^{2}}\bigl(e^{\frac{h\nu}{kT}} -
  1\bigr)^{-1}\:,
\end{equation}
where $h$ is Planck's constant and $k$ is the Boltzmann constant.
Utilizing this definition of $\phi$ we can write
\begin{equation}
  \label{eq:partial-T}
  \partial T = \frac{1}{4acT^{3}}\partial\phi\:.
\end{equation}
Inserting \eqref{eq:partial-T} into \eqref{eq:eq-diff-T} yields
\begin{equation}
  \label{eq:eq-diff}
  \Bigl(\frac{\Cv}{4acT^{3}} + \frac{1}{c}\Bigr)\pder{\phi}{t} - 
  \nabla\cdot D\grad{\phi} = Q\:,
\end{equation}
where $D$ is the diffusion coefficient, $D\equiv
D(\ve{r},T(\ve{r},t))\ [\text{cm}]$, and is defined
\begin{equation}
  D = \frac{1}{3\ros}\:.
\end{equation}
Equation~\eqref{eq:eq-diff} is the model equation we will solve using
the Monte Carlo techniques presented in
\S~\ref{sec:monte-carlo-matrix} and \ref{sec:iter-refin-monte}.

\subsection{Discrete Diffusion Equation}

Applying backward-Euler time differencing to Eq.~(\ref{eq:eq-diff})
and lagging the temperature-dependent coefficients at $t^n$ gives
\begin{equation}
  -\nabla\cdot \Dn\grad{\phi} + \sign\phi = \qn\:,
\end{equation}
where we have suppressed the $n+1$ indices on $\phi$ and 
\begin{align}
  \sign &= \frac{\rho \Cv^n}{4ac(\Tn)^3\dt} + \frac{1}{c\dt}\:,\\
  \qn &= \sign\phin + Q^n\:,\\
  \phin &= ac(\Tn)^4\:.
\end{align}
Applying Fick's Law,
\begin{equation}
  \ve{F} = -D\grad{\phi}\:,
  \label{eq:fick}
\end{equation}
where $\ve{F}$ is the radiation flux, we define a flux-balance
equation,
\begin{equation}
  \dive{F} + \sign\phi = \qn\:.
  \label{eq:flux-balance}
\end{equation}
Integrating Eq.~(\ref{eq:flux-balance}) over the cell illustrated in
Fig.~\ref{fig:cell} gives
\begin{equation}
  \iiint\dive{F}\,d\ve{r} +
  \sign_{ijk}\phi_{ijk}V_{ijk}=\qn_{ijk}V_{ijk}\:,
  \label{eq:flux-balance-vol-integral}
\end{equation}
and $V_{ijk} = \Delta_i\Delta_j\Delta_k$.  We now apply the divergence
theorem,
\begin{equation}
  \begin{split}
    \iiint\dive{F}\,d\ve{r} &= \oint\hat{\ve{n}}\cdot\ve{F}\,dA\\
    &= \sum_{f=1}^{6}\hat{\ve{n}}_f\cdot\ve{F}_f A_f\:,
  \end{split}
\end{equation}
to Eq.(\ref{eq:flux-balance-vol-integral}) yielding
\begin{equation}
  \begin{split}
    (F_{i+1/2} - F_{i-1/2})\Dj\Dk + (F_{j+1/2} &- F_{j-1/2})\Di\Dk \\
    &+ (F_{k+1/2} - F_{k-1/2})\Di\Dj
    + \sign_{ijk}\phi_{ijk}V_{ijk} = \qn_{ijk}V_{ijk}\:.
  \end{split}
  \label{eq:flux-balance-difference}
\end{equation}

Using Fick's Law, Eq.~(\ref{eq:fick}), we can write $O(\Delta^2)$
expressions for the face-centered fluxes.  Defining $l\in\{i,j,k\}$,
the face-centered fluxes are
\begin{align}
  F_{l+1/2} &= -D_{l+1/2}\frac{\phi_{l+1} -
    \phi_{l}}{\Delta_{l+1/2}}\:, \label{eq:F_l+1/2}\\
  F_{l-1/2} &= -D_{l-1/2}\frac{\phi_{l} -
    \phi_{l-1}}{\Delta_{l-1/2}}\:, \label{eq:F_l-1/2}
\end{align}
where
\begin{align}
  \Delta_{l+1/2} &= \frac{\Delta_{l+1} + \Delta_{l}}{2}\:,\\
  \Delta_{l-1/2} &= \frac{\Delta_{l} + \Delta_{l-1}}{2}\:.
\end{align}
In order to be conservative, the flux must be continuous across a
face.  Thus, the face-centered flux evaluated from either side of the
face must be equivalent,
\begin{equation}
  \begin{aligned}
    F_{l+1/2}^{+} &= F_{l+1/2}^{-}\:,\\
    2D_{l+1}\frac{\phi_{l+1} - \phi_{l+1/2}}{\Delta_{l+1}} &=
    2D_{l}\frac{\phi_{l+1/2} - \phi_{l}}{\Delta_l}\:,\\
    \frac{1}{3\sigma_{l+1}}\frac{\phi_{l+1} - \phi_{l+1/2}}{\Delta_{l+1}} &=
    \frac{1}{3\sigma_{l}}\frac{\phi_{l+1/2} - \phi_{l}}{\Delta_l}\:,
  \end{aligned}
\end{equation}
where we have dropped the subscript $\mathrm{R}$ from $\sigma$ for
clarity.  Solving this system for $\phi_{l+1/2}$ and plugging into
either $F_{l+1/2}^{+}$ or $F_{l+1/2}^{-}$ and setting the resulting
expression equal to Eq.~(\ref{eq:F_l+1/2}), one finds the
face-centered diffusion coefficient,
\begin{equation}
  D_{l+1/2} = \frac{2\Delta_{l+1/2}}{3\sigma_l\Delta_l + 
    3\sigma_{l+1}\Delta_{l+1}}\:.
  \label{eq:D_l+1/2}
\end{equation}
Following the same procedure at the $l-1/2$ face gives the diffusion
coefficient on the low-side face,
\begin{equation}
  D_{l-1/2} = \frac{2\Delta_{l-1/2}}{3\sigma_l\Delta_l + 
    3\sigma_{l-1}\Delta_{l-1}}\:.
  \label{eq:D_l-1/2} 
\end{equation}
Using Eqs.~(\ref{eq:F_l+1/2}), (\ref{eq:F_l-1/2}), (\ref{eq:D_l+1/2}),
and (\ref{eq:D_l-1/2}) in the discrete flux-balance equation,
(\ref{eq:flux-balance-difference}), we obtain
\begin{equation}
  \begin{aligned}
    \sigT\phi_{ijk} 
    - \sigm_{i+1}\frac{\Delta_{i+1}}{\Di}\phi_{i+1\,jk} &
    - \sigp_{i-1}\frac{\Delta_{i-1}}{\Di}\phi_{i-1\,jk}
    - \sigm_{j+1}\frac{\Delta_{j+1}}{\Dj}\phi_{i\,j+1\,k} \\
    &- \sigp_{j-1}\frac{\Delta_{j-1}}{\Dj}\phi_{i\,j-1\,k} 
    - \sigm_{k+1}\frac{\Delta_{k+1}}{\Dk}\phi_{ij\,k+1}
    - \sigp_{k-1}\frac{\Delta_{k-1}}{\Dk}\phi_{ij\,k-1}
    = \qn_{ijk}\:.
  \end{aligned}
  \label{eq:discrete-diffusion}
\end{equation}
Here,
\begin{equation}
  \sigT = \sigp_i + \sigm_i + \sigp_j + \sigm_j + \sigp_k + \sigm_k
  + \sign_{ijk}\:,
\end{equation}
and
\begin{subequations}
  \begin{align}
    \sigp_{i} &= \frac{1}{\Di}\frac{2}{3\sigma_{ijk}\Di
      + 3\sigma_{i+1\,jk}\Delta_{i+1}}\:,\\
    \sigm_{i} &= \frac{1}{\Di}\frac{2}{3\sigma_{ijk}\Di
      + 3\sigma_{i-1\,jk}\Delta_{i-1}}\:,\\
    \sigp_{j} &= \frac{1}{\Dj}\frac{2}{3\sigma_{ijk}\Dj
      + 3\sigma_{i\,j+1\,k}\Delta_{j+1}}\:,\\
    \sigm_{j} &= \frac{1}{\Dj}\frac{2}{3\sigma_{ijk}\Dj
      + 3\sigma_{i\,j-1\,k}\Delta_{j-1}}\:,\\
    \sigp_{k} &= \frac{1}{\Dk}\frac{2}{3\sigma_{ijk}\Dk
      + 3\sigma_{ij\,k+1}\Delta_{k+1}}\:,\\
    \sigm_{k} &= \frac{1}{\Dk}\frac{2}{3\sigma_{ijk}\Dk
      + 3\sigma_{ij\,k-1}\Delta_{k-1}}\:. 
  \end{align}
  \label{eq:sigmas}
\end{subequations}
Also, note that
\begin{align*}
  \sigp_{i-1} &= \frac{1}{\Delta_{i-1}}\frac{2}{3\sigma_{ijk}\Di
    + 3\sigma_{i-1\,jk}\Delta_{i-1}}\:,\\
  \sigm_{i+1} &= \frac{1}{\Delta_{i+1}}\frac{2}{3\sigma_{ijk}\Di
    + 3\sigma_{i+1\,jk}\Delta_{i+1}}\:,\\
  \ldots\:.
\end{align*}
Equations~(\ref{eq:discrete-diffusion}) through (\ref{eq:sigmas})
represent the discretized, 1T diffusion equation on an orthogonal
3D grid. 

Equations~(\ref{eq:discrete-diffusion}) through (\ref{eq:sigmas}) have
been written in a manner that is amenable to Monte Carlo
interpretation.  One can envision the terms in Eq.~(\ref{eq:sigmas})
representing leakage out of each face of a cell.  The off-diagonal
terms then represent sources into the cell, or, equivalently, these
terms are leakage out of the neighboring cells.  A limitation of this
interpretation is that it requires a symmetric system to implement the
Monte Carlo scheme.  More specifically, the operator $\ve{D}$ must be
an H-matrix \cite{kelley_1995} such that the off-diagonal elements are
negative and symmetric.  In other words, the leakage out of a cell
must be equivalent to the source into the adjacent cell.

The methods described in \S\S~\ref{sec:monte-carlo-matrix} and
\ref{sec:iter-refin-monte} do not suffer this limitation.  Thus, we
will interpret the solution via MCSA as a transport-like process, but
we do not require any constraints on the system other than the
spectral radius must be less than 1. 

Before completing the derivation of discrete equations required to
solve Eq.~(\ref{eq:discrete-diffusion}), we must specify boundary
conditions.  For this work, the Marshak Boundary condition suffices,
\begin{equation}
  f_b^\pm = \frac{1}{4}\phi - \frac{1}{2}\ve{F}\cdot\hat{\ve{n}}\:,\quad
  \vr\in \partial\Gamma\:,
\end{equation}
where $f_b$ is the incoming partial current on the problem boundary,
$\partial\Gamma$.  On any low boundary we can write the flux using a
first-order discrete form of Fick's Law,
\begin{equation}
  F_{1/2} = -\frac{2}{3\sigma_1}\frac{\phi_1 - \phi_{1/2}}{\Delta_1}\:.
  \label{eq:low-boundary-flux}
\end{equation}
As before, we enforce continuity of flux on the boundary using the
Marshak Boundary condition,
\begin{equation}
  -\frac{2}{3\sigma_1}\frac{\phi_1 - \phi_{1/2}}{\Delta_1} = 2f_b^{+}
  -\frac{1}{2}\phi_{1/2}\:. 
\end{equation}
Solving for $\phi_{1/2}$ and plugging into
Eq.~(\ref{eq:low-boundary-flux}) yields an expression for the low-side
boundary flux,
\begin{equation}
  F_{1/2} = \frac{2}{3\sigma_1\Delta_1 + 4}(4f_b^{+}-\phi_1)\:.
\end{equation}
Following an identical procedure on high-side problem boundaries gives
\begin{equation}
  F_{N_l+1/2} = \frac{2}{3\sigma_{N_l}\Delta_{N_l} +
    4}(\phi_{N_l}-4f_b^{-})\:, 
\end{equation}
where $l\in\{i,j,k\}$ is defined over the range $[1,N_l]$.

These boundary fluxes take the place of the interior face-centered
fluxes (defined in Eqs.~(\ref{eq:F_l+1/2}) and (\ref{eq:F_l-1/2})) in
Eq.~(\ref{eq:flux-balance-difference}) yielding the following
modifications to Eqs.~(\ref{eq:sigmas}) on a boundary,
\begin{align}
  \text{low boundary:}\quad & \sigma_1^{-} = \frac{1}{\Delta_1}
  \frac{2}{3\sigma_1\Delta_1 + 4}\:,\\
  \text{high boundary:}\quad & \sigma_{N_l}^{+} = \frac{1}{\Delta_{N_l}}
  \frac{2}{3\sigma_{N_l}\Delta_{N_l} + 4}\:.
\end{align}
Furthermore, the following contributions are added to the source,
$\qn_{ijk}$ on a boundary cell,
\begin{align}
  \text{low boundary:}\quad & \frac{8}{\Delta_1}
  \frac{1}{3\sigma_1\Delta_1 + 4}f_b^{+}\:,\\
  \text{high boundary:}\quad & \frac{8}{\Delta_{N_l}}
  \frac{1}{3\sigma_{N_l}\Delta_{N_l} + 4}f_b^{-}\:.
\end{align}
Note that for vacuum boundary conditions, $f_b^{\pm} = 0$.

For reflective boundary conditions we have the simple modification
that
\begin{align}
  F_{1/2} &= 0\:, \\
  F_{N_l+1/2} &= 0\:.
\end{align}
Thus, there are no leakage or adjacent cell source terms resulting
from this boundary face.

\subsection{Preconditioning}
\label{sec:preconditioning}

The discrete diffusion equation in (\ref{eq:discrete-diffusion}) can
be written in matrix-operator form as follows,
\begin{equation}
  \ve{D}\bphi = \ve{q}\:,
  \label{eq:diffusion-matrix}
\end{equation}
where $\ve{D}$ is a symmetric, positive-definite (SPD) operator.
Unfortunately, for many problems $\rho(\ve{D}) > 1$.  Therefore, we
must precondition Eq.~(\ref{eq:diffusion-matrix}) in order to use the
MCSA method.  Preconditioning accomplishes two purposes: it reduces
the spectral radius of $\ve{D}$ so that Monte Carlo solution methods
are applicable, and it reduces the number of iterations required for
convergence.  We apply left-preconditioning and write
\begin{equation}
  \ve{M}^{-1}\ve{D}\bphi = \ve{M}^{-1}\ve{q}\:.
  \label{eq:preconditioned-diffusion}
\end{equation}
A good choice for the preconditioner, $\ve{M}$, is
\begin{equation}
  \ve{M}=\diag(\ve{D})\:.
  \label{eq:preconditioner}
\end{equation}
Setting $\ve{M}$ equal to the diagonal elements of $\ve{D}$
accomplishes several objectives:
\begin{itemize}
\item It reduces the spectral radius such that
  $\rho(\ve{M}^{-1}\ve{D}) < \rho(\ve{D})$ and
  $\rho(\ve{M}^{-1}\ve{D}) < 1$.  Thus, convergence is guaranteed and
  fewer iterations are required resulting in accelerated convergence.
\item Makes the diagonal elements of the iteration matrix zero,
  $\diag(\vI - \ve{M}^{-1}\ve{D}) = 0$.  Thus, there will be no time
  wasted in transitions $i\rightarrow i$ during the random walk
  process.
\end{itemize}
The second item is quite important for Monte Carlo solutions of linear
systems, and this type of scaling should be performed regardless of
any additional preconditioner choices.  

A side effect of this preconditioner choice is that the preconditioned
system, $\ve{M}^{-1}\ve{D}$, is no longer SPD.  This situation can be
rectified by doing right-left preconditioning and defining
\begin{equation}
  \ve{M} = \ve{L}\ve{L}^{T}\:,
\end{equation}
where $\ve{L} = \ve{L}^{T} = \sqrt{\ve{M}}$.  Applying right-left
preconditioning gives
\begin{equation}
  \begin{gathered}
    \ve{M}^{-1}\ve{D}\ve{M}^{-1}\ve{M}\bphi = \ve{M}^{-1}\ve{q}\:,\\
    (\ve{LL}^T)^{-1}\ve{D}(\ve{LL}^T)^{-1}\ve{LL}^T\bphi =
    (\ve{LL}^T)^{-1}\ve{q}\:, \\
    \ve{L}^{-T}\ve{L}^{-1}\ve{D}\ve{L}^{-T}\ve{L}^T\phi =
    \ve{L}^{-T}\ve{L}^{-1}\ve{q}\:,\\ 
    (\ve{L}^{-1}\ve{D}\ve{L}^{-T})\ve{L}^T\bphi = \ve{L}^{-1}\ve{q}\:,\\
    \hat{\ve{D}}\hat{\bphi} = \hat{\ve{q}}\:.
  \end{gathered}
  \label{eq:right-left-preconditioning}
\end{equation}
This type of preconditioning is also referred to as row-column
scaling.  The resulting preconditioned matrix, $\hat{\ve{D}}$, is now
SPD, and we can use Conjugate Gradient (CG) solvers.  Therefore, we
will use this preconditioning scheme for deterministic solutions of
Eq.~(\ref{eq:diffusion-matrix}).  However, given that the convergence
properties of $\hat{\ve{D}}$ are similar to $\ve{M}^{-1}\ve{D}$ and
the Monte Carlo solvers do not require SPD matrices, we will be
content with solving Eq.~(\ref{eq:preconditioned-diffusion}) using
MCSA.

The preconditioned system we will solve is defined in
Eq.~(\ref{eq:preconditioned-diffusion}).  Applying $\ve{M}^{-1}$ to
Eq.~(\ref{eq:discrete-diffusion}) yields
\begin{equation}
  \begin{aligned}
    \phi_{ijk} 
    - \frac{\sigm_{i+1}}{\sigT}\frac{\Delta_{i+1}}{\Di}&\phi_{i+1\,jk}
    - \frac{\sigp_{i-1}}{\sigT}\frac{\Delta_{i-1}}{\Di}\phi_{i-1\,jk}
    - \frac{\sigm_{j+1}}{\sigT}\frac{\Delta_{j+1}}{\Dj}\phi_{i\,j+1\,k} \\
    &- \frac{\sigp_{j-1}}{\sigT}\frac{\Delta_{j-1}}{\Dj}\phi_{i\,j-1\,k} 
    - \frac{\sigm_{k+1}}{\sigT}\frac{\Delta_{k+1}}{\Dk}\phi_{ij\,k+1}
    - \frac{\sigp_{k-1}}{\sigT}\frac{\Delta_{k-1}}{\Dk}\phi_{ij\,k-1}
    = \frac{\qn_{ijk}}{\sigT}\:.
  \end{aligned}
  \label{eq:preconditioned-discrete-diffusion}
\end{equation}
In the next section we will derive an MCSA method to solve this
equation.

%%---------------------------------------------------------------------------%%
\section{Solution of the Radiation Diffusion Equation}
\label{sec:solut-radi-diff}

Here we derive the equations required to solve
Eq.~(\ref{eq:preconditioned-discrete-diffusion}) using the MCSA
method.  As discussed in \S~\ref{sec:iter-refin-monte}, the MCSA
method requires two solution steps:
\begin{enumerate}
\item A fixed-point iteration to estimate $\bphi^{l+1/2}$.  This is a
  matrix-vector multiply operation.
\item An adjoint Monte Carlo solve to estimate $\delta\bphi^{l+1/2}$.
\end{enumerate}
For clarity, we rewrite the iteration scheme in
Eqs.~(\ref{eq:general-MCSA}) using the diffusion operator notation
from Eq.~(\ref{eq:preconditioned-diffusion}),
\begin{align*}
  %%
  \bphi^{l+1/2} = (\vI - \ve{M}^{-1}\ve{D})\bphi^{l} +
  \ve{M}^{-1}\ve{q}\:, \quad & (\text{fixed-point iteration})\\
  %%
  \vr^{l+1/2} = \ve{M}^{-1}\ve{q} - \ve{M}^{-1}\ve{D}\bphi^{l+1/2}\:,
  \quad & (\text{compute residual})\\
  %%
  \ve{M}^{-1}\ve{D}\delta\bphi^{l+1/2} = \vr^{l+1/2}\:, \quad&
  (\text{estimate $\delta\phi$ using adjoint Monte Carlo method})\\
  %%
  \bphi^{l+1} = \bphi^{l+1/2} + \delta\bphi^{l+1/2}\:. \quad&
  (\text{update}\ \bphi)
\end{align*}
As described in \S~\ref{sec:iter-refin-monte}, the adjoint Monte Carlo
method only approximately inverts $\ve{M}^{-1}\ve{D}$.  We now
describe each of these steps in turn.

First, we use Eq.~(\ref{eq:preconditioned-discrete-diffusion}) to
evaluate $\bphi^{l+1/2}$ via a single fixed-point iteration as
follows:
\begin{equation}
  \begin{aligned}
    \phi_{ijk}^{l+1/2} =  
    \frac{\sigm_{i+1}}{\sigT}\frac{\Delta_{i+1}}{\Di}&\phi_{i+1\,jk}^l
    + \frac{\sigp_{i-1}}{\sigT}\frac{\Delta_{i-1}}{\Di}\phi_{i-1\,jk}^l
    +
    \frac{\sigm_{j+1}}{\sigT}\frac{\Delta_{j+1}}{\Dj}\phi_{i\,j+1\,k}^l \\ 
    &+ \frac{\sigp_{j-1}}{\sigT}\frac{\Delta_{j-1}}{\Dj}\phi_{i\,j-1\,k}^l 
    + \frac{\sigm_{k+1}}{\sigT}\frac{\Delta_{k+1}}{\Dk}\phi_{ij\,k+1}^l
    + \frac{\sigp_{k-1}}{\sigT}\frac{\Delta_{k-1}}{\Dk}\phi_{ij\,k-1}^l
    + \frac{\qn_{ijk}}{\sigT}\:.
  \end{aligned}
\end{equation}
This result is used to compute the residual of the system.  The
residual then becomes the source for the Monte Carlo simulation.

The adjoint Monte Carlo method described in
\S~\ref{sec:adjoint-method} is used to estimate
\begin{equation}
  \ve{M}^{-1}\ve{D}\delta\bphi^{l+1/2} = \vr^{l+1/2}\:.
\end{equation}
As noted above, the residual acts as the source for this simulation,
and the operator $\ve{M}^{-1}\ve{D}$ is implicitly defined in
Eq.~(\ref{eq:preconditioned-discrete-diffusion}).  There are two Monte
Carlo interpretations that can be applied to this system.  The first
follows the mathematical presentation given in
\S~\ref{sec:monte-carlo-matrix}.  Namely, we form probabilities from
the iteration matrix and perform random walks to generate unbiased
estimates of the solution vector, $\delta\bphi^{l+1/2}$, using the
estimator in Eq.~(\ref{eq:adjoint-tally}).

However, a more natural interpretation for Monte Carlo transport
practitioners is to use the machinery described in
\S~\ref{sec:monte-carlo-matrix} to define Probability Distribution
Functions (PDF) that describe the transport of particles between
cells.  We then do a Monte Carlo transport calculation using
Eq.~(\ref{eq:adjoint-weight}) to calculate the weight change at each
collision. Equation~(\ref{eq:adjoint-probability}) gives the
probability of scattering to an adjacent cell. We tally the
contribution of $\phi$ in each cell using
Eq.~(\ref{eq:adjoint-tally}).  The transport is terminated using a
relative weight cutoff that is defined in
Eq.~(\ref{eq:weight_cutoff}).  We now examine each of these steps in
more detail.

The connectivity of an arbitrary cell is illustrated in the 3D mesh
segment in Fig.~\ref{fig:mesh_segment}.  For a particle residing in
the blue cell, any collision results in a scatter to one of the six
adjacent cells since $p_{ii}=0$ in the preconditioned system.  We can
easily calculate the probabilities for each scattering event from
Eq.~(\ref{eq:preconditioned-discrete-diffusion}) without explicitly
forming the iteration matrix.  In this mesh cell, the probabilities
for leakage out of each face are derived from the equations from each
adjacent cell.  Using Eq.~(\ref{eq:preconditioned-discrete-diffusion})
we can explicitly write the equations from each adjacent cell,
\begin{align*}
  \phi_1 &= \frac{\sigm_{7\rightarrow
      1}}{\sigma_{T\,1}}\frac{\Delta_7}{\Delta_1}\phi_7 + \ldots\:,\\
  %%
  \phi_2 &= \frac{\sigp_{7\rightarrow
      2}}{\sigma_{T\,2}}\frac{\Delta_7}{\Delta_2}\phi_7 + \ldots\:,\\
  %%
  \phi_3 &= \frac{\sigm_{7\rightarrow
      3}}{\sigma_{T\,3}}\frac{\Delta_7}{\Delta_3}\phi_7 + \ldots\:,\\
  %%
  \phi_4 &= \frac{\sigp_{7\rightarrow
      4}}{\sigma_{T\,4}}\frac{\Delta_7}{\Delta_4}\phi_7 + \ldots\:,\\
  %%
  \phi_5 &= \frac{\sigm_{7\rightarrow
      5}}{\sigma_{T\,5}}\frac{\Delta_7}{\Delta_5}\phi_7 + \ldots\:,\\
  %%
  \phi_6 &= \frac{\sigp_{7\rightarrow
      6}}{\sigma_{T\,6}}\frac{\Delta_7}{\Delta_6}\phi_7 + \ldots\:,\\
  %%
  \ldots\:.
\end{align*} 
A quick view of these equations reveals that the only non-zero entries
in column 7 of the iteration matrix are
\begin{equation*}
  \begin{pmatrix}
    h_{1,7} &
    h_{2,7} &
    h_{3,7} &
    h_{4,7} &
    h_{5,7} &
    h_{6,7} 
  \end{pmatrix}^{T}
  = 
  \begin{pmatrix}
    \frac{\sigm_{7\rightarrow 1}}{\sigma_{T\,1}}\frac{\Delta_7}{\Delta_1} \\
    \frac{\sigp_{7\rightarrow 2}}{\sigma_{T\,2}}\frac{\Delta_7}{\Delta_2} \\
    \frac{\sigm_{7\rightarrow 3}}{\sigma_{T\,3}}\frac{\Delta_7}{\Delta_3} \\
    \frac{\sigp_{7\rightarrow 4}}{\sigma_{T\,4}}\frac{\Delta_7}{\Delta_4} \\
    \frac{\sigm_{7\rightarrow 5}}{\sigma_{T\,5}}\frac{\Delta_7}{\Delta_5} \\
    \frac{\sigp_{7\rightarrow 6}}{\sigma_{T\,6}}\frac{\Delta_7}{\Delta_6}
  \end{pmatrix}\:,
\end{equation*}
where, following Eq.~(\ref{eq:point_iteration}), $\ve{H} = (\vI -
\ve{M}^{-1}\ve{D})$ is the iteration matrix.  Now, using
Eq.~(\ref{eq:adjoint-probability}), we define the scattering
probabilities for cell 7 as
\begin{equation*}
  P_{7\rightarrow n} = p_{7,n} = \frac{|h_{n,7}|}{|h_{1,7} + 
    h_{2,7} + h_{3,7} + h_{4,7} + h_{5,7} + h_{6,7}|}\:,\quad
  n\in \{1,\ldots,6\}\:.
\end{equation*}
The weight change for each scattering event is calculated using
Eq.~(\ref{eq:adjoint-weight}),
\begin{equation*}
  W_{m} = \frac{h_{n,7}}{p_{7,n}}W_{m-1}\:,\quad
  n\in \{1,\ldots,6\}\:,
\end{equation*}
where $m$ is the step-index in the particle history.  We also note
that, for the 3D Cartesian mesh considered in this work, the maximum
number of non-zero entries in columns of the iteration matrix is 6.
Cells along the problem boundary can have fewer than 6 entries.

%%---------------------------------------------------------------------------%%
\section{Results}
\label{sec:results}

Here we examine the effectiveness of the MCSA method and compare it to
traditional deterministic solution methods.  The deterministic methods
considered are preconditioned Conjugate Gradient (PCG) and
preconditioned fixed-point (Richardson) iteration (PFIX).  The PCG
method uses right-left preconditioning a shown in
Eq.~(\ref{eq:right-left-preconditioning}).  The PFIX scheme is used to
solve Eq.~(\ref{eq:preconditioned-diffusion}), which is identical to
the system being solved with MCSA.  We examine each method on two 3D
problems: a Marshak wave problem and a multi-material duct problem.

\subsection{Marshak Wave}

The Marshak wave problem under consideration in this study has an
incoming 1~keV blackbody boundary flux at $x=0$ and propagates
in the $x$-dimension in time.  In other words, this a 3D
representation of a 1D problem.  The full set of problem conditions
is:
\begin{center}
  \begin{tabular}{ll}\hline
    %%
    B.C. & $f_b^{+}(x=0,T=1.0) = \frac{ac}{4}T^4$ \\
    & $\ve{F} = 0\quad \partial\Gamma \in \{x^{+}, y^{-}, y^{+},
    z^{-}, z^{+}\}$ \\
    %%
    material properties & $\rho = 3.0\,\text{g}\cdot\text{cm}^{-3}$ \\
    & $\Cv = 0.1\,\text{Jerks}\cdot\text{g}^{-1}\cdot\text{keV}^{-1}$\\
    & $T(t=0) = \sn{1}{-4}\,\text{keV}$\\
    %% 
    opacity & $\sigma = 100T^{-3}\,\text{cm}^2\cdot\text{g}^{-1}$ \\
    %%
    mesh & $\Di = \Dj = \Dk = 0.01\,\text{cm}$ \\
    & $N_i = 100\quad N_j = N_k = 4$ \\
    \hline
  \end{tabular}
\end{center}
The problem is run to $10$~shakes.

Figure~\ref{fig:marshak_10} shows the analytic Marshak Wave solution
compared to the MCSA solution.  As opposed to most Monte Carlo
techniques, the MCSA method generates solutions that are numerically
equivalent to the PCG and PFIX results.  Because the MCSA method
converges the residual of the system, the numerical accuracy is
determined by the stopping criterion shown in
Eq.~(\ref{eq:stopping-criteria}).  Also, the MCSA method is not bound
by the Central Limit Theorem so convergence is not restricted to
$1/\sqrt{N}$~\cite{halton_1994,evans_2003}.

As stated in \S~\ref{sec:monte-carlo-matrix}, the MCSA method will
converge when $\rho(\ve{H})< 1$ where $\ve{H}$ is the iteration matrix
and is defined $\ve{H} = (\vI - \ve{M}^{-1}\ve{D})$.  The spectral
radius for the Marshak problem is plotted in
Fig.~\ref{fig:marshak_spectral}.  As stated in
\S~\ref{sec:preconditioning} the preconditioning effectively reduces
the spectral radius of $\ve{H}$ to less than 1.

There are two principal degrees of freedom when utilizing the MCSA
method: (1) the number of particles per stage, and (2) the weight
cutoff.  Figure~\ref{fig:CPU_Np} shows the CPU time as a function of
the requested number of particles per stage for a relative weight
cutoff of $\sn{1}{-4}$.  For less than 5 particles per stage teh
method does not converge.  Figure~\ref{fig:CPU_Ns} shows the variation
in the maximum number of stages (iterations) per cycle with number of
particles.  Analysis of these two figures shows that increasing the
number of particles per stage reduces the number of iterations
required to converge the solution.  However, the cost of the Monte
Carlo transport is high; thus, better performance is attained by
running more stages with fewer particles per stage \cite{evans_2003}.

Figure~\ref{fig:CPU_wc} shows the CPU time as a function of relative
weight cutoff.  These results indicate that the performance of the
MCSA method is relatively insensitive to the weight cutoff.  This is
especially true when running small numbers of particles per stage.
When larger numbers of particles are used the weight cutoff has a more
pronounced effect on the efficiency of the method.

Having analyzed the variation of the MCSA algorithm efficiency with
number of particles and weight cutoff, we now compare to traditional
deterministic algorithms.  Table~\ref{tab:marshak_comparision} shows
timing results for the Marshak problem using preconditioned CG (PCG),
preconditioned Richardson iteration (PFIX), and MCSA. The results show
that the MCSA method is the most efficient method, although the PCG
results are nearly as fast.  More importantly, the MCSA method clearly
accelerates Richardson iteration, as indicated by the fact that MCSA
is 42\% faster than Richards iteration alone.

\subsection{Multi-material Problem}

We now test the performance of the MCSA method on a large
multi-material problem.  The multi-material problem has a dog-legged
duct through a thick wall where the radiation flows into a thin region
bound by a foil on one side.  The mesh and geometry are illustrated in
Fig.~\ref{fig:multi-mat-mesh}.  A blackbody boundary flux is defined
on the low-$x$ face at $T=0.5$~keV.  The full set of problem
conditions is:
\begin{center}
  \begin{tabular}{ll}\hline
    %%
    B.C. & $f_b^{+}(x=0,T=0.5) = \frac{ac}{4}T^4$\\
    & $\ve{F} = 0\quad \partial\Gamma \in \{x^{+}, y^{-}, y^{+},
    z^{-}, z^{+}\}$ \\
    %%
    material properties 
    & $\rho = 1.5\,\text{g}\cdot\text{cm}^{-3}$ (duct) \\
    & $\rho = 8.0\,\text{g}\cdot\text{cm}^{-3}$ (wall) \\
    & $\rho = 4.0\,\text{g}\cdot\text{cm}^{-3}$ (foil) \\
    & $\Cv = 0.1\,\text{Jerks}\cdot\text{g}^{-1}\cdot\text{keV}^{-1}$\\
    & $T(t=0) = \sn{1}{-4}\,\text{keV}$\\
    %% 
    opacity 
    & $\sigma = T^{-3}\,\text{cm}^2\cdot\text{g}^{-1}$ (duct) \\
    & $\sigma = 1000T^{-1}\,\text{cm}^2\cdot\text{g}^{-1}$ (wall) \\
    & $\sigma = 5T^{-2}\,\text{cm}^2\cdot\text{g}^{-1}$ (foil) \\
    %%
    mesh & $\Di = \Dj = \Dk = 0.025\,\text{cm}$ \\
    & $N_i =  N_j = N_k = 60$ \\
    \hline
  \end{tabular}
\end{center}
A 2D plot of the temperature at 100~sh elapsed time is illustrated in
Fig.~\ref{fig:multi-mat_T}.  The time-evolution of the temperature is
shown at 4 edit points in Fig.~\ref{fig:edit-points}.  All three
methods, PCG, PFIX, and MCSA, give numerically identical answers when
using a stopping criterion of $\sn{1}{-8}$.

Table~\ref{tab:multimat_comparision} shows the timing results for the
multi-material problem using PCG, PFIX, and MCSA.  The results
correspond roughly with the results from the Marshak problem.  The
MCSA is marginally faster than PCG and 43\% faster than PFIX.

%%---------------------------------------------------------------------------%%
\section{Conclusions}
\label{sec:conclusions}

We have presented a new Monte Carlo solution method for solving the
discrete, time-dependent diffusion equation in 3D.  The MCSA method
has been shown to match results using standard solution techniques to
arbitrary precision.  Also, the new method is faster than
preconditioned CG and Richardson iteration.

While we have demonstrated marginal improvements over standard
solution schemes in this study, significant improvements could be
realized in fully nonlinear-consistent solutions.  In these cases, the
MCSA method competes with GMRES, which is more costly in memory and
time than CG.  Another area where the MCSA scheme may have advantages
over traditional solution methods is on dentritic, or adaptive,
meshes.  These meshes yield matrices with poor condition numbers
because of the changing cell volumes at different refinement levels.
A smart transport algorithm could be developed that more efficiently
solves the residual on these types of meshes.  These topics will be
the focus of future study.


%%---------------------------------------------------------------------------%%
%% The Appendices part is started with the command \appendix;
%% appendix sections are then done as normal sections
%% \appendix

%% \section{}
%% \label{}

%% References
%%
%% Following citation commands can be used in the body text:
%% Usage of \cite is as follows:
%%   \cite{key}         ==>>  [#]
%%   \cite[chap. 2]{key} ==>> [#, chap. 2]
%%

%%---------------------------------------------------------------------------%%
%% References with bibTeX database:

\bibliographystyle{elsarticle-num}
\bibliography{jcomp2012.bib}

%% Authors are advised to submit their bibtex database files. They are
%% requested to list a bibtex style file in the manuscript if they do
%% not want to use elsarticle-num.bst.

%% References without bibTeX database:

% \begin{thebibliography}{00}

%% \bibitem must have the following form:
%%   \bibitem{key}...
%%

% \bibitem{}

% \end{thebibliography}


\end{document}

%%
%% End of file `elsarticle-template-num.tex'.
