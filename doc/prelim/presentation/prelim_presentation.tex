\documentclass{beamer}
\usetheme[white]{Wisconsin}
\usepackage{longtable}
\usepackage{listings}
\usepackage{color}
%% The amssymb package provides various useful mathematical symbols
\usepackage{amssymb}
%% The amsthm package provides extended theorem environments
\usepackage{amsthm} \usepackage{amsmath} \usepackage{tmadd,tmath}
\usepackage[mathcal]{euscript} \usepackage{color}
\usepackage{textcomp}
\definecolor{listinggray}{gray}{0.9}
\definecolor{lbcolor}{rgb}{0.9,0.9,0.9}
\lstset{
  backgroundcolor=\color{lbcolor},
  tabsize=4,
  rulecolor=,
  language=c++,
  basicstyle=\scriptsize,
  upquote=true,
  aboveskip={1.5\baselineskip},
  columns=fixed,
  showstringspaces=false,
  extendedchars=true,
  breaklines=true,
  prebreak =
  \raisebox{0ex}[0ex][0ex]{\ensuremath{\hookleftarrow}},
  frame=single,
  showtabs=false,
  showspaces=false,
  showstringspaces=false,
  identifierstyle=\ttfamily,
  keywordstyle=\color[rgb]{0,0,1},
  commentstyle=\color[rgb]{0.133,0.545,0.133},
  stringstyle=\color[rgb]{0.627,0.126,0.941},
}

%%---------------------------------------------------------------------------%%
\newcommand{\vA}{\ensuremath{\ve{A}}}
\newcommand{\vb}{\ensuremath{\ve{b}}}
\newcommand{\vx}{\ensuremath{\ve{x}}}
\newcommand{\vr}{\ensuremath{\ve{r}}}
\newcommand{\vu}{\ensuremath{\ve{u}}}
\newcommand{\vI}{\ensuremath{\ve{I}}}
\newcommand{\vH}{\ensuremath{\ve{H}}}
\newcommand{\vP}{\ensuremath{\ve{P}}}

\newcommand{\Cv}{\ensuremath{C_{v}}}
\newcommand{\ros}{\ensuremath{\sigma_{\scriptscriptstyle\mathrm{R}}}}
\newcommand{\bOmega}{\ensuremath{\mathbf{\Omega}}}

\newcommand{\dt}{\ensuremath{\Delta t}}
\newcommand{\sign}{\ensuremath{\tilde{\sigma}^n}}
\newcommand{\qn}{\ensuremath{q^n}} \newcommand{\Tn}{\ensuremath{T^n}}
\newcommand{\Dn}{\ensuremath{D^n}}
\newcommand{\phin}{\ensuremath{\phi^n}}
\newcommand{\Di}{\ensuremath{\Delta_i}}
\newcommand{\Dj}{\ensuremath{\Delta_j}}
\newcommand{\Dk}{\ensuremath{\Delta_k}}

\newcommand{\sigT}{\ensuremath{\sigma_{T\,ijk}}}
\newcommand{\sigm}{\ensuremath{\sigma^{-}}}
\newcommand{\sigp}{\ensuremath{\sigma^{+}}}

\newcommand{\bphi}{\ensuremath{\boldsymbol{\phi}}}

\DeclareMathOperator{\diag}{diag}
%%---------------------------------------------------------------------------%%

\author{Stuart R. Slattery
  \\ Engineering Physics Department
  \\ University of Wisconsin - Madison
  \\ \href{mailto:sslattery@wisc.edu}{\texttt{sslattery@wisc.edu}}
}

\date{\today}
\title{Massively Parallel Monte Carlo Methods for Linear and Nonlinear
Problems}
\begin{document}
\maketitle

%%---------------------------------------------------------------------------%%

\begin{frame}{My Research and Development Activities}

  \begin{itemize}
  \item Stochastic computational linear algebra.
  \item Multiphysics methods for reactor physics.
  \item High performance computing.
  \item Software Engineering.
  \end{itemize}

\end{frame}

%%---------------------------------------------------------------------------%%
\begin{frame}{Data Transfer Kit (DTK)}

  DTK is a software component for generating and applying parallel
  topology map operators based on multiphysics parameters. Some
  requirements:

  \begin{itemize}
  \item A component based design not limited to specific mesh or
    field data structures
  \item Compatible with both mesh-based and mesh-free physics codes
  \item Based on load-balanced rendezvous algorithms of a desirable
    time complexity
  \item Operate on $O(1,000,000,000)$ element meshes
  \item Operate on $O(100,000)$ cores
  \item Integrate readily with current CASL physics components
  \item Open-source implementation (DTK has a BSD 3-clause license)
  \end{itemize}

\end{frame}

%%---------------------------------------------------------------------------%%
\begin{frame}{Stochastic Linear and Nonlinear Solvers}

  Ph.D. Research: Massively Parallel Stochastic Methods for Linear and
  Nonlinear Problems.

  \begin{itemize}
  \item Moving to exascale facilities requires methods tolerant to
    both soft and hard errors
  \item Solver scalability of primary concern
  \item Difficult to form the Jacobian analytically for block operator
    problems, Jacobian-free methods are too coarse or do not perform
    well for poorly scaled problems
  \item Subspace (e.g. Krylov) methods are memory intensive
  \item Stochastic solvers are a novel research area that may bear
    fruit
  \item We can leverage bleeding-edge Monte Carlo methods in reactor
    physics for parallelism
  \end{itemize}

\end{frame}

%%---------------------------------------------------------------------------%%
\begin{frame}{MCSA and FANM}

  \textbf{Monte Carlo Synthetic Acceleration (MCSA)}: A stochastic
  solution method for sparse linear problems.

  \begin{subequations}
    \label{eq:MCSA-iteration}
    \begin{gather}
      \vx^{l+1/2} = (\vI - \vA)\vx^l + \vb\:,\\ \vr^{l+1/2} = \vb -
      \vA\vx^{l+1/2}\:,
      \label{eq:MCSA-residual}\\     
      \hat{\vA}\delta\vx^{l+1/2} = \vr^{l+1/2}\:,
      \label{eq:MCSA-residual_solve}\\ 
      \vx^{l+1}=\vx^{l+1/2}+\delta\vx^{l+1/2}\:.
    \end{gather}
  \end{subequations}

  \textbf{Forward-Automated Newton-MCSA (FANM)}: A stochastic solution
  method for nonlinear problems

  \begin{itemize}
  \item Form the Jacobian explicitly with forward automatic
    differentiation. (No nasty derivatives, no Jacobian
    approximations, compressed row storage cheaper than the subspace
    over many iterations)
  \item Use MCSA as the linear solver. (All the benefits of the
    stochastic solver)
  \end{itemize}

\end{frame}

%%---------------------------------------------------------------------------%%
\begin{frame}{General MCSA Preliminary Results}

  Transient Poisson equation

  \begin{figure}[htpb!]
    \begin{center}
      \includegraphics[width=4in]{basic_results.png}
    \end{center}
  \end{figure}
  
  \begin{itemize}
  \item MCSA implemented in C++ using Epetra linear algebra
    framework. General linear algebra implemenation possible with
    MCSA!
  \item Competitive and GMRES and Conjugate Gradient solvers.
  \end{itemize}

  An initial FANM implementation is close and will be based on the
  Epetra solver. Parallel MCSA will be later in the fall per
  development of the Chimera code base.

\end{frame}

%%---------------------------------------------------------------------------%%

\end{document}


