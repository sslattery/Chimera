\documentclass{beamer}
\usetheme[white]{Wisconsin}
\usepackage{longtable}
\usepackage{listings}
\usepackage{color}
%% The amssymb package provides various useful mathematical symbols
\usepackage{amssymb}
%% The amsthm package provides extended theorem environments
\usepackage{amsthm} \usepackage{amsmath} \usepackage{tmadd,tmath}
\usepackage[mathcal]{euscript} \usepackage{color}
\usepackage{textcomp}
\definecolor{listinggray}{gray}{0.9}
\definecolor{lbcolor}{rgb}{0.9,0.9,0.9}
\lstset{
  backgroundcolor=\color{lbcolor},
  tabsize=4,
  rulecolor=,
  language=c++,
  basicstyle=\scriptsize,
  upquote=true,
  aboveskip={1.5\baselineskip},
  columns=fixed,
  showstringspaces=false,
  extendedchars=true,
  breaklines=true,
  prebreak =
  \raisebox{0ex}[0ex][0ex]{\ensuremath{\hookleftarrow}},
  frame=single,
  showtabs=false,
  showspaces=false,
  showstringspaces=false,
  identifierstyle=\ttfamily,
  keywordstyle=\color[rgb]{0,0,1},
  commentstyle=\color[rgb]{0.133,0.545,0.133},
  stringstyle=\color[rgb]{0.627,0.126,0.941},
}

%%---------------------------------------------------------------------------%%
\newcommand{\vA}{\ensuremath{\ve{A}}}
\newcommand{\vb}{\ensuremath{\ve{b}}}
\newcommand{\vx}{\ensuremath{\ve{x}}}
\newcommand{\vr}{\ensuremath{\ve{r}}}
\newcommand{\vu}{\ensuremath{\ve{u}}}
\newcommand{\vI}{\ensuremath{\ve{I}}}
\newcommand{\vH}{\ensuremath{\ve{H}}}
\newcommand{\vP}{\ensuremath{\ve{P}}}

\newcommand{\Cv}{\ensuremath{C_{v}}}
\newcommand{\ros}{\ensuremath{\sigma_{\scriptscriptstyle\mathrm{R}}}}
\newcommand{\bOmega}{\ensuremath{\mathbf{\Omega}}}

\newcommand{\dt}{\ensuremath{\Delta t}}
\newcommand{\sign}{\ensuremath{\tilde{\sigma}^n}}
\newcommand{\qn}{\ensuremath{q^n}} \newcommand{\Tn}{\ensuremath{T^n}}
\newcommand{\Dn}{\ensuremath{D^n}}
\newcommand{\phin}{\ensuremath{\phi^n}}
\newcommand{\Di}{\ensuremath{\Delta_i}}
\newcommand{\Dj}{\ensuremath{\Delta_j}}
\newcommand{\Dk}{\ensuremath{\Delta_k}}

\newcommand{\sigT}{\ensuremath{\sigma_{T\,ijk}}}
\newcommand{\sigm}{\ensuremath{\sigma^{-}}}
\newcommand{\sigp}{\ensuremath{\sigma^{+}}}

\newcommand{\bphi}{\ensuremath{\boldsymbol{\phi}}}

\DeclareMathOperator{\diag}{diag}
%%---------------------------------------------------------------------------%%

\author{Stuart R. Slattery
  \\ Engineering Physics Department
  \\ University of Wisconsin - Madison
  \\ \href{mailto:sslattery@wisc.edu}{\texttt{sslattery@wisc.edu}}
}

\date{\today}
\title{Massively Parallel Monte Carlo Methods for Linear and Nonlinear
Problems}
\begin{document}
\maketitle

%%---------------------------------------------------------------------------%%

\begin{frame}{Introduction}

\end{frame}

%%---------------------------------------------------------------------------%%
\begin{frame}{Physics-Based Motivation}

\end{frame}

%%---------------------------------------------------------------------------%%
\begin{frame}{Hardware-Based Motivation}

\end{frame}

%%---------------------------------------------------------------------------%%
\begin{frame}{Research Outline}

\end{frame}

%%---------------------------------------------------------------------------%%
\begin{frame}{Conventional Solution Methods for Linear Problems}

\end{frame}

%%---------------------------------------------------------------------------%%
\begin{frame}{Linear Problem Preliminaries}

\end{frame}

%%---------------------------------------------------------------------------%%
\begin{frame}{Stationary Methods}

\end{frame}

%%---------------------------------------------------------------------------%%
\begin{frame}{Projection Methods}

\end{frame}

%%---------------------------------------------------------------------------%%
\begin{frame}{Parallel Projection Methods}

\end{frame}

%%---------------------------------------------------------------------------%%
\begin{frame}{Monte Carlo Solution Methods for Linear Problems}

\end{frame}

%%---------------------------------------------------------------------------%%
\begin{frame}{Monte Carlo Linear Solver Preliminaries}

\end{frame}

%%---------------------------------------------------------------------------%%
\begin{frame}{Direct Method}

\end{frame}

%%---------------------------------------------------------------------------%%
\begin{frame}{Adjoint Method}

\end{frame}

%%---------------------------------------------------------------------------%%
\begin{frame}{Sequential Monte Carlo}

\end{frame}

%%---------------------------------------------------------------------------%%
\begin{frame}{Monte Carlo Synthetic-Acceleration}

\end{frame}

%%---------------------------------------------------------------------------%%
\begin{frame}{Parallelization of Stochastic Methods}

\end{frame}

%%---------------------------------------------------------------------------%%
\begin{frame}{Monte Carlo Solution Methods for Nonlinear Problems}

\end{frame}

%%---------------------------------------------------------------------------%%
\begin{frame}{Monte Carlo Nonlinear Solver Preliminaries}

\end{frame}

%%---------------------------------------------------------------------------%%
\begin{frame}{Inexact Newton Methods}

\end{frame}

%%---------------------------------------------------------------------------%%
\begin{frame}{The FANM Method}

\end{frame}

%%---------------------------------------------------------------------------%%
\begin{frame}{Research Proposal}

\end{frame}

%%---------------------------------------------------------------------------%%
\begin{frame}{Experimental Framework}

\end{frame}

%%---------------------------------------------------------------------------%%
\begin{frame}{Progress to Date}

\end{frame}

%%---------------------------------------------------------------------------%%
\begin{frame}{Monte Carlo Methods Verification}

\end{frame}

%%---------------------------------------------------------------------------%%
\begin{frame}{Proposed Numerical Experiments}

\end{frame}

%%---------------------------------------------------------------------------%%
\begin{frame}{Proposed Challenge Problem}

\end{frame}

%%---------------------------------------------------------------------------%%
\begin{frame}{Conclusion}

\end{frame}

%%---------------------------------------------------------------------------%%

\end{document}


