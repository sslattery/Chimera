\documentclass{beamer}
\usetheme[white]{Wisconsin}
\usepackage{longtable}
\usepackage{listings}
\usepackage{color}
%% The amssymb package provides various useful mathematical symbols
\usepackage{amssymb}
%% The amsthm package provides extended theorem environments
\usepackage{amsthm} \usepackage{amsmath} \usepackage{tmadd,tmath}
\usepackage[mathcal]{euscript} \usepackage{color}
\usepackage{textcomp}
\definecolor{listinggray}{gray}{0.9}
\definecolor{lbcolor}{rgb}{0.9,0.9,0.9}
\lstset{
  backgroundcolor=\color{lbcolor},
  tabsize=4,
  rulecolor=,
  language=c++,
  basicstyle=\scriptsize,
  upquote=true,
  aboveskip={1.5\baselineskip},
  columns=fixed,
  showstringspaces=false,
  extendedchars=true,
  breaklines=true,
  prebreak =
  \raisebox{0ex}[0ex][0ex]{\ensuremath{\hookleftarrow}},
  frame=single,
  showtabs=false,
  showspaces=false,
  showstringspaces=false,
  identifierstyle=\ttfamily,
  keywordstyle=\color[rgb]{0,0,1},
  commentstyle=\color[rgb]{0.133,0.545,0.133},
  stringstyle=\color[rgb]{0.627,0.126,0.941},
}

%% colors
\setbeamercolor{boxheadcolor}{fg=white,bg=UWRed}
\setbeamercolor{boxbodycolor}{fg=black,bg=white}


%%---------------------------------------------------------------------------%%
\author{Stuart R. Slattery
  \\ Engineering Physics Department
  \\ University of Wisconsin - Madison
}

\date{\today} 
\title{Massively Parallel Monte Carlo Methods for Discrete Linear and
  Nonlinear Systems} 
\begin{document}
\maketitle

%%---------------------------------------------------------------------------%%
\begin{frame}{Introduction}

  \begin{itemize}
    \item Predictive modeling and simulation enhances engineering
      capability
    \item Modern work focused on this task leverages multiple physics
      simulation (CASL, NEAMS)
    \item New hardware drives algorithm development (petascale and
      exascale)
    \item Monte Carlo methods have the potential to provide great
      improvements that permit finer simulations and better mapping to
      future hardware
    \item A set of massively parallel Monte Carlo methods is proposed
      to advance multiple physics simulation on contemporary and
      future leadership class machines
  \end{itemize}

\end{frame}

%%---------------------------------------------------------------------------%%
\begin{frame}{Physics-Based Motivation}
 
  \begin{beamerboxesrounded}[upper=boxheadcolor,lower=boxbodycolor,shadow=true]
    {Predictive nuclear reactor analysis enables...}
    \begin{itemize}
    \item Tighter design tolerance for improved thermal performance
      and efficiency
    \item Higher fuel burn-up
    \item High confidence in accident scenario models
    \end{itemize}
  \end{beamerboxesrounded}

  \pause 
  \begin{beamerboxesrounded}[upper=boxheadcolor,lower=boxbodycolor,shadow=true]
  {Multiple physics simulations are complicated...}
    \begin{itemize}
    \item Neutronics, thermal hydraulics, computational fluid
      dynamics, structural mechanics, and many other physics
    \item Consistent models yield nonlinearities in the variables
      through feedback effects
    \item Tremendous computational resources are required with
      $O(\sn{1}{9})$ element meshes and $O(100,000)+$ cores used in
      today's simulations.
    \end{itemize}
  \end{beamerboxesrounded}

\end{frame}

%%---------------------------------------------------------------------------%%
\begin{frame}{Physics-Based Motivation: DNB}

  \begin{columns}

    \begin{column}{0.35\textwidth}
      \begin{figure}[htpb!]
        \begin{center}
          \scalebox{1}{ \input{dnb_schematic.pdftex_t} }
        \end{center}
        \caption{\textbf{Departure from nucleate boiling scenario.} }
      \end{figure}
    \end{column}

    \begin{column}{0.65\textwidth}
      \begin{figure}[htpb!]
        \begin{center}
          \scalebox{0.8}{ \input{dnb_example.pdftex_t} }
        \end{center}
        \caption{\textbf{Multiphysics dependency analysis of departure
            from nucleate boiling.} }
      \end{figure}
    \end{column}

  \end{columns}

\end{frame}

%%---------------------------------------------------------------------------%%
\begin{frame}{Hardware-Based Motivation}
  \begin{itemize}
    \item Modern hardware is moving in two directions: lightweight
      machines and heterogeneous machines characterized by low power
      and high concurrency.
    \item High concurrency and low cost units means a higher potential
      for both soft and hard failures.
    \item Monte Carlo methods bury soft failures within the variance
      of the tallies while hard failures are high variance events.
    \item New machines will also be memory restricted with a continued
      decrease of memory/FLOPS predicted.
    \item Compared to conventional methods, we aim show that Monte
      Carlo methods offer a memory savings.
  \end{itemize}
\end{frame}

%%---------------------------------------------------------------------------%%
\begin{frame}{Research Outline}
  \begin{itemize}
    \item Parallelization of classic Monte Carlo methods.
    \item Parallel strategies taken from modern reactor physics
      methods in Monte Carlo.
    \item Research is required to explore how domain decomposition
      patterns and the discrete system properties are related.
    \item Research is required to explore how these methods perform
      in modern multiple physics simulations where strong
      nonlinearities are present.
  \end{itemize}
\end{frame}

%%---------------------------------------------------------------------------%%
\begin{frame}{The Linear Problem}

  We will seek solutions of the general linear problem:

  \[
  \ve{A} \ve{x} = \ve{b}\:,
  \]
  \[
  \ve{A} \in \mathbb{R}^{N \times N},\ \ve{A} : \mathbb{R}^{N}
  \rightarrow \mathbb{R}^{N},\ \ve{x} \in \mathbb{R}^N,\ \ve{b} \in
  \mathbb{R}^N\:.
  \]

  We will assert that $\ve{A}$ is \textit{nonsingular}. The solution
  is then:
  \[
  \ve{x} = \ve{A}^{-1}\ve{b}\:.
  \]

  We can also define the residual of the system as:
  \[
  \ve{r} = \ve{b} - \ve{A}\ve{x}\:,
  \]
  where $\ve{r}=\ve{0}$ when an exact solution is found.

\end{frame}

%%---------------------------------------------------------------------------%%
\begin{frame}{Stationary Methods}

  \begin{itemize}
  \item General stationary methods are formed by splitting the linear
    operator
  \end{itemize}

  \[
  \ve{A} = \ve{M} - \ve{N}\:.
  \]

  \[
  \ve{x} = \ve{M}^{-1}\ve{N}\ve{x} + \ve{M}^{-1}\ve{b}\:.
  \]

  We identify $\ve{H} =\ve{M}^{-1}\ve{N}$ as the \textit{iteration
    matrix}

  \[
  \ve{x}^{k+1} = \ve{H}\ve{x}^{k} + \ve{c}\:.
  \]

\end{frame}

%%---------------------------------------------------------------------------%%
\begin{frame}{Stationary Methods Convergence}

  \begin{itemize}
  \item The qualities of the iteration matrix dictate convergence 
  \item Define $\ve{e}^k = \ve{x}^k-\ve{x}$ as the error at the
    $k^{th}$ iterate
  \item The iteration error is generated by a recurrence relation
  \end{itemize}

  \[
  \ve{e}^{k+1} = \ve{H} \ve{e}^k\:
  \]


  \begin{itemize}
  \item We diagonalize $\ve{H}$ to extract its Eigenvalues
  \end{itemize}

  \[
  ||\ve{e}^{k}||_2 = \rho(\ve{H})^k ||\ve{e}^0||_2\:,
  \]

  \begin{itemize}
  \item We bound $\ve{H}$ by $\rho(\ve{H}) < 1$ for convergence
  \end{itemize}

\end{frame}

%%---------------------------------------------------------------------------%%
\begin{frame}{Projection Methods}

  We choose a \textit{search subspace} $\mathcal{K}$ and a
  \textit{constraint subspace} $\mathcal{L}$ and determine the solution
  to the linear problem by extracting the solution from the search
  subspace:
  \[
  \tilde{\ve{x}} = \ve{x}_0 +
  \boldsymbol{\delta},\ \boldsymbol{\delta} \in \mathcal{K}\:,
  \]
  and by constraining it with the constraint subspace:
  \[
  \langle \tilde{\ve{r}},\ve{w} \rangle = 0,\ \forall \ve{w} \in
  \mathcal{L}\:.
  \]

  We can generate a more physical and geometric-based understanding of
  these constraints by writing the new residual as:
  \[
  \tilde{\ve{r}} = \ve{r}_0 - \ve{A}\boldsymbol{\delta}
  \]

  If $\tilde{\ve{r}}$ is to be orthogonal to $\mathcal{L}$, then
  $\ve{A}\boldsymbol{\delta}$ must be the projection of $\ve{r}_0$
  onto the subspace $\mathcal{L}$ that eliminates the components of
  the residual that exist in $\mathcal{L}$.

\end{frame}

%%---------------------------------------------------------------------------%%
\begin{frame}{The Orthogonality Constraint}

  \begin{figure}[htpb!]
    \begin{center}
      \scalebox{1.25}{
        \input{orthogonal_residual.pdftex_t} }
    \end{center}
    \caption{\textbf{Orthogonality constraint of the new residual with
        respect to $\mathcal{L}$.} }
  \end{figure}

  \pause
  \begin{beamerboxesrounded}[upper=boxheadcolor,lower=boxbodycolor,shadow=true]
    {Minimization Property}

    The residual of the system will always be \textit{minimized} with
    respect to the constraints
    \[
    ||\tilde{\ve{r}}||_2 \leq ||\ve{r}_0||_2,\ \forall \ve{r}_0 \in
    \mathbb{R}^N\:,
    \]
  \end{beamerboxesrounded}

\end{frame}

%%---------------------------------------------------------------------------%%
\begin{frame}{Projection Iteration}

  Consider a matrix $\ve{V}$ to form a basis of $\mathcal{K}$ and a
  matrix $\ve{W}$ to form a basis of $\mathcal{L}$.

  \[
  \boldsymbol{\delta} = \ve{V}\ve{y},\ \forall \ve{y} \in
  \mathbb{R}^N\:.
  \]

  From the orthogonality constraint it then follows that:
  \[
  \ve{y} = (\ve{W}^T\ve{A}\ve{V})^{-1}\ve{W}^T\ve{r}_0\:,
  \]

  We can then form an iteration sequence:
  \[
  \ve{r}^k = \ve{b} - \ve{A}\ve{x}^k
  \]
  \[
  \ve{y}^k = (\ve{W}^T\ve{A}\ve{V})^{-1}\ve{W}^T\ve{r}^k
  \]
  \[
  \ve{x}^{k+1} = \ve{x}^k + \ve{V}\ve{y}^k\:,
  \]

  with $\ve{V}$ and $\ve{W}$ updated prior to each iteration.

\end{frame}

%%---------------------------------------------------------------------------%%
\begin{frame}{Krylov Subspace Methods}

\end{frame}

%%---------------------------------------------------------------------------%%
\begin{frame}{GMRES}

\end{frame}

%%---------------------------------------------------------------------------%%
\begin{frame}{Parallel Projection Methods}

\end{frame}

%%---------------------------------------------------------------------------%%
\begin{frame}{Monte Carlo Solution Methods for Linear Problems}

\end{frame}

%%---------------------------------------------------------------------------%%
\begin{frame}{Monte Carlo Linear Solver Preliminaries}

\end{frame}

%%---------------------------------------------------------------------------%%
\begin{frame}{Direct Method}

\end{frame}

%%---------------------------------------------------------------------------%%
\begin{frame}{Adjoint Method}

\end{frame}

%%---------------------------------------------------------------------------%%
\begin{frame}{Sequential Monte Carlo}

\end{frame}

%%---------------------------------------------------------------------------%%
\begin{frame}{Monte Carlo Synthetic-Acceleration}

\end{frame}

%%---------------------------------------------------------------------------%%
\begin{frame}{Parallelization of Stochastic Methods}

\end{frame}

%%---------------------------------------------------------------------------%%
\begin{frame}{Monte Carlo Solution Methods for Nonlinear Problems}

\end{frame}

%%---------------------------------------------------------------------------%%
\begin{frame}{Monte Carlo Nonlinear Solver Preliminaries}

\end{frame}

%%---------------------------------------------------------------------------%%
\begin{frame}{Inexact Newton Methods}

\end{frame}

%%---------------------------------------------------------------------------%%
\begin{frame}{The FANM Method}

\end{frame}

%%---------------------------------------------------------------------------%%
\begin{frame}{Research Proposal}

\end{frame}

%%---------------------------------------------------------------------------%%
\begin{frame}{Experimental Framework}

\end{frame}

%%---------------------------------------------------------------------------%%
\begin{frame}{Progress to Date}

\end{frame}

%%---------------------------------------------------------------------------%%
\begin{frame}{Monte Carlo Methods Verification}

\end{frame}

%%---------------------------------------------------------------------------%%
\begin{frame}{Proposed Numerical Experiments}

\end{frame}

%%---------------------------------------------------------------------------%%
\begin{frame}{Proposed Challenge Problem}

\end{frame}

%%---------------------------------------------------------------------------%%
\begin{frame}{Conclusion}

\end{frame}

%%---------------------------------------------------------------------------%%

\end{document}


