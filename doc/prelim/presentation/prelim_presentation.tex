\documentclass{beamer}
\usetheme[white]{Wisconsin}
\usepackage{longtable}
\usepackage{listings}
\usepackage{color}
%% The amssymb package provides various useful mathematical symbols
\usepackage{amssymb}
%% The amsthm package provides extended theorem environments
\usepackage{amsthm} \usepackage{amsmath} \usepackage{tmadd,tmath}
\usepackage[mathcal]{euscript} \usepackage{color}
\usepackage{textcomp}
\definecolor{listinggray}{gray}{0.9}
\definecolor{lbcolor}{rgb}{0.9,0.9,0.9}
\lstset{
  backgroundcolor=\color{lbcolor},
  tabsize=4,
  rulecolor=,
  language=c++,
  basicstyle=\scriptsize,
  upquote=true,
  aboveskip={1.5\baselineskip},
  columns=fixed,
  showstringspaces=false,
  extendedchars=true,
  breaklines=true,
  prebreak =
  \raisebox{0ex}[0ex][0ex]{\ensuremath{\hookleftarrow}},
  frame=single,
  showtabs=false,
  showspaces=false,
  showstringspaces=false,
  identifierstyle=\ttfamily,
  keywordstyle=\color[rgb]{0,0,1},
  commentstyle=\color[rgb]{0.133,0.545,0.133},
  stringstyle=\color[rgb]{0.627,0.126,0.941},
}

%%---------------------------------------------------------------------------%%
\author{Stuart R. Slattery
  \\ Engineering Physics Department
  \\ University of Wisconsin - Madison
  \\ \href{mailto:sslattery@wisc.edu}{\texttt{sslattery@wisc.edu}}
}

\date{\today}
\title{Massively Parallel Monte Carlo Methods for Linear and Nonlinear
Problems}
\begin{document}
\maketitle

%%---------------------------------------------------------------------------%%
\begin{frame}{Introduction}
  \begin{itemize}
    \item Predictive modeling and simulation capability yields nuclear
      system designs that are safer and better performing.
    \item Modern work focused on this task leverages multiple physics
      simulation (CASL, NEAMS).
    \item New hardware drives algorithm development (petascale and
      exascale).
    \item Monte Carlo methods have the potential to provide great
      improvements that permit finer simulations and better mapping to
      future hardware.
    \item A set of massively parallel Monte Carlo methods is proposed
      to advance multiple physics simulation on contemporary and
      future leadership class machines.
  \end{itemize}
\end{frame}

%%---------------------------------------------------------------------------%%
\begin{frame}{Physics-Based Motivation}
  \begin{itemize}
    \item Predictive nuclear reactor analysis enables tighter design
      tolerance for improved thermal performance and efficieny, higher
      fuel burn-up, and high confidence in accident scenario models.
    \item Neutronics, thermal hydraulics, computational fluid
      dynamics, and structural mechanics among other physics
      contribute to these predictive models.
    \item Couplings are complicated and consistent models yield
      nonlinearities in the variables through feedback effects.
    \item  Capturing  these   effects  requires  tremendous  resources
      with $O(\sn{1}{9})$ element meshes and $O(100,000)+$ cores used
      in today's simulations.
  \end{itemize}
\end{frame}

%%---------------------------------------------------------------------------%%
\begin{frame}{Physics-Based Motivation: DNB}
  \begin{figure}[htpb!]
    \begin{center}
      \scalebox{1}{ \input{dnb_example.pdftex_t} }
    \end{center}
    \caption{\textbf{Multiphysics dependency analysis of departure from
        nucleate boiling.} \textit{A neutronics solution is required to
        compute power generation in the fuel pins, fluid dynamics is
        required to characterize boiling and fluid temperature and
        density, heat transfer is required to compute the fuel and
        cladding temperature, and the nuclear data modified with the
        temperature and density data. Strong coupling among the
        variables creates strong nonlinearities.}}
    \label{fig:dnb_example}
  \end{figure}
\end{frame}

%%---------------------------------------------------------------------------%%
\begin{frame}{Hardware-Based Motivation}
  \begin{itemize}
    \item Modern hardware is moving in two directions: lightweight
      machines and heterogeneous machines characterized by low power
      and high concurrency.
    \item High concurrency and low cost units means a higher potential
      for both soft and hard failures.
    \item Monte Carlo methods bury soft failures within the variance
      of the tallies while hard failures are high variance events.
    \item New machines will also be memory restricted with a continued
      decrease of memory/FLOPS predicted.
    \item Compared to conventional methods, we will show that Monte
      Carlo methods offer a memory savings.
  \end{itemize}
\end{frame}

%%---------------------------------------------------------------------------%%
\begin{frame}{Research Outline}

\end{frame}

%%---------------------------------------------------------------------------%%
\begin{frame}{Conventional Solution Methods for Linear Problems}

\end{frame}

%%---------------------------------------------------------------------------%%
\begin{frame}{Linear Problem Preliminaries}

\end{frame}

%%---------------------------------------------------------------------------%%
\begin{frame}{Stationary Methods}

\end{frame}

%%---------------------------------------------------------------------------%%
\begin{frame}{Projection Methods}

\end{frame}

%%---------------------------------------------------------------------------%%
\begin{frame}{Parallel Projection Methods}

\end{frame}

%%---------------------------------------------------------------------------%%
\begin{frame}{Monte Carlo Solution Methods for Linear Problems}

\end{frame}

%%---------------------------------------------------------------------------%%
\begin{frame}{Monte Carlo Linear Solver Preliminaries}

\end{frame}

%%---------------------------------------------------------------------------%%
\begin{frame}{Direct Method}

\end{frame}

%%---------------------------------------------------------------------------%%
\begin{frame}{Adjoint Method}

\end{frame}

%%---------------------------------------------------------------------------%%
\begin{frame}{Sequential Monte Carlo}

\end{frame}

%%---------------------------------------------------------------------------%%
\begin{frame}{Monte Carlo Synthetic-Acceleration}

\end{frame}

%%---------------------------------------------------------------------------%%
\begin{frame}{Parallelization of Stochastic Methods}

\end{frame}

%%---------------------------------------------------------------------------%%
\begin{frame}{Monte Carlo Solution Methods for Nonlinear Problems}

\end{frame}

%%---------------------------------------------------------------------------%%
\begin{frame}{Monte Carlo Nonlinear Solver Preliminaries}

\end{frame}

%%---------------------------------------------------------------------------%%
\begin{frame}{Inexact Newton Methods}

\end{frame}

%%---------------------------------------------------------------------------%%
\begin{frame}{The FANM Method}

\end{frame}

%%---------------------------------------------------------------------------%%
\begin{frame}{Research Proposal}

\end{frame}

%%---------------------------------------------------------------------------%%
\begin{frame}{Experimental Framework}

\end{frame}

%%---------------------------------------------------------------------------%%
\begin{frame}{Progress to Date}

\end{frame}

%%---------------------------------------------------------------------------%%
\begin{frame}{Monte Carlo Methods Verification}

\end{frame}

%%---------------------------------------------------------------------------%%
\begin{frame}{Proposed Numerical Experiments}

\end{frame}

%%---------------------------------------------------------------------------%%
\begin{frame}{Proposed Challenge Problem}

\end{frame}

%%---------------------------------------------------------------------------%%
\begin{frame}{Conclusion}

\end{frame}

%%---------------------------------------------------------------------------%%

\end{document}


