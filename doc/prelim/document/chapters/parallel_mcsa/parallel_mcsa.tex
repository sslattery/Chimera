\chapter{Parallelization of the MCSA Algorithm for Linear Problems}
\label{ch:parallel_mcsa}

For the prelim, I suspect that I will not have a parallel MCSA
algorithm up and running as I will be focusing on generating the code
base needed to build the physics operators and tools for linear and
nonlinear systems analysis. Therefore, here we should focus on
motivating a parallel scheme for this work. We should start by first
looking at current parallel methods for solving linear systems using
the conventional techniques from the previous chapter. It must be
clearly pointed out that for the new hardware coming online, these
methods (or at least their current implementations) will be lacking in
scalability and the potential for resiliency, and therefore a parallel
MCSA is attractive, even if it ultimately does not show performance
gains from a wall time perspective.

Once we have done this, MSOD will need to be defined. The literature
is actually quite lacking here even for reactor physics problems as
this is really bleeding edge. Therefore, I should strive here to more
rigorously define this methodology in terms of solving general linear
systems. Furthermore, we should discuss the aspects of being able to
vary the overlap such that we can generate a spectrum between complete
domain decomposition and complete domain replication. Finally, the
beginnings of the algorithm should be sketched out.

\section{Current Parallel Methodologies for Solving Linear Systems}
\label{sec:current_parallel_methods}
Modern parallel implementations of linear solver methods rely heavily
on capabilities provided by general linear algebra frameworks. In
particular, Krylov methods that are formulated through matrix-vector
multiply operations, are easily formulated in parallel using these
frameworks.

\section{Multiple-Set Overlapping-Domain Decomposition}
\label{sec:msod}

\section{Algorithm Outline}
\label{sec:parallel_mcsa_algorithm_outline}

