\chapter{Introduction}
\label{ch:introduction}

In many fields of engineering and physics, linear and nonlinear
problems are a primary focus of study. Recent focus on multiple
physics systems (and in particular nuclear systems) adds a new level
of complexity to common nonlinear systems as solution strategies
change when they are coupled to other problems. Furthermore, a desire
for predictive simulations to enhance the safety and performance of
engineered systems creates a need for extremely high fidelity
computations to be performed for these coupled systems as a means to
capture effects not modeled by coarser methods.

In order to achieve this high fidelity, state-of-the-art computing
facilities must be leveraged in a way that is both efficient and
considerate of hardware-related issues. As scientific computing moves
towards exascale facilities with machines of $O(1,000,000)$ cores
already coming online, new algorithms to solve these complex problems
must be developed to leverage this new hardware. Issues such as
resiliency to node failure and scaling to large numbers of cores will
be pertinent to robust algorithms aimed at this new
hardware. Considering these issues, this thesis proposes the
development of a massively parallel stochastic method for linear
problems and a novel stochastic method to advance solution techniques
for nonlinear problems.

This preliminary report oultines the conventional methods used in
practice for solving linear and nonlinear problems to provide a
mathematical basis upon which to build new algorithms that aim to
solve some of the aforementioned issues. These new algorithms will be
outlined in full with links made to past work and their potential for
offering improvements to the computational physics community. Progress
to date will be reported including the development of a new physics
framework that will serve as a test bed for these new methods that
leverages a preliminary implementation of the stochastic method. A
research plan will then be provided that continues the development and
implementation of these new methods and verifies them with benchmark
solutions, culminating in the analysis of a multiphysics simulation of
a pressurized water reactor assembly.

\section{Motivation}
\label{sec:motivation}

For some time, the particle transport community has been utilizing
Monte Carlo methods for the solution of transport problems
\citep{lewis_1993}. The partial differential equation (PDE) community
has focused on various deterministic methods for solutions to linear
problems \citep{saad_2003, kelley_1995}. In between these two areas
are a not widely known small group of stochastic methods for solving
sparse linear systems \citep{hammersley_1964, halton_1962,
  halton_1994}. In recent years, these methods have been further
developed for transport problems in the form of Monte Carlo
Synthetic-Acceleration (MCSA) \citep{evans_2003, evans_2009} but have
yet to be applied to more general sparse linear systems commonly
generated by the computational physics community. Compared to other
methods in these regime, MCSA offers three attractive qualities; (1)
the linear problem operator need not be symmetric or
positive-definite, thereby reducing preconditioning complexity, (2)
the stochastic nature of the solution method provides a natural
solution to the issue of resiliency, and (3) is amenable to
parallelization using modern methods developed by the transport
community \citep{wagner_2011}. The developement of MCSA as a general
linear solver and the development of a parallel MCSA method will be
new and unique features of this work. Resiliency will not be
addressed.

In addition to linear solver advancements, nonlinear solvers may also
benefit from a general and parallel MCSA scheme. In the engineering
community, nonlinear problems are often addressed by either
linearizing the problem or building a segregated scheme and using
traditionaly iterative or direct methods to solve the resulting
system \citep{tannehill_1997}. In the mathematics community, various
Newton methods have been popular \citep{kelley_1995}. Recently,
Jacobian-Free Newton-Krylov (JFNK) schemes \citep{knoll_2004} have
been utilized in multiple physics architectures and advanced single
physics codes \citep{gaston_2009}. The benefits of JFNK schemes are
that the Jacobian is never formed, simplifying the implementation, and
a Krylov solver is leveraged (typically GMRES or Conjugate Gradient),
providing excellent convergence properties for well-conditioned and
well-scaled systems. However, there are two potential drawbacks to
these methods for high fidelity predictive simulations: (1) the
Jacobian is approximated by a first-order differencing method on the
order of machine precision such that this error can grow beyond that
of those in a fine-grained system and (2) for systems that are not
symmetric positive-definite (which will be the case for most
multiphysics systems and certainly for most preconditioned systems)
the Krylov subspace generated by the GMRES solver may become
prohibitively large. To address these issues, this thesis proposes a
new and novel method for nonlinear systems based on the MCSA method.

The Forward-Automated Newton-MCSA (FANM) method is proposed as new
nonlinear solution method. The key features of FANM are: full Jacobian
generation using modern Forward Automated Differencing (FAD) methods
\citep{bartlett_2006}, and MCSA as the inner linear solver. This method
has several attractive properties. First, the first-order
approximation to the Jacobian used in JFNK type methods is eliminated
by generating the Jacobian explicitly with the model equations through
FAD. Second, the Jacobian need not be explicitly formed by the user
but is instead automated through FAD; this eleminates the complexity
of hand-coding derivatives and has also been demonstrated to be more
efficient computationally than evaluating differenced derivatives
\citep{bartlett_2006}. Third, unlike GMRES, MCSA does not build a
subspace during iterations. Although the Jacobian must be explicitly
formed to use MCSA, for problems that take more than a few GMRES
iterations to converge the size of the Krylov subspace will grow
beyond that of the Jacobian. Finally, using MCSA for the linear solve
provides its benefits for preconditioning, potential resiliency, and
parallelism.

\section{The Multiphysics System}
\label{sec:multiphysics_system}

\section{Research Outline}
\label{sec:research_outline}
