\chapter{The Linear Problem}
\label{ch:linear_problem}

The purpose of this chapter is to provide a full background on solving
linear systems as it relates to this work. It may seem like there is a
lot here, but we really do need it to fully define our new methods and
compare both qualitatively and quantitatively to convential
methods. For example, we need a solid explanation of stationary
methods because MCSA is in fact an acceleration of a stationary method
and therefore shares many properties with them. Subspace methods are
the current class of methods most widely used for sparse systems and
are at the core of the Newton-Krylov methods that we will be comparing
the FANM method with. The most rigorous piece of this chapter should
of course be the stochastic solver definitions, also providing me a
place to include some of my work in this area. Finally,
preconditioning is important for this work, MCSA requires it and
Newton-Krylov methods almost always need some type of
preconditioning. Therefore, we must also discuss these aspects.

\section{Preliminaries}
\label{sec:linear_preliminaries}
We seek solutions of the general linear problem in the following form:
\begin{equation}
  \ve{A} \ve{x} = \ve{b}\:,
  \label{eq:linear_problem}
\end{equation}
where $\ve{A} \in \mathbb{R}^{N \times N}$ is a matrix operator such
that $\ve{A} : \mathbb{R}^{N} \rightarrow \mathbb{R}^{N}$, $\ve{x} \in
\mathbb{R}^N$ is the solution vector, and $\ve{b} \in \mathbb{R}^N$ is
the forcing term. The solutions to Eq~(\ref{eq:linear_problem}) will
be generated by inverting $\ve{A}$ either directly or indirectly:
\begin{equation}
  \ve{x} = \ve{A}^{-1} \ve{b}
  \label{eq:linear_problem_solution}\:.
\end{equation}
In addition we can define the residual:
\begin{equation}
  \ve{r} = \ve{b} - \ve{A}\ve{x}\:,
  \label{eq:linear_residual}
\end{equation}
such that an exact solution to $\ve{x}$ has been found when
$\ve{r}=\ve{0}$.  From the statement in
Eq~(\ref{eq:linear_problem_solution}) we can already place a
restriction on $\ve{A}$ by requiring that it be \textit{nonsingular},
meaning that we can in fact compute $\ve{A}^{-1}$. In this work we
will focus our efforts on approximately inverting the operator through
various means.

In a discussion of methods for solving linear systems, several
mathematical tools are useful in characterizing the qualities of the
linear system. Among the most useful are the \textit{Eigenvalues} of
the matrix, $\sigma(\ve{A})$. We find these by solving the Eigenvalue
problem:
\begin{equation}
  \ve{A} \ve{x} = \lambda \ve{x},\ \lambda \in \sigma(\ve{A})\:.
  \label{eq:eigenvalue_problem}
\end{equation}
By writing Eq~(\ref{eq:eigenvalue_problem}) in a different form,
\begin{equation}
  (\ve{A} - \lambda \ve{I})\ve{x} = 0 \:,
  \label{eq:eigenvalue_problem_2}
\end{equation}
and demanding that non-trivial solutions for $\ve{x}$ exist, it is
then required that $|\ve{A} - \lambda \ve{I}| = 0$. Expanding this
determinant yields a characteristic polynomial in terms of $\lambda$
with roots that form the set of Eigenvalues, $\sigma(\ve{A})$. Each
component of $\sigma(\ve{A})$ can then be used to solve
Eq~(\ref{eq:eigenvalue_problem_2}) for a particular permutation of
$\ve{x}$. The set of all permutations for the \textit{Eigenvectors} of
$\ve{A}$. A quantity of particular interest that is computatable from
the eigenvalues of a matrix is the \textit{spectral radius},
$\rho(\ve{A})$ defined by Saad \citep{saad_2003} as:
\begin{equation}
  \rho(\ve{A}) = \max_{\lambda \in \sigma(\ve{A})} |\lambda| \:.
  \label{eq:spectral_radius}
\end{equation}
In addition, for problems that have a large scale over which the
independent variables may exist (e.g. a problem with events on
timescales ranging from nanoseconds to hours), a good measure of this
range is supplied by the \textit{stiffness ratio}:
\begin{equation}
  Stiffness Ratio = \frac{\max_{\lambda \in \sigma(\ve{A})}
    |\lambda|}{\min_{\lambda \in \sigma(\ve{A})} |\lambda|}
\end{equation}
Those problems that have a wide range of scales in their independent
variables, which will then be reflected in the operator, will then
have a large stiffness ratio. We will define such problems with large
stiffness ratios as \textit{stiff}.

General to both matrices and vectors, \textit{norms} are a mechanism
for collapsing objects of many elements to a single value. Per
LeVeque's text \citep{leveque_2007}, the q-norm of a vector is defined
as:
\begin{equation}
  ||\ve{v}||_q = \Bigg[ \sum_{i=1}^N |v_i|^q \Bigg]^{1/q},\ \ve{v} \in
  \mathbb{R}^N\:,\ q \in \mathbb{Z}^+
  \label{eq:q_norm}
\end{equation}
where ${v_i}$ is the $i^{th}$ component of the vector. Depending on
the value chosen for $q$, local or global qualities of the vector may
be obtained. For example, $q=2$ provides the root of a quadrature sum
of all elements in the vector giving a global measure of the vector
while $q=\infty$ gives the maximum value in the vector, a local
quantity that does not give information regarding the other elements
in the vector.

We can also compute the norm of a matrix by inferring from the norm of
the vector on which it is operating. Per LeVeque, we search for a
constant that is equivalent to $||\ve{A}||$:
\begin{equation}
  ||\ve{A}\ve{x}|| \leq C ||\ve{x}||\:,
  \label{eq:matrix_norm_inequality}
\end{equation}
where the minimum value of $C$ that satisfies
Eq~(\ref{eq:matrix_norm_inequality}) is equivalent to $||\ve{A}||$
that is valid $\forall \ve{x} \in \mathbb{R}^N$. The general
definition in Eq~(\ref{eq:matrix_norm_inequality}) can be expanded in
simple terms for common norms including the infinity norm:
\begin{equation}
  ||\ve{A}||_{\infty} = \max_{1 \leq i \leq N} \sum^N_{j=1}|a_{ij}|\:,
  \label{eq:matrix_infinity_norm}
\end{equation}
and the 2-norm:
\begin{equation}
  ||\ve{A}||_{2} = \sqrt{\rho(\ve{A}^T\ve{A})}\:,
  \label{eq:matrix_2_norm}
\end{equation}
where $\rho$ is the spectral radius as defined in
Eq~(\ref{eq:spectral_radius}).

Knowing this, we can then define several useful properties of matrices
including the \textit{condition number} \citep{saad_2003}:
\begin{equation}
  \kappa(\ve{A}) = ||\ve{A}||\ ||\ve{A}^{-1}||\:,
  \label{eq:condition_number}
\end{equation}
which gives as a metric on assessing how close to singular the system
is. This is due to the fact $||\ve{A}^{-1}||$ is large near
singularities (and undefined for a singular matrix) and thus a large
condition number will be generated. We define such matrices as
\textit{ill-conditioned}. 

\section{Iterative Methods for Solving Sparse Linear Systems}
\label{sec:linear_methods}
The discretization of partial differential equations (\textit{PDEs})
through common methods such as finite differences
\citep{leveque_2007}, finite volumes \citep{leveque_2002}, and finite
elements \citep{zienkiewicz_1977} ultimately generate matrix problems
as in Eq~(\ref{eq:linear_problem}).  In addition, these matrices are
typically sparse, meaning that the vast majority of their constituent
elements are zero. This sparsity is due to the fact that the influence
of a particular grid element only expands as far as a few of its
nearest neighbors depending on the order of discretization used, and
therefore coupling among variables in a particular discrete equation
in the system leads to a few non-zero entries. Because of the natural
occurrence of sparse matrices in common numerical methods many
iterative techniques have been developed to solve such systems. We
discuss here conventional stationary and projection methods to provide
background and then give detail on a stochastic class of methods.

\subsection{Stationary Methods}
\label{subsec:stationary_methods}
Stationary methods arise from splitting the operator in
Eq~(\ref{eq:linear_problem})
\begin{equation}
  \ve{A} = \ve{M} - \ve{N}\:,
  \label{eq:split_linear_operator}
\end{equation}
where the choice of $\ve{M}$ and $\ve{N}$ will be dictated by the
particular method chosen. Using this split definition of the operator
we can then write:
\begin{equation}
  \ve{M}\ve{x} + \ve{N}\ve{x} = \ve{b}\:.
  \label{eq:linear_split_equation1}
\end{equation}
By rearranging, we can generate a form more useful for analysis:
\begin{equation}
  \ve{x} = \ve{H}\ve{x} + \ve{c}\:,
  \label{eq:linear_split_equation2}
\end{equation}
where $\ve{H}=\ve{M}^{-1}\ve{N}$ is defined as the \textit{iteration
  matrix} and $\ve{c}=\ve{M}^{-1}\ve{b}$. With the solution vector on
both the left and right hand sides, an iterative method can then be
formed:
\begin{equation}
    \ve{x}^{k+1} = \ve{H}\ve{x}^k + \ve{c}\:,
  \label{eq:linear_iterative_method}
\end{equation}
with $k \in \mathbb{Z}^+$ defined as the \textit{iteration index}. In
general, we will define methods in the form of
Eq~(\ref{eq:linear_iterative_method}) as \textit{stationary
  methods}. Given this, we can then generate a few statements
regarding the convergence of such stationary methods. Defining
$\ve{e}^k = \ve{u}^k - \ve{u}$ as the solution error at the
$k^{th}$ iterate, we can subtract Eq~(\ref{eq:linear_split_equation2})
from Eq~(\ref{eq:linear_iterative_method}) to arrive at an error form
of the linear problem:
\begin{equation}
  \ve{e}^{k+1} = \ve{H}\ve{e}^k\:. 
  \label{eq:linear_iterative_error}
\end{equation}
Our error after $k$ iterations is then:
\begin{equation}
  \ve{e}^{k} = \ve{H}^k\ve{e}^0\:. 
  \label{eq:linear_k_iter_error}
\end{equation}
In other words, successive application of the iteration matrix is the
mechanism driving down the error in a stationary method. We can then
place restrictions on the iteration matrix by using the tools
developed in \S~(\ref{sec:linear_preliminaries}). By assuming $\ve{H}$
is diagonalizable (although we may generalize this with the Jordan
canonical form of $H$ \citep{saad_2003}, we then have:
\begin{equation}
  \ve{e}^{k} =
  \ve{R}\boldsymbol{\Lambda}^k\ve{R}^{-1}\ve{e}^0\:,
  \label{eq:linear_k_iter_error_diag}
\end{equation}
where $\boldsymbol{\Lambda}$ contains the Eigenvalues of $\ve{H}$ on
its diagonal and the columns of $\ve{R}$ contain the Eigenvectors of
$\ve{H}$. Computing the 2-norm of the above form then gives:
\begin{equation}
  ||\ve{e}^{k}||_2 \leq ||\boldsymbol{\Lambda}^k||_2\ 
  ||\ve{R}||_2\ ||\ve{R}^{-1}||_2\ ||\ve{e}^0||_2\:,
  \label{eq:linear_k_iter_norm1}
\end{equation}
which gives:
\begin{equation}
  ||\ve{e}^{k}||_2 \leq \rho(\ve{H})^k \kappa(\ve{R})
  ||\ve{e}^0||_2\:.
  \label{eq:linear_k_iter_norm2}
\end{equation}
For iteration matrices where the Eigenvectors are orthogonal,
$\kappa(\ve{R})=0$ and the error bound reduces to:
\begin{equation}
  ||\ve{e}^{k}||_2 \leq \rho(\ve{H})^k
  ||\ve{e}^0||_2\:.
  \label{eq:linear_k_iter_norm3}
\end{equation}
We can now resrtict $\ve{H}$ by asserting that $\rho(\ve{H}) < 1$
for a stationary method to converge such that $k$ applications of the
iteration matrix will not cause the error to grow in
Eq~(\ref{eq:linear_k_iter_norm3}). 

\subsection{Projection Methods}
\label{subsec:projection_methods}
Among the most common iterative methods used in scientific computing
today for sparse systems are a broad class known as \textit{projection
  methods}. These methods not only provide access to more powerful
means of reaching a solution, but also a powerful means of
encapsulating the majority of common iterative methods including the
stationary methods just discussed in a common mathematical
framework. All projection methods are built around a core structure
where the solution to Eq~(\ref{eq:linear_problem}) is extracted from a
\textit{search subspace} $\mathcal{K}$ and bound by a
\textit{constraint subspace} $\mathcal{L}$ that will vary in
definition depending on the iterative method selected. We build the
approximate solution $\tilde{\ve{x}}$ by starting with an initial
guess $\ve{x}_0$ and extracting a correction $\boldsymbol{\delta}$
from $\mathcal{K}$ such that:
\begin{equation}
  \tilde{\ve{x}} = \ve{x}_0 +
  \boldsymbol{\delta},\ \boldsymbol{\delta} \in \mathcal{K}\:.
  \label{eq:linear_projection_step}
\end{equation}
We bound this correction by asserting that the new residual,
$\tilde{\ve{r}}$, be orthogonal to $\mathcal{L}$:
\begin{equation}
  \langle \tilde{\ve{r}},\ve{w} \rangle = 0,\ \forall \ve{w} \in
  \mathcal{L}\:.
  \label{eq:linear_projection_constraint}
\end{equation}

We can generate a more physical and geometric-based understanding of
these constraints by writing the new residual as $\tilde{\ve{r}} =
\ve{r}_0 - \ve{A}\boldsymbol{\delta}$ and again asserting the residual
must be orthogonal to $\mathcal{L}$. If $\tilde{\ve{r}}$ is to be
orthogonal to $\mathcal{L}$, then $\ve{A}\boldsymbol{\delta}$ must be
the projection of $\ve{r}_0$ onto the subspace $\mathcal{L}$ that
eliminates the components of the residual that exist in
$\mathcal{L}$. This situation is geometrically presented in
Figure~\ref{fig:linear_projection_constraint}.

\begin{figure}[htpb!]
  \begin{center}
    \scalebox{1.75}{
      \input{chapters/linear_problem/orthogonal_residual.pdftex_t} }
  \end{center}
  \caption{Orthogonality constraint of the new residual with respect
    to $\mathcal{L}$.}
  \label{fig:linear_projection_constraint}
\end{figure}

From Figure~\ref{fig:linear_projection_constraint} we then note that
the following geometric condition must hold:
\begin{equation}
  ||\tilde{\ve{r}}||_2 \leq ||\ve{r}_0||_2,\ \forall \ve{r}_0 \in
  \mathbb{C}^N\:,
  \label{eq:linear_min_res}
\end{equation}
meaning that the residual of the system will always be
\textit{minimized} with respect to the constraints.

Given this minimization condition for the residual, we can form the
outline of an iterative projection method. Consider a matrix $\ve{V}$
to form a basis of $\mathcal{K}$ and a matrix $\ve{W}$ to form a basis
of $\mathcal{L}$. As $\boldsymbol{\delta} \in \mathcal{K}$ by
definition in Eq~(\ref{eq:linear_projection_step}), then
$\boldsymbol{\delta}$ can instead be rewritten as:
\begin{equation}
  \boldsymbol{\delta} = \ve{V}\ve{y},\ \forall \ve{y} \in \mathbb{C}^N:\,
  \label{eq:linear_delta_projection}
\end{equation}
where $\ve{V}$ \textit{projects} $\ve{y}$ onto $\mathcal{K}$. From the
orthogonality constraint in Eq~(\ref{eq:linear_projection_constraint})
it then follows that:
\begin{equation}
  \ve{y} = (\ve{W}^T\ve{A}\ve{V})^{-1}\ve{W}^T\ve{r}_0\:,
  \label{eq:linear_constraint_projection}
\end{equation}
where here the projection onto $\mathcal{K}$ is constrained by the
projection onto $\mathcal{L}$. Knowing this, we can then outline the
following iteration scheme for a projection method:
\begin{subequations}
  \begin{gather}
    \ve{r}^k = \ve{b} - \ve{A}\ve{x}^k\:,\\  
    \ve{y} = (\ve{W}^T\ve{A}\ve{V})^{-1}\ve{W}^T\ve{r}_0\:,\\
    \ve{x}^{k+1} = \ve{x}^k + \ve{V}\ve{y}\:,
  \end{gather}
  \label{eq:linear_projection_iteration}
\end{subequations}
where $\ve{V}$ and $\ve{W}$ are generated from the definitions of
$\mathcal{K}$ and $\mathcal{L}$ which are defined prior to each
iteration.

From an iteration standpoint, as we choose $\boldsymbol{\delta}$ from
$\mathcal{K}$ and constrain it with $\mathcal{L}$, each iteration
performs a projection that systematically anihilates the components of
the residual that exists in $\mathcal{L}$. This then means that if our
convergence criteria for an iterative method is bound the residual of
the system, then Eq~(\ref{eq:linear_min_res}) tells us that each
projection step gaurantees us that the norm of the new residual will
never be worse than that of the previous step and will typically move
us towards convergence. Depending on the qualities of the system in
Eq~(\ref{eq:linear_problem}), the selection of the subspaces
$\mathcal{K}$ and $\mathcal{L}$ can serve to both gaurantee
convergence and optimize the rate at which the residual is decreased.

\subsubsection{Krylov Subspace Methods}
\label{subsubsec:krylov_methods}
Among the most common projection techniques used in practice are a
class of methods known as \textit{Krylov subspace methods}. Here, the
search subspace is defined as the \textit{Krylov subspace}:
\begin{equation}
  \mathcal{K}_m(\ve{A},\ve{r}_0) = span\{\ve{r}_0, \ve{A}\ve{r}_0,
  \ve{A}^2\ve{r}_0, \dots, \ve{A}^{m-1}\ve{r}_0\}\:,
  \label{eq:krylov_subspace}
\end{equation}
where $m$ denotes the dimensionality of the subspace. In order to
accomodate a more general structure for the operator in
Eq~(\ref{eq:linear_problem}), we often choose an \textit{oblique}
projection method where $\mathcal{K} \neq \mathcal{L}$. If we choose
$\mathcal{L} = \ve{A} \mathcal{K}_m(\ve{A},\ve{r}_0)$, then we are
ultimately solving the normal system $\ve{A}^T\ve{A}\ve{x} =
\ve{A}^T\ve{b}$ where $\ve{A}^T\ve{A}$ will be symmetric positive
definite if $\ve{A}$ is nonsingular, thereby expanding the range of
operators over which these methods are valid. This choice of
constraint subspace also then gives us the result via
Eq~(\ref{eq:linear_projection_constraint}) that the residual is
minimized for all $\boldsymbol{\delta} \in \mathcal{K}$, forming the
basis for the \textit{generalized minimum Residual method} (GMRES)
\citep{saad_1986}.

Choosing GMRES as our model Krylov method, we are first tasked with
finding a projector onto the subspace. We seek an orthonormal basis
for $\mathcal{K}_m(\ve{A},\ve{r}_0)$ by an orthogonalization procedure
that is commonly based on, but not limited to, the \textit{Arnoldi}
recurrence relation. The Arnoldi procedure will generate an
orthonormal basis, $\ve{V}_m \in \mathbb{C}^{N \times m}$, via a
variant of the Gram-Schmidt procedure that re-applies the operator for
each consecutive vector, thus forming a basis that spans the subspace
in Eq~(\ref{eq:krylov_subspace}). Due to its equivalent
dimensionality, $m$, to that of the subspace, we will refer to such
recurrence relations as \textit{long recurrence relations}. Those
orthogonal projection procedures that have a dimensionality less than
$m$ will be refferred to as \textit{short recurrence relations}.  Once
$\ve{V}_m$ is found, per the constraint subspace definition it then
follows that its basis is defined as $\ve{W}_m = \ve{A} \ve{V}_m$. 

Knowing the projections onto the search and constraint subspaces, the
GMRES method may be formulated. We note here several properties of
this formulation and how they may facilitate or hinder the solution of
large-scale, sparse linear problems, also noting that these properties
are common among many Krylov methods. First, from a memory perspective
GMRES is efficient from the standpoint that the operator $\ve{A}$ need
not be explicitly stored. Rather, only the ability to compute the
action of that operator on a vector of valid size is
required. However, these savings in memory are balanced by the fact
that the long recurrence relations used in the Arnoldi procedure
require all vectors that span the Krylov space to be stored. If the
size of these vectors becomes prohibitive, the Arnoldi procedure can
be restarted at the cost of losing information in the
orthogonalization process, creating the potentitial to generate new
search directions that are not orthogonal to all previous search
directions (and therefore less than optimal). From an implementation
perspective, because the operator is not required to be formed, GMRES
is significantly more flexible in its usage in that there are many
instances where various processes serve to provide the action of that
operator (e.g. radiation transport sweeps) that may not be amenable to
its construction.

\subsection{Stochastic Methods}
\label{subsec:stochastic_methods}

\subsubsection{Direct Method}
\label{subsubsec:direct_mc}

\subsubsection{Adjoint Method}
\label{subsubsec:adjoint_mc}

\subsubsection{Sequential Monte Carlo}
\label{subsusbsec:sequential_mc}

\subsection{Monte Carlo Synthetic-Acceleration}
\label{subsec:mcsa}

\section{Preconditioning the Linear System}
\label{sec:linear_preconditioning}

