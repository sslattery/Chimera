\chapter{The Linear Problem}
\label{ch:linear_problem}

The purpose of this chapter is to provide a full background on solving
linear systems as it relates to this work. It may seem like there is a
lot here, but we really do need it to fully define our new methods and
compare both qualitatively and quantitatively to convential
methods. For example, we need a solid explanation of stationary
methods because MCSA is in fact an acceleration of a stationary method
and therefore shares many properties with them. Subspace methods are
the current class of methods most widely used for sparse systems and
are at the core of the Newton-Krylov methods that we will be comparing
the FANM method with. The most rigorous piece of this chapter should
of course be the stochastic solver definitions, also providing me a
place to include some of my work in this area. Finally,
preconditioning is important for this work, MCSA requires it and
Newton-Krylov methods almost always need some type of
preconditioning. Therefore, we must also discuss these aspects.

\section{Formulation of the Linear System}
\label{sec:linear_system}

\section{Iterative Methods for Solving Sparse Linear Systems}
\label{sec:linear_methods}

\subsection{Stationary Methods}
\label{subsec:stationary_methods}

\subsection{Subspace Methods}
\label{subsec:subspace_methods}

\subsubsection{Conjugate Gradient}
\label{subsubsec:conjugate_gradient}

\subsubsection{GMRES}
\label{subsubsec:gmres}

\subsection{Stochastic Methods}
\label{subsec:stochastic_methods}

\subsubsection{Direct Method}
\label{subsubsec:direct_mc}

\subsubsection{Adjoint Method}
\label{subsubsec:adjoint_mc}

\subsubsection{Sequential Monte Carlo}
\label{subsusbsec:sequential_mc}

\subsection{Monte Carlo Synthetic-Acceleration}
\label{subsec:mcsa}

\section{Preconditioning the Linear System}
\label{sec:linear_preconditioning}

