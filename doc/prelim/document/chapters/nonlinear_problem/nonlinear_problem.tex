\chapter{The Nonlinear Problem}
\label{ch:nonlinear_problem}

This work will result in a new and novel nonlinear solver, FANM. To
movitate the development of FANM and thoroughly understand both its
mathematical properties and its qualities as related to other
nonlinear methods, we need a full workup of current iterative methods
for solving nonlinear systems. We should acknowledge operator split
and linearization methods for dealing with nonlinear systems
(e.g. think pressure correction strategies for Navier-Stokes) but the
focus here should be on fully-implicit nonlinear solves that are
amenable to solving tightly-coupled multi-physics systems in a way that
is fully consistent.

The key here will be a full understanding of Newton methods (and they
actually are quite simple compared to some of the linear solver
methods like subspace methods). It is important to understand how the
correction is computed from the Jacobian and how we may improve this
with globalization methods. Although I may not use globalization
methods in my work, they are important enough in practice that we must
be sure that a FANM method is not prohibited from using these. Current
methods, primarily Newton-Krylov methods, will of primary concern here
as these will serve as the baseline for assessing the qualities of a
FANM solver. 

Finally, as we will choose transient problems for our models, time
integration in the context of nonlinear systems will be consider and
we will survey applications to model problems in the field ( there is
a vast amount of literature on fully-implicit nonlinear solves for the
Navier-Stokes equations) that will give good benchmarks to test (such
as natural convection and driven cavity problems).

\section{Formulation of the Nonlinear System}
\label{sec:nonlinear_system}

\section{Iterative Methods for Solving Nonlinear Systems}
\label{sec:nonlinear_methods}

\subsection{Inexact Newton Methods}
\label{subsec:newton_methods}

\subsection{Newton-Krylov Methods}
\label{subsec:newton_krylov_methods}

\subsection{Globalization Methods}
\label{subsec:globalization_methods}

\section{Preconditioning the Nonlinear System}
\label{sec:nonlinear_preconditioning}

\section{Time Integration}
\label{sec:time_integration}

\section{Applications to Model Problems}
\label{sec:nonlinear_applications}
