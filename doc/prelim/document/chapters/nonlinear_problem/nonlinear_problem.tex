\chapter{The Nonlinear Problem}
\label{ch:nonlinear_problem}

This work will result in a new and novel nonlinear solver, FANM. To
movitate the development of FANM and thoroughly understand both its
mathematical properties and its qualities as related to other
nonlinear methods, we need a full workup of current iterative methods
for solving nonlinear systems. We should acknowledge operator split
and linearization methods for dealing with nonlinear systems
(e.g. think pressure correction strategies for Navier-Stokes) but the
focus here should be on fully-implicit nonlinear solves that are
amenable to solving tightly-coupled multiphysics systems in a way that
is fully consistent.

The key here will be a full understanding of Newton methods (and they
actually are quite simple compared to some of the linear solver
methods like subspace methods). It is important to understand how the
correction is computed from the Jacobian and how we may improve this
with globalization methods. Although I may not use globalization
methods in my work, they are important enough in practice that we must
be sure that a FANM method is not prohibited from using these. Current
methods, primarily Newton-Krylov methods, will of primary concern here
as these will serve as the baseline for assessing the qualities of a
FANM solver. Finally, we will survey applications to model problems in
the field ( there is a vast amount of literature on fully-implicit
nonlinear solves for the Navier-Stokes equations) that will give good
benchmarks to test (such as natural convection and driven cavity
problems).

\section{Preliminaries}
\label{sec:nonlinear_preliminaries}

\section{Inexact Newton Methods}
\label{sec:newton_methods}

\subsection{Newton-Krylov Methods}
\label{subsec:newton_krylov_methods}

\subsection{Globalization Methods}
\label{subsec:globalization_methods}

\section{The FANM Method}
\label{sec:fanm}
In this section we will devise the FANM method. Based on the
background in the previous section, we first need to motivate the
development of this work by pointing out some of the issues with
conventional methods. This includes JFNK coarseness and either
prohibitively large Krylov subspaces or a loss of orthogonality
information in the subspace due to restarts. We can than point out
some of the attractive qualities of the FANM method. Much of this will
rely on advanced concepts including automatic differentiation and
on-the-fly residual and Jacobian generation from nonlinear function
evaluations. In addition, we will want to show that the Newton method
will converge using MCSA as the linear solver. I suspect I will have
some preliminary results for a serial implementation here as I already
have a serial MCSA implemented. Finally, we will want to design the
model problems that we will test FANM with. These should be problems
that have already been worked up in the literature with benchmark-type
solutions. Parallelism in the context of the linear solvers chapter
should also be discussed.

\subsection{Automatic Differentiation}
\label{sec:automatic_differentiation}

\subsection{Residual and Jacobian Generation}
\label{sec:fanm_generation}

\subsection{Jacobian Storage vs. Subspace Storage and Restarts}
\label{sec:fanm_storage}
