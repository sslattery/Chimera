\chapter{Research Proposal}
\label{ch:research_proposal}
To develop the parallel MCSA and FANM methods, a research plan is
proposed to drive their development and quantify their merits and
feasiblity as compared to more conventional methods for linear and
nonlinear problems. We seek to rigorously verify both of these methods
and there implementation and will describe the means by which that
will be accomplished. Further more, the numerical experiments
performed will be constructed within a software framework that will
permit an agile development strategy.

\section{Experimental Framework}
\label{sec:experimental_framework}

The Chimera code base will be the primary research tool for answering
the questions posed in the previous chapters of the document. We may
not need a chapter for this, but it will likely be a good idea
somewhere in this document to outline the design of the software and
the instrumentation that it will provide.

\subsection{Algorithm-Based Design}
\label{subsec:chimera_design}

\subsection{Graph-Based Multiphysics Formulation}
\label{subsec:multiphysics_graph}

\subsection{Finite Element Assembly}
\label{subsec:fem_assembly}

\subsection{Software Engineering}
\label{subsec:software_engineering}

\section{Progress to Date}
\label{sec:progress}

\subsection{Generalization of MCSA for Linear Problems}
\label{subsec:mcsa_generalization}

\subsection{Direct vs. Adjoint Analysis}
\label{subsec:mcsa_direct_vs_adjoint}

\subsection{MCSA Application to Finite Element Problems}
\label{subsec:mcsa_finite_element}

\section{Numerical Experiments}
\label{sec:key_questions}

\section{Solution Strategies}
\label{sec:solution_strategies}

\subsection{Parallel MCSA Verification}
\label{subsubsec:mcsa_verification}

\subsection{FANM Verification}
\label{subsubsec:fanm_verification}

\section{Challenge Problem}
\label{sec:challenge_problem}
