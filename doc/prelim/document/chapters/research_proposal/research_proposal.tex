\chapter{Research Proposal}
\label{ch:research_proposal}

After all of the background information and explanations of the new
work, this section is the most important of all. Here I will clearly
define the questions I aim to answer and the strategy that I will use
to attempt to answer them. I think that it will be important to pick a
challenge problem here that FANM has a chance to show gains in either
memory footprint or parallelism over current methods. In particular, I
would like to have this challenge problem derive from nuclear reactor
multiphysics. I think Drekar would be a great test bed for FANM once
complete as it has all the tools to build a FANM solve (it is in fact
based on the same core libraries as Chimera) and a complex reactor
core assembly multiphysics system capable of scaling to large
machines. In addition, all of the complex diagnostic tools necessary
for analysis of FANM performance as compared to the other methods
should be available. 

\section{Progress to Date}
\label{sec:progress}

\subsection{Generalization of MCSA for Linear Problems}
\label{subsec:mcsa_generalization}

\subsection{Direct vs. Adjoint Analysis}
\label{subsec:mcsa_direct_vs_adjoint}

\subsection{MCSA Application to Finite Element Problems}
\label{subsec:mcsa_finite_element}

\section{Key Questions}
\label{sec:key_questions}

\section{Solution Strategies}
\label{sec:solution_strategies}

\subsection{Parallel MCSA Verification}
\label{subsubsec:mcsa_verification}

\subsection{FANM Verification}
\label{subsubsec:fanm_verification}

\section{Challenge Problem}
\label{sec:challenge_problem}
