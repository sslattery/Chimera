%        File: model_problems.tex
%
\documentclass[letterpaper,12pt]{article}
\usepackage[top=1.0in,bottom=1.0in,left=1.25in,right=1.25in]{geometry}
\usepackage{verbatim}
\usepackage{graphicx}
\usepackage{longtable}
\usepackage{amsfonts}
\usepackage{amsmath}
\usepackage{amsthm}
\usepackage{amssymb}
\usepackage{tmadd,tmath}
\usepackage[mathcal]{euscript}
\usepackage[usenames]{color}
\usepackage[
naturalnames = true, 
colorlinks = true, 
linkcolor = black,
anchorcolor = black,
citecolor = black,
menucolor = black,
urlcolor = blue
]{hyperref}

%%---------------------------------------------------------------------------%%
\author{Stuart R. Slattery
\\ \href{mailto:sslattery@wisc.edu}{\texttt{sslattery@wisc.edu}}
}

\date{\today}
\title{Chimera Model Problems and Solution Schemes}
\begin{document}
\maketitle

%%---------------------------------------------------------------------------%%
\section{Introduction}
In this document, finite volume formulations for the 2D incompressible
Navier-Stokes equations and conduction equation are generated. The
computational grid, boundary conditions, time integrations, and
solution methods are also formulated. Using these formulations, the
conjugate heat transfer problem is posed. These model problems and
solution schemes will be used within the Chimera code framework.

%%---------------------------------------------------------------------------%%
\section{Computational Grid}

%%---------------------------------------------------------------------------%%
\section{Conduction Equation}
The conduction equation describes the movement of thermal energy
through a solid material. We begin the derivation of the conduction
equation by defining the conserved quantity, energy, in terms of
temperature:

\begin{equation}
  E(x,y,t) = C_p(x,y) T(x,y,t)\:,
  \label{eq:conduction_energy}
\end{equation}

where $C_P$ is the specific heat capacity of the material. We also
define the energy flux, $F$, in terms of temperature using Fick's law:

\begin{equation}
  F(T(x,y,t)) = -\beta(x,y) \nabla T(x,y,t)\:,
  \label{eq:conduction_ficks_law}
\end{equation}

where $\beta$ is the thermal conductivity of the material. With these
two relations in hand, we can then write a conservation statement for
energy over a grid cell \cite{leveque_2002}:

\begin{equation}
  \frac{d}{dt} \iint_{C_{ij}} E(x,y,t) dx dy = -F(T(x,y,t))
  |_{B_{ij}}\:.
  \label{eq:conduction_conservation}
\end{equation}

Here we balance the time rate of change in energy in the unit cell
with the flux on the cell boundaries. Inserting
Eq.~(\ref{eq:conduction_energy}) into
Eq.~(\ref{eq:conduction_conservation}) and dividing by the cell volume
recovers the cell-averaged temperature and specific heat on the left
hand side:

\begin{equation}
  \frac{d}{dt} C_{P_{ij}} T_{ij} = -\frac{1}{V_{ij}}F(T(x,y,t))
  |_{B_{ij}}\:.
  \label{eq:conduction_cell_averaged}
\end{equation}

\subsection{Spatial Discretization}

We begin the spatial discretization by inserting
Eq.~(\ref{eq:conduction_ficks_law}) and differencing the left hand
side, and integrate, maintaining the conservation properties of the
system:

\begin{equation}
  C_{P_{ij}} ( T^{n+1}_{ij} - T^n_{ij} ) = \frac{1}{V_{ij}}
  \int_{t^n}^{t^{n+1}} \Big[(F_{i+1/2,j}-F_{i-1/2,j})\Delta_j +
    (F_{i,j+1/2}-F_{i,j-1/2})\Delta_i \Big] dt\:.
  \label{eq:conduction_time_integral}
\end{equation}

We choose to approximate the integral on the right hand side and use a
first order central difference for the derivative in the heat flux
terms with a backwards Euler representation for the time integration:

\begin{equation}
  C_{P_{ij}}( T^{n+1}_{ij} - T^n_{ij} ) = \frac{\Delta_t}{V_{ij}}
  \Big[ (F^{n+1}_{i+1/2,j}-F^{n+1}_{i-1/2,j})\Delta_j +
    (F^{n+1}_{i,j+1/2}-F^{n+1}_{i,j-1/2})\Delta_i \Big]
  \label{eq:conduction_fv_equation}
\end{equation}

The energy fluxes on the boundaries will be approximated with a
second-order accurate central differencing scheme with the thermal
conductivities lagged:

\begin{equation}
  \begin{aligned}
  F^{n+1}_{i-1/2,j} = -\beta^n_{i-1/2,j} \frac{T^{n+1}_{ij}
    -T^{n+1}_{i-1,j}}{\Delta_{i-1/2}} \\ F^{n+1}_{i+1/2,j} =
  -\beta^n_{i+1/2,j} \frac{T^{n+1}_{i+1,j}
    -T^{n+1}_{ij}}{\Delta_{i+1/2}} \\ F^{n+1}_{i,j-1/2} =
  -\beta^n_{i,j-1/2} \frac{T^{n+1}_{ij}
    -T^{n+1}_{i,j-1}}{\Delta_{j-1/2}} \\ F^{n+1}_{i,j+1/2} =
  -\beta^n_{i,j+1/2} \frac{T^{n+1}_{i,j+1}
    -T^{n+1}_{i,j}}{\Delta_{j+1/2}}\:.
    \label{eq:conduction_fluxes}
  \end{aligned}
\end{equation}

Here, we define:

\begin{equation}
  \begin{aligned}
    \Delta_{l-1/2} = \frac{\Delta_l + \Delta_{l-1}}{2}
    \\ \Delta_{l+1/2} = \frac{\Delta_{l+1} + \Delta_l}{2}\:,
    \label{eq:conduction_half_delta}
  \end{aligned}
\end{equation}

where $l \in [i,j]$. Rearranging for temperature, we can then form a
system of equations amenable to matrix solutions, not yet considering
boundary conditions:

\begin{equation}
 a_{ij}T^{n+1}_{ij} + a_{i-1,j}T^{n+1}_{i-1,j} +
 a_{i+1,j}T^{n+1}_{i+1,j} + a_{i,j-1}T^{n+1}_{i,j-1} +
 a_{i,j+1}T^{n+1}_{i,j+1} = \frac{C_{P_{ij}}V_{ij}}{\Delta t}
 T^n_{ij}\:,
 \label{eq:conduction_inner_terms}
\end{equation}

where:

\begin{equation}
  \begin{aligned}
    a_{i+1,j} = -\frac{\Delta_j \beta^n_{i+1/2}}{\Delta_{i+1/2}}
    \\ a_{i-1,j} = -\frac{\Delta_j \beta^n_{i-1/2}}{\Delta_{i-1/2}}
    \\ a_{ij} = \frac{\Delta_j \beta^n_{i+1/2}}{\Delta_{i+1/2}} +
    \frac{\Delta_j \beta^n_{i-1/2}}{\Delta_{i-1/2}} + \frac{\Delta_i
      \beta^n_{j-1/2}}{\Delta_{j-1/2}} + \frac{\Delta_i
      \beta^n_{j+1/2}}{\Delta_{j+1/2}} +
    \frac{C_{P_{ij}}V_{ij}}{\Delta t} \\ a_{i,j-1} = -\frac{\Delta_i
      \beta^n_{j-1/2}}{\Delta_{j-1/2}} \\ a_{i,j+1} = -\frac{\Delta_i
      \beta^n_{j+1/2}}{\Delta_{j+1/2}} \:.
    \label{eq:conduction_coeffs}
  \end{aligned}
\end{equation}

We are then left to resolve the thermal conductivity values defined on
the cell boundaries with values defined at the cell centers. We choose
these values such that the energy flux across a cell boundary is
continuous from cell to cell, even if two adjacent cells have
different materials. We do this be equating the boundary flux computed
from the perspective of each neighboring cell with the gradient
generated using a one sided difference:

\begin{equation}
  \begin{aligned}
    F^{l+1}_{l+1/2} = -2 \beta_{l+1}
    \frac{T_{l+1}-T_{l+1/2}}{\Delta_l} \\ F^{l}_{l+1/2} = -2 \beta_{l}
    \frac{T_{l+1/2}-T_l}{\Delta_l} \\
    F^{l+1}_{l+1/2} = F^{l}_{l+1/2} \:,
    \label{eq:conduction_tc_step1}
  \end{aligned}
\end{equation}

with $F^{l}_{l+1/2}$ defined by the cell-centered values in cell $l$
and $l \in [i,j]$. We can follow the same procedure for the other side
of the cell:

\begin{equation}
  \begin{aligned}
    F^{l}_{l-1/2} = -2 \beta_{l} \frac{T_{l}-T_{l-1/2}}{\Delta_l}
    \\ F^{l-1}_{l-1/2} = -2 \beta_{l-1}
    \frac{T_{l-1/2}-T_{l-1}}{\Delta_l} \\ F^{l}_{l-1/2} =
    F^{l-1}_{l-1/2} \:,
    \label{eq:conduction_tc_step2}
  \end{aligned}
\end{equation}

We solve Eqs~(\ref{eq:conduction_tc_step1}) and
(\ref{eq:conduction_tc_step2}) for $T_{l+1/2}$ and $T_{l-1/2}$ to
eliminate the cell centered temperatures:

\begin{equation}
  \begin{aligned}
    T_{l+1/2} = \frac{\Delta_l \beta_{l+1} T_{l} + \Delta_{l+1}
      \beta_{l} T_{l+1}} {\Delta_l \beta_{l+1} + \Delta_{l+1} \beta_{l}}
    \\ T_{l-1/2} = \frac{\Delta_{l-1} \beta_{l} T_{l} + \Delta_{l}
      \beta_{l-1} T_{l-1}} {\Delta_{l} \beta_{l-1} + \Delta_{l}
      \beta_{l-1}} \:.
    \label{eq:conduction_tc_step3}
  \end{aligned}
\end{equation}

We then substitute into either of the the equations in both
Eqs~(\ref{eq:conduction_tc_step1}) and (\ref{eq:conduction_tc_step2})
and equate the result with Eq~(\ref{eq:conduction_fluxes}) to arrive
at the final cell-centered thermal conductivity values:

\begin{equation}
  \begin{aligned}
    \beta^n_{i+1/2,j} = \\
    \beta^n_{i-1/2,j} = \\
    \beta^n_{i,j+1/2} = \\
    \beta^n_{i,j-1/2} = \:.
    \label{eq:conduction_tc_step_4}
  \end{aligned}
\end{equation}

\subsection{Boundary Conditions}

We will consider both Neumann and Dirichlet type boundary conditions
for the conduction equation. For the Neumann condition we have a
constant temperature at the boundary:

\begin{equation}
  T_{ij} = T_0, \forall (i,j) \in \Gamma
  \label{eq:conduction_neumann}
\end{equation}

For the Dirichlet condition, we have a constant heat flux at the
boundary.

\subsection{Linear System Form}

%%---------------------------------------------------------------------------%%
\section{Incompressible Navier-Stokes Equations}

\subsection{Boundary Conditions}

\subsection{Nonlinear System Form}

%%---------------------------------------------------------------------------%%
\pagebreak
\bibliographystyle{ieeetr} \bibliography{references}
\end{document}


