%        File: model_problems.tex
%
\documentclass[letterpaper,12pt]{article}
\usepackage[top=1.0in,bottom=1.0in,left=1.25in,right=1.25in]{geometry}
\usepackage{verbatim}
\usepackage{graphicx}
\usepackage{longtable}
\usepackage{amsfonts}
\usepackage{amsmath}
\usepackage{amsthm}
\usepackage{amssymb}
\usepackage{tmadd,tmath}
\usepackage[mathcal]{euscript}
\usepackage[usenames]{color}
\usepackage[
naturalnames = true, 
colorlinks = true, 
linkcolor = black,
anchorcolor = black,
citecolor = black,
menucolor = black,
urlcolor = blue
]{hyperref}

%%---------------------------------------------------------------------------%%
\author{Stuart R. Slattery
\\ \href{mailto:sslattery@wisc.edu}{\texttt{sslattery@wisc.edu}}
}

\date{\today}
\title{A Finite Volume Formulation for the 2D Incompressible
  Navier-Stokes and Conduction Equations}
\begin{document}
\maketitle

%%---------------------------------------------------------------------------%%
\section{Computational Grid}

%%---------------------------------------------------------------------------%%
\section{Conduction Equation}
We begin the derivation of the conduction equation by defining the
conserved quantity, energy, in terms of temperature:

\begin{equation}
  E(x,y,t) = C_p(x,y) T(x,y,t)\:,
  \label{eq:conduction_energy}
\end{equation}

where $C_P$ is the specific heat capacity of the material. We also
define the heat flux, $F$, in terms of temperature using Fick's law:

\begin{equation}
  F(T(x,y,t)) = -\beta(x,y) \nabla T(x,y,t)\:,
  \label{eq:conduction_ficks_law}
\end{equation}

where $\beta$ is the thermal conductivity of the material. With these
two relations in hand, we can then write a conservation law for energy
\cite{leveque_2002}:

\begin{equation}
  \frac{d}{dt} \int \int_{C_{ij}} E(x,y,t) dx dy = -F(T(x,y,t)) |_{B_{ij}}\:.
  \label{eq:conduction_conservation}
\end{equation}

Here we balance the time rate of change in energy in the unit cell
with the flux on the cell boundaries. Inserting
Eq.~(\ref{eq:conduction_energy}) int
Eq.~(\ref{eq:conduction_conservation}) and dividing by the cell volume
recovers the cell averaged temperature and specific heat on the left
hand side:

\begin{equation}
  \frac{d}{dt} C_{P_{ij}} T_{ij} = -\frac{1}{\Delta_i
    \Delta_j}F(T(x,y,t)) |_{B_{ij}}\:.
  \label{eq:conduction_cell_averaged}
\end{equation}

We then insert Eq.~(\ref{eq:conduction_ficks_law}), difference the
left hand side, and integrate, maintaining the conservation properties
of the system:

\begin{multline}
  C_{P_{ij}} ( T^{n+1}_{ij} - T^n_{ij} ) = \frac{1}{\Delta_i \Delta_j}
  \int_{t^n}^{t^{n+1}} [ -\beta_{i-1/2,j} \nabla T_{i-1/2,j}\\
    + \beta_{i+1/2,j} \nabla T_{i+1/2,j} - \beta_{i,j-1/2} \nabla
    T_{i,j-1/2} +\beta_{i,j+1/2} \nabla T_{i,j+1/2} ] dt\:.
  \label{eq:conduction_time_integral}
\end{multline}

We choose to approximate the integral on the right hand side and use a
first order central difference for the derivative in the heat flux
terms giving a backwards Euler representation for the time
integration:

\begin{multline}
  C_{P_{ij}}( T^{n+1}_{ij} - T^n_{ij} ) = \frac{\Delta_t}{\Delta_i
    \Delta_j} -\beta_{i-1/2,j} \frac{T^{n+1}_{ij}
    -T^{n+1}_{i-1,j}}{\Delta_i} +\beta_{i+1/2,j} \frac{T^{n+1}_{i+1,j}
    -T^{n+1}_{ij}}{\Delta_i}\\
  -\beta_{i,j-1/2} \frac{T^{n+1}_{ij}
    -T^{n+1}_{i,j-1}}{\Delta_j} +\beta_{i,j+1/2} \frac{T^{n+1}_{i,j+1}
    -T^{n+1}_{i,j-1}}{\Delta_j}\:.
  \label{eq:conduction_fv_multline}
\end{multline}

We will define $\alpha$ as:

\begin{equation}
  \alpha_{ij} = \frac{\Delta_t}{C_{P_{ij}} \Delta_i \Delta_j}
  \label{eq:conduction_coeff}\:.
\end{equation}

Rearranging for temperature, we can then form a system of equations
amenable to matrix solutions, not yet considering boundary conditions,
arriving at the discrete, time-dependent conduction equation:

\begin{multline}
 \Big[1 + \frac{\alpha_{ij}}{\Delta_i}(\beta_{i-1/2,j} + \beta_{i+1/2,j})\Big]
 T^{n+1}_{ij} + \Big[-\frac{\alpha_{ij}}{\Delta_i} \beta_{i-1/2,j}\Big]
 T^{n+1}_{i-1,j} + \Big[\frac{-\alpha_{ij}}{\Delta_i} \beta_{i+1/2,j}\Big]
 T^{n+1}_{i+1,j}\\
 + \Big[-\frac{\alpha_{ij}}{\Delta_j} \beta_{i-1/2,j}\Big]
 T^{n+1}_{i,j-1} + \Big[-\frac{\alpha_{ij}}{\Delta_j} \beta_{i+1/2,j}\Big]
 T^{n+1}_{i,j+1} = T^n_{ij}\:.
 \label{eq:conduction_inner_terms}
\end{multline}

\subsection{Boundary Conditions}

We will consider both Neumann and Dirichlet type boundary conditions
for the conduction equation. For the Neumann condition we have a
constant temperature at the boundary:

\begin{equation}
  T_{ij} = T_0, \forall (i,j) \in \Gamma
  \label{eq:conduction_neumann}
\end{equation}

For the Dirichlet condition, we have a constant heat flux at the boundary.

\subsection{Linear System Form}

%%---------------------------------------------------------------------------%%
\section{Navier-Stokes Equations}

\subsection{Boundary Conditions}

\subsection{Nonlinear System Form}

%%---------------------------------------------------------------------------%%
\pagebreak
\bibliographystyle{ieeetr} \bibliography{references}
\end{document}


