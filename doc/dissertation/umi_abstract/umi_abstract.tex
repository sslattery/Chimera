This work researches and develops Monte Carlo Synthetic Acceleration
(MCSA) methods as a new class of solution techniques for discrete
neutron transport and fluid flow problems. Monte Carlo Synthetic
Acceleration methods use a traditional Monte Carlo process to
approximate the solution to the discrete problem as a means of
accelerating traditional fixed-point methods. To apply these methods
to neutronics and fluid flow and determine the feasibility of these
methods on modern hardware, three complementary research and
development exercises are performed.

First, solutions to the $SP_N$ discretization of the linear Boltzmann
neutron transport equation are obtained using MCSA with a difficult
criticality calculation for a light water reactor fuel assembly used
as the driving problem. To enable MCSA as a solution technique a group
of modern preconditioning strategies are researched. MCSA when
compared to conventional Krylov methods demonstrated improved
iterative performance over GMRES by converging in fewer iterations
when using the same preconditioning.

Second, solutions to the compressible Navier-Stokes equations were
obtained by developing the Forward-Automated Newton-MCSA (FANM) method
for nonlinear systems based on Newton's method. Three difficult fluid
benchmark problems in both convective and driven flow regimes were
used to drive the research and development of the method. For 8 out of
12 benchmark cases, it was found that FANM had better iterative
performance than the Newton-Krylov method by converging the nonlinear
residual in fewer linear solver iterations with the same
preconditioning.

Third, a new domain decomposed algorithm to parallelize MCSA aimed at
leveraging leadership-class computing facilities was developed by
utilizing parallel strategies from the radiation transport
community. The new algorithm utilizes the Multiple-Set
Overlapping-Domain strategy in an attempt to reduce parallel overhead
and add a natural element of replication to the algorithm. It was
found that for the current implementation of MCSA, both weak and
strong scaling improved on that observed for production
implementations of Krylov methods.
