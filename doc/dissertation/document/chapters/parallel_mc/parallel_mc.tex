\chapter{Parallel Monte Carlo Solution Methods for Linear Systems}
\label{ch:parallel_methods}

For Monte Carlo methods, and in particular MCSA, to be viable at the
production scale, scalable parallel implementations are required. In a
general linear algebra context for discrete systems, this has not yet
been achieved. Therefore, we will develop and implement parallel
algorithms for these methods leveraging both the knowledge gained from
the general parallel implementations of Krylov methods and modern
parallel strategies for Monte Carlo as developed by the reactor
physics community. In order to formulate a parallel MCSA algorithm, we
recognize that the algorithm occurs in two stages, an outer iteration
performing Richardson's iteration and applying the correction, and an
inner Monte Carlo solver that is generating the correction via the
adjoint method. The parallel aspects of both these components must be
considered. In this chapter we briefly review particle transport
methods for domain decomposed Monte Carlo and then devise an algorithm
based on these methods for efficient parallel solutions using the
adjoint Neumann-Ulam method and MCSA.

%%---------------------------------------------------------------------------%%
\section{Domain Decomposition for Monte Carlo}
\label{sec:msod}
As observed in the discussion on parallel Krylov methods, large-scale
problems will surely have their data partitioned such that each
parallel process owns a subset of the equations in the linear
system. Given this convention, the adjoint Monte Carlo algorithm must
perform random walks over a domain that is decomposed and must remain
decomposed due to memory limitations. This naturally leads us to seek
parallel Monte Carlo algorithms that handle domain decomposition.

In the context of radiation transport, Brunner and colleagues provided
a survey of algorithms for achieving this as implemented in production
implicit Monte Carlo codes \citep{brunner_comparison_2006}. In their
work they identify two data sets that are required to be communicated:
the sharing of particles that are transported from one domain to
another and therefore from one processor to another and a global
communication that signals if particle transport has been completed on
all processors. The algorithms presented are a fully-locking
synchronous scheme, an asynchronous-send/synchronous-receive pattern,
a traditional master/slave scheme, and a modified master/slave scheme
the implements a binary tree pattern for the global reduction type
operations needed to communicate between the master and slave
processes. They observed that the modified master/slave scheme
performed best in that global communications were implemented more
efficiently than those required by the asynchronous
scheme. Furthermore, none of these schemes handled load-imbalanced
cases efficiently. Such cases will be common if the source sampled in
the Monte Carlo random walk is not isotropic and not evenly
distributed throughout the global domain. It was noted that
efficiencies were improved by increasing the frequency by which
particle data was communicated between domain-adjacent
processors. However, this ultimately increases communication costs. In
2009, Brunner extended his work by using a more load-balanced approach
with a fully asynchronous communication pattern
\citep{brunner_efficient_2009}. Although the extended implementation
was more robust and allowed for scaling to larger numbers of
processors, performance issues were still noted with parallel
efficiency improvements needed in both the weak and strong scaling
cases for unbalanced problems. These results led Brunner to conclude
that a combination of domain decomposition and domain replication
could be used to solve some of these issues.

\subsection{Multiple-Set Overlapping-Domain Decomposition}
\label{subsec:msod}
In 2010, Wagner and colleagues developed the \textit{multiple-set
  overlapping-domain} (MSOD) decomposition for parallel Monte Carlo
applications for full-core light water reactor analysis
\citep{wagner_hybrid_2010}. In their work, an extension of Brunner's,
their scheme employed similar parallel algorithms for particle
transport but a certain amount of overlap between adjacent domains was
used to decrease the number of particles leaving the local domain. In
addition, Wagner utilized a level of replication of the domain such
that the domain was only decomposed on $O(100)$ processors and if
replicated $O(1,000)$ times achieves simulation on $O(100,000)$
processors, thus providing spatial and particle parallelism. Each
collection of processors that constitutes a representation of the
entire domain is referred to as a set, and within a set overlap occurs
among its sub-domains. The original motivation was to decompose the
domain in a way that it remained in a physical cabinet in a large
distributed machine, thus reducing latency costs during
communication. A multiple set scheme is also motivated by the fact
that communication during particle transport only occurs within a set,
limiting communications during the transport procedure to a group of
$O(100)$ processors, a number that was shown to have excellent
parallel efficiencies in Brunner's work and therefore will scale well
in this algorithm. The overlapping domains within each set also
demonstrated reduced communication costs. On each processor, the
source is sampled in the local domain that would exist if no overlap
was used while tallies can be made over the entire overlapping domain.

To demonstrate this, consider the example adapted from Mervin's work
with Wagner and others in the same area \citep{mervin_variance_2012}
and presented in Figure~\ref{fig:msod_example}.
\begin{figure}[t!]
  \begin{center}
    \scalebox{1.5}{
      \input{chapters/parallel_mc/msod_example.pdftex_t} }
  \end{center}
  \caption{\textbf{Overlapping domain example illustrating how domain
      overlap can reduce communication costs.}
    \textit{All particles start in the blue region of interest. The
      dashed line represents 0.5 domain overlap between domains.}}
  \label{fig:msod_example}
\end{figure}
In this example, 3 particle histories are presented emanating from the
blue region of interest. Starting with particle A, if no domain
overlap is used then the only the blue domain exists on the starting
processor. Particle A is then transported through 3 other domains
before the history ends, therefore requiring three communications to
occur in Brunner's algorithm. If a 0.5 domain overlap is permitted as
shown by the dashed line, then the starting process owns enough of the
domain such that no communications must occur in order to complete the
particle A transport process. Using 0.5 domain overlap also easily
eliminates cases such as the represented by the path of particle C. In
this case, particle C is scattering between two adjacent domains,
incurring a large latency cost for a single particle. Finally, with
particle B we observe that 0.5 domain overlap will still not eliminate
all communications. However, if 1 domain overlap were used, the entire
geometry shown in Figure~\ref{fig:msod_example} would be contained on
the source processor and therefore transport of all 3 particles
without communication would occur.

Wagner and colleagues used this methodology for a 2-dimensional
calculation of a pressurized water reactor core and varied the domain
overlap from 0 to 3 domain overlap (a $7 \times 7$ box in the context
of our example) where a domain constituted a fuel assembly. For the
fully domain decomposed case, they observed that 76.12\% of all source
particles leave the domain. At 1.5 domain overlap, the percentage of
source particles born in the center assembly leaving the processor
domain dropped to 1.05\% and even further for 0.02\% for the 3 domain
overlap. Based on these results, this overlap approach, coupled with
the multiple sets paradigm that will scale for existing parallel
transport algorithms, provides a scalable Monte Carlo algorithm for
today's modern machines.

%%---------------------------------------------------------------------------%%
\section{Fully Asynchronous MSOD Neumann-Ulam Algorithm}
\label{sec:asynchronous_algorithm}
We can take much of what was learned from Brunner and Wagner's
parallel Monte Carlo methods for radiation transport and directly
apply it to a parallel formulation of our stochastic linear
solvers. Direct analogs can be derived from these works by noting that
the primary difference between solving a linear system with Monte
Carlo methods and fixed source Monte Carlo transport problems is the
content of the Markov chains that are generated. The transitions
represented by these chains are bound by probabilities and weights and
are initiated by the sampling of a source. In the context of transport
problems, those transitions represent events such as particle
scattering and absorption with probabilities that are determined by
physical data in the form of cross sections. For stochastic matrix
inversion, those transitions represent moving between the equations of
the linear system (and therefore the physical domain which they
represent) and their probabilities are defined by the coefficients of
those equations. Ultimately, we tally the contributions to generate
expectation values in the desired states as we progress through the
chains. Therefore, parallel methods for Monte Carlo radiation
transport can be abstracted and we can use those concepts that apply
to matrix inversion methods as an initial means of developing a
parallel Neumann-Ulam-type solver. In this section the fully
asynchronous MSOD Neumann-Ulam algorithm developed for this work is
presented step-by-step in detail.

\subsection{Parallel Random Number Generation}
\label{subsec:parallel_rng}
Correct parallel Monte Carlo requires parallel random number
generation. To ensure distinct Markov chains for each history in the
problem, regardless of its state in phase space or parallel space, a
unique stream of random numbers is necessary to achieve uncorrelated
tally results for the solution. 

\subsection{Generating the Transport Domain with Overlap and Preconditioning}
\label{subsec:domain_generation}

\subsection{Generating the Transport Source with Preconditioning}
\label{subsec:source_generation}

\subsection{Transition Processing Kernel}
\label{subsec:transition_processing}

\subsection{Local Domain Transport Kernel}
\label{subsec:local_domain_transport}

\subsection{Domain-to-Domain Communication Kernel}
\label{subsec:domain_to_domain_kernel}

\subsection{Asynchronous Transport Kernel}
\label{subsec:async_transport_kernel}

\subsection{Multiple Set Algorithm}
\label{subsec:multiple_sets_algorithm}

%%---------------------------------------------------------------------------%%
\section{Parallel MCSA}
\label{sec:parallel_mcsa}
With the parallel adjoint Neumann-Ulam solver implementation described
aboive, the parallel implementation of the MCSA method is
trivial. Recall the MCSA iteration procedure outlined in
Eq~(\ref{eq:mcsa}). In \S\ref{sec:parallel_krylov_methods} we
discussed parallel matrix and vector operations as utilized in
conventional Krylov methods. We utilize these here for the parallel
MCSA implementation. In the first step, a parallel matrix-vector
multiply is used to apply the split operator to the previous iterate's
solution. A parallel vector update is then performed with the source
vector to arrive at the initial iteration guess. In the next step, the
residual is computed by the same operations where now the operator is
applied to the solution guess with a parallel matrix-vector multiply
and then a parallel vector update with the source vector is
performed. Once the correction is computed with a parallel adjoint
Neumann-Ulam solve, this correction is applied to the guess with a
parallel vector update to get the new iteration
solution. Additionally, as given by
Eq~(\ref{eq:mcsa_stopping_criteria}), 2 parallel vector reductions
will be required to check the stopping criteria: one initially to
compute the infinity norm of the source vector, and another at every
iteration to compute the infinity norm of the residual vector. For
this implementation, all of the issues that will be potentially
generated by the parallel adjoint solver implementation will manifest
themselves here as the quality of the correction will be of intense
study.


%%---------------------------------------------------------------------------%%
\section{Load Balancing Concerns}
\label{sec:mc_load_balancing}
Although domain decomposition was shown to be efficient in a perfectly
load balanced situation in Siegel's work \citep{siegel_analysis_2012},
careful consideration must be made for situations where this is not
the case. Given the stochastic nature of the problem and lack of a
globally homogeneous domain, parallel Monte Carlo simulations are
inherently load imbalanced. Procassini and others worked to alleviate
some of the load imbalances that are generated by both particle and
spatial parallelism and are therefore applicable to the MSOD algorithm
\citep{procassini_dynamic_2005}. They chose a dynamic balancing scheme
in which the number of times a particular domain was replicated was
dependent on the amount of work in that domain (i.e. domains with a
high particle flux and therefore more particle histories to compute
require more work). In this variation, domains that require more work
will be replicated more frequently at reduced particle counts in each
replication. Furthermore, Procassini and colleagues noted that as the
simulation progressed and particles were transported throughout the
domain, the amount of replication for each domain would vary as
particle histories began to diffuse, causing some regions to have
higher work loads and some to have smaller work loads than the initial
conditions.

Consider the example in Figure~\ref{fig:procassini_example} adapted
from Procassini's work.
\begin{figure}[t!]
  \begin{center}
    \scalebox{1.5}{
      \input{chapters/parallel_mc/procassini_example.pdftex_t} }
  \end{center}
  \caption{\textbf{Example illustrating how domain decomposition can
      create load balance issues in Monte Carlo.}  \textit{A domain is
      decomposed into 4 zones on 8 processors with a point source in
      the lower left zone. As the particles diffuse from the source in
      the random walk sequence as shown in the top row, their tracks
      populate the entire domain. As given in the bottom row, as the
      global percentage of particles increases in a zone, that zone's
      replication count is increased.}}
  \label{fig:procassini_example}
\end{figure}
In this example, a geometry is decomposed into 4 domains on 8
processors with a point source in the bottom left domain. To begin,
because the point source is concentrated in one domain, that domain is
replicated 5 times such the amount of work it has to do per processor
is roughly balanced with the others. As the particles begin to diffuse
away from the point source, the amount of replication is adjusted to
maintain load balance. Near the end of the simulation, the diffusion
of particles is enough that all domains have equal replication.  By
doing this, load balance is improved as each domain has approximately
equal work although each domain may represent a different spatial
location and have a differing number of histories to
compute. Compared to Wagner's work where the fission source was
distributed relatively evenly throughout the domain, fixed source
problems (and especially those that have a point-like source) like
those presented in Procassini's work will be more prone to changing
load balance requirements.

%%---------------------------------------------------------------------------%%
\section{Reproducible Domain Decomposed Results}
\label{sec:reproducible_mc}
The 2006 work of Brunner is notable in that the Monte Carlo codes used
to implement and test the algorithms adhered to a strict policy of
generating identical results independent of domain decomposition or
domain replication as derived from the work of Gentile and colleagues
\citep{gentile_obtaining_2005}. In Gentile's work, a procedure is
given for obtaining results reproducible to machine precision for an
arbitrary number of processors and domains. Differences can arise from
using a different random number sequence in each domain and performing
a sequence of floating point operations on identical data in a
different order, leading to variations in round-off error and
ultimately a non-identical answer. They use a simple example,
recreated below in Figure~\ref{fig:gentile_example}, that illustrates
these issues.
\begin{figure}[t!]
  \begin{center}
    \scalebox{1.5}{
      \input{chapters/parallel_mc/gentile_example.pdftex_t} }
  \end{center}
  \caption{\textbf{Gentile's example illustrating how domain
      decomposition can create reproducibility issues in Monte Carlo.}
    \textit{Both particles A and B start in zone 1 on processor
      1. Particle A moves to zone 2 on processor 2 and scatters back
      to zone 1 while B scatters in zone 1 and remains there. A1 and
      A2 denote the track of particle A that is in zone 1 while B1 and
      B2 denote the track of particle B that is in zone 1.}}
  \label{fig:gentile_example}
\end{figure}
In this example, the domain is decomposed on two processors with each
processor owning one of the two zones. Starting with particle A, it is
born in zone 1 and is transported to zone 2 where a scattering event
occurs. Concerning the first reproducibility issue, if the same
sequence of random numbers is not used to compute the trajectory from
the new scattering event, we cannot expect to achieve the same result
if the domain were not decomposed. If the numbers are different, the
scattering event in zone 2 may keep the particle there or even eject
it from the domain if the sequence were different. The second issue is
demonstrated by adding another particle B that remains in the
domain. In this case, an efficient algorithm will transport particle A
on processor 1 until it leaves zone 1 and then transport particle
B. Particle A will not renter the domain until it has been
communicated to processor 2, processor 2 performs the transport, and
it is communicated back to processor 1. If we are doing a track-length
tally in zone 1, then we sum the tracks lengths observed in that
zone. In the single processor, single zone case particle A would be
transported in its entirety and then particle B transported. This
would result in a tally sum with the following order of operations:
$(((A1+A2)+B1)+B2)$. If we were instead to use 2 processors, we would
instead have the following order: $(((A1+B1)+B2)+A2)$. In the context
of floating point operations, we cannot expect these to have an
identical result to machine precision as round-off approximations will
differ resulting in non-commutative addition.

Procassini's solutions to these problems are elegant in that they
require a minimal amount of modification to be applied to the Monte
Carlo algorithm. To solve the first issue, in order to ensure each
particle maintains an invariant random number sequence that determines
its behavior regardless of domain decomposition, each particle is
assigned a random number seed that describes its current state upon
entering the domain of a new processor. These seeds are derived from
the actual geometric location of the particle such that it is
decomposition invariant. Non-commutative floating point operations are
overcome by instead mapping floating point values to 64-bit integer
values for which additions will always be commutative. Once the
operations are complete, these integers are mapped back to floating
point values.

