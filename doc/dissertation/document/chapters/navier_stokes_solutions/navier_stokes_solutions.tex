\chapter{Monte Carlo Solution Methods for the Navier-Stokes Equations}
\label{ch:mc_ns_solutions}

\subsection{FANM Verification}
\label{subsec:fanm_verification}
To verify the FANM method for nonlinear problems, we choose benchmark
solutions for the 2-dimensional, steady, incompressible Navier-Stokes
equations on a rectilinear grid in much the same way as Shadid and
Pawlowski's work on Newton-Krylov methods for the solution of these
equations \citep{shadid_inexact_1997,pawlowski_globalization_2006}. We
define these equations as follows:
\begin{subequations}
  \begin{gather}
    \rho \ve{u} \cdot \nabla \ve{u} - \nabla \cdot \ve{T} - \rho
    \ve{g} = \ve{0}
    \label{eq:ns_momentum}\\
    \nabla \cdot \ve{u} = 0
    \label{eq:ns_continuity}\\
    \rho C_p \ve{u} \cdot \nabla T + \nabla \cdot \ve{q} = 0\:,
    \label{eq:ns_energy}
  \end{gather}
  \label{eq:navier_stokes}
\end{subequations}
where $\rho$ is the fluid density, $\ve{u}$ is the fluid velocity,
$\ve{g}$ gravity, $C_p$ the specific heat capacity at constant
pressure of the fluid and $T$ the temperature of the
fluid. Eq~(\ref{eq:ns_momentum}) provides momentum transport,
Eq~(\ref{eq:ns_continuity}) provides the mass balance, and
Eq~(\ref{eq:ns_energy}) provides energy transport with viscous
dissipation effects neglected. In addition, we close the system with
the following equations:
\begin{subequations}
  \begin{gather}
    \ve{T} = -P \ve{I} + \mu[\nabla \ve{u} + \nabla \ve{u}^T]
    \label{eq:ns_stress_tensor}\\
    \ve{q} = - k \nabla T\:,
    \label{eq:ns_heat_flux}
  \end{gather}
  \label{eq:ns_closure}
\end{subequations}
where $\ve{T}$ is the stress tensor, $P$ is the hydrodynamic pressure,
$\mu$ is the dynamic viscosity of the fluid, $\ve{q}$ is the heat flux
in the fluid, and $k$ is the thermal conductivity of the fluid. This
set of strongly coupled equations possesses both the nonlinearities
and asymmetries that we are seeking for qualification of the FANM
method. Further, physical parameters within these equations can be
tuned to enhance the nonlinearities. We will then apply these
equations to the following three standard benchmark problems.

\subsubsection{Thermal Convection Cavity Problem}
\label{subsubsec:natural_convection_cavity}
In 1983 a benchmark solution for the natural convection of air in a
square cavity was published \citep{de_vahl_davis_natural_1983} as
shown in Figure~\ref{fig:natural_convection_cavity} for the solution
of the energy, mass, and momentum equations.
\begin{figure}[t!]
  \begin{center}
    \scalebox{1.5}{
      \input{chapters/navier_stokes_solutions/natural_convection_cavity.pdftex_t} }
  \end{center}
  \caption{\textbf{Problem setup for the natural convection cavity
      benchmark.} \textit{Dirichlet conditions are set for the
      temperature on the left and right while Neumann conditions are
      set on the top and bottom of the Cartesian grid. The temperature
      gradients will cause buoyancy-driven flow. Zero velocity
      Dirichlet conditions are set on each boundary. No thermal source
      was present.}}
  \label{fig:natural_convection_cavity}
\end{figure}
In this problem, a rectilinear grid is applied to the unit square. No
heat flow is allowed out of the top and bottom of the square with a
zero Neumann condition specified. Buoyancy driven flow is generated by
the temperature gradient from the cold and hot Dirichlet conditions on
the left and right boundaries of the box. By adjusting the Rayleigh
number of the fluid (and therefore adjusting the ratio of convective
to conductive heat transfer), we can adjust the influence of the
nonlinear convection term in Eq~(\ref{eq:ns_momentum}). In Shadid's
work, Rayleigh numbers of up to \sn{1}{6} were used for this benchmark
on a $100 \times 100$ square mesh grid.

\subsubsection{Lid Driven Cavity Problem}
\label{subsubsec:lid_driven_cavity}
As an extension of the convection problem, the second benchmark
problem given by Ghia \citep{ghia_high-re_1982} adds a driver for the
flow to introduce higher Reynolds numbers into the system, providing
more inertial force to overcome the viscous forces in the fluid. The
setup for this problem is equally simple, containing only the
Dirichlet conditions as given in Figure~\ref{fig:lid_driven_cavity}
and is only applied to the mass and momentum equations on the unit
square.
\begin{figure}[t!]
  \begin{center}
    \scalebox{1.5}{
      \input{chapters/navier_stokes_solutions/lid_driven_cavity.pdftex_t} }
  \end{center}
  \caption{\textbf{Problem setup for the lid driven cavity benchmark.}
    \textit{Dirichlet conditions of zero are set for the velocity on
      the left and right and bottom while the Dirichlet condition set
      on the top provides a driving force on the fluid.}}
  \label{fig:lid_driven_cavity}
\end{figure}
The top boundary condition will provide a driver for the flow and its
variation will in turn vary the Reynolds number of the fluid. An
increased velocity will generate more inertial forces in the fluid,
which will overcome the viscous forces and again increase the
influence of the nonlinear terms in Eq~(\ref{eq:ns_momentum}). Shadid
also used a $100 \times 100$ grid for this benchmark problem with
Reynolds numbers up to \sn{1}{4}.

\subsubsection{Backward-Facing Step Problem}
\label{subsubsec:backward_facing_step}
The third benchmark was generated by Gartling in 1990 and consists of
both flow over a backward step and an outflow boundary condition
\citep{gartling_test_1990}. Using the mass and momentum equations
while neglecting the energy equation, this problem utilizes a longer
domain with a 1/30 aspect ratio with the boundary conditions as shown
in Figure~\ref{fig:backward_facing_step}.
\begin{figure}[t!]
  \begin{center}
    \scalebox{1.3}{
      \input{chapters/navier_stokes_solutions/backward_facing_step.pdftex_t} }
  \end{center}
  \caption{\textbf{Problem setup for the backward facing step
      benchmark.} \textit{Zero velocity boundary conditions are
      applied at the top and bottom of the domain while the outflow
      boundary condition on the right boundary is represented by zero
      stress tensor components in the direction of the flow. For the
      inlet conditions, the left boundary is split such that the top
      half has a fully formed parabolic flow profile and the bottom
      half has a zero velocity condition, simulating flow over a
      step.}}
  \label{fig:backward_facing_step}
\end{figure}
In this problem, the inflow condition is specified by a fully-formed
parabolic flow profile over a zero velocity boundary representing a
step. The flow over this step will generate a recirculating backflow
under the inlet flow towards the step. As in the lid driven cavity
problem, the nonlinear behavior of this benchmark and the difficulty
in obtaining a solution is dictated by the Reynolds number of the
fluid. In Shadid's work, a $20 \times 400$ non-square rectilinear grid
was used to discretize the domain with Reynolds number up to
\sn{5}{2}.
