\chapter{Two-Dimensional One-Speed Neutron Diffusion Model Problem}
\label{chap:diffusion_problem}
For many numerical experiments in this work, we choose the one-speed,
two-dimensional neutron diffusion equation as a model neutron
transport problem \citep{duderstadt_nuclear_1976}:
\begin{equation}
  -\boldsymbol{\nabla} \cdot D \boldsymbol{\nabla} \phi + \Sigma_a
  \phi = S\:,
  \label{eq:diffusion_eq}
\end{equation}
where $\phi$ is the neutron flux, $\Sigma_a$ is the absorption cross
section, and $S$ is the source of neutrons. In addition, $D$ is the
diffusion coefficient defined as:
\begin{equation}
  D = \frac{1}{3 ( \Sigma_t - \bar{\mu}\Sigma_s )}\:,
  \label{eq:diffusion_coeff}
\end{equation}
where $\Sigma_s$ is the scattering cross section, $\Sigma_t = \Sigma_a
+ \Sigma_s$ is the total cross section, and $\bar{\mu}$ is the cosine
of the average scattering angle. For simplicity, we will take
$\bar{\mu} = 0$ for our analysis giving $D=(3 \Sigma_t)^{-1}$. In
addition, to further simplify we will assume a homogeneous domain such
that the cross sections remain constant throughout. Doing this permits
us to rewrite Eq~(\ref{eq:diffusion_eq}) as:
\begin{equation}
  -D \boldsymbol{\nabla}^2 \phi + \Sigma_a \phi = S\:.
  \label{eq:diffusion_eq_simple}
\end{equation}

We choose a finite difference scheme on a square Cartesian grid to
discretize the problem. For the Laplacian, we choose the 9-point
stencil shown in Figure~\ref{fig:stencil} over a grid of size $h$
\citep{leveque_finite_2007}:
\begin{multline}
  \nabla^2_9\phi = \frac{1}{6h^2}[4 \phi_{i-1,j} + 4 \phi_{i+1,j}
    + 4 \phi_{i,j-1} + 4 \phi_{i,j+1} + \phi_{i-1,j-1}\\ +
    \phi_{i-1,j+1} + \phi_{i+1,j-1} + \phi_{i+1,j+1} - 20
    \phi_{i,j}]\:.
  \label{eq:nine_point_stencil}
\end{multline}
\begin{figure}[t!]
  \begin{center}
    \scalebox{1.25}{\input{chapters/spn_equations/stencil.pdftex_t}}
  \end{center}
  \caption{\textbf{Nine-point Laplacian stencil.}}
  \label{fig:stencil}
\end{figure}
We then have the following linear system to solve:
\begin{multline}
  -\frac{1}{6h^2}[4 \phi_{i-1,j} + 4 \phi_{i+1,j} + 4
    \phi_{i,j-1} + 4 \phi_{i,j+1} + \phi_{i-1,j-1}\\ + \phi_{i-1,j+1}
    + \phi_{i+1,j-1} + \phi_{i+1,j+1} - 20 \phi_{i,j}] + \Sigma_a
  \phi_{i,j} = s_{i,j}\:,
  \label{eq:fd_system}
\end{multline}
and in operator form:
\begin{equation}
  \ve{D}\boldsymbol{\phi}=\ve{s}\:,
  \label{eq:operator_system}
\end{equation}
where $\ve{D}$ is the diffusion operator, $\ve{s}$ is the source in
vector form and $\boldsymbol{\phi}$ is the vector of unknown fluxes.

To close the system, a set of boundary conditions is required. In the
case of a non-reentrant current condition applied to all global
boundaries of the domain, we choose the formulation of Duderstadt by
assuming the flux is zero at some ghost point beyond the
grid. Consider for example the equations on the $i=0$ boundary of the
domain:
\begin{multline}
  -\frac{1}{6h^2}[4 \phi_{-1,j} + 4 \phi_{1,j} + 4 \phi_{0,j-1} +
    4 \phi_{0,j+1} + \phi_{-1,j-1}\\ + \phi_{-1,j+1} + \phi_{1,j-1} +
    \phi_{1,j+1} - 20 \phi_{0,j}] + \Sigma_a \phi_{0,j} = s_{0,j}\:.
  \label{eq:x_min_bnd}
\end{multline}
Here we note some terms where $i=-1$ and therefore are representative
of grid points beyond the boundary of the domain. We set the flux at
these points to be zero, giving a valid set of equations for the $i=0$
boundary:
\begin{multline}
  -\frac{1}{6h^2}[4 \phi_{1,j} + 4 \phi_{0,j-1} + 4 \phi_{0,j+1}
    \\ + \phi_{-1,j+1} + \phi_{1,j-1} + \phi_{1,j+1} - 20 \phi_{0,j}]
  + \Sigma_a \phi_{0,j} = s_{0,j}\:.
  \label{eq:x_min_bnd_2}
\end{multline}
We repeat this procedure for the other boundaries of the domain. For
reflecting boundary conditions, the net current across a boundary is
zero.

