\chapter{Derivation of the Multigroup $SP_N$ Equations\ }
\label{chap:mg_spn_equations}
In the text, we formulated the $SP_N$ equations for a single neutron
energy. To expand these equations for multiple energies, we start by
stating the multigroup neutron transport equation for a single
dimension in planar geometry:
\begin{multline}
  \mu \frac{\partial}{\partial x} \psi^g(x,\mu) + \sigma^g(x)
  \psi^g(x,\mu) = \\ \sum_{g'=0}^{G} \int
  \sigma_s^{gg'}(x,\hat{\Omega}' \rightarrow \hat{\Omega})
  \psi^{g'}(x,\hat{\Omega}') d\Omega' + \frac{q^g(x)}{4 \pi}\:,
  \label{eq:cart_1d_multigroup}
\end{multline}
where $g$ denotes the group index of $0$ to $G$ groups, $G=N_g-1$, and
the integration of the scattering emission term over energy has been
replaced by a discrete summation. For scattering, $\sigma_s^{gg'}$
provides the probability of scattering at a particular angle from
group $g$ to $g'$. The result is an equation nearly identical in form
to Eq~(\ref{eq:cart_1d_transport}) where now instead of forming the
$SP_N$ equations for a single energy group, we form them for each of
the energy groups with group coupling occurring through the scattering
term. The multigroup $P_N$ equations are then:
\begin{equation}
   \frac{1}{2n+1} \frac{\partial}{\partial x}\Big[ (n+1) \phi^g_{n+1}
     + n \phi^g_{n-1} \Big] +
   \sum_{g'}(\sigma^g\delta_{gg'}-\sigma^{gg'}_{sn}) \phi^g_n =
   q\delta_{n0} \:,
  \label{eq:multigroup_pn_equations}
\end{equation}
for $n = 0,1,\dotsc,N$ where the flux and scattering moments are
defined in a group. We observe that a $N_g \times N_g$ scattering
matrix is formed:
\begin{equation}
  \mathbf{\Sigma}_n =
  \sum_{g'}(\sigma^g\delta_{gg'}-\sigma^{gg'}_{sn})\:,
  \label{eq:scattering_matrix}
\end{equation}
and when expanded gives:
%% Also grabbed this matrix from Tom's tech note.
\begin{equation}
  \mathbf{\Sigma}_n =
  \begin{bmatrix}
    (\sigma^0-\sigma_{sn}^{00}) & -\sigma_{sn}^{01} & \dots &
    -\sigma_{sn}^{0G} \\ &&&\\ -\sigma_{sn}^{10} &
    (\sigma^1-\sigma_{sn}^{11}) & \dots & -\sigma_{sn}^{1G}
    \\ &&&\\ \vdots & \vdots & \ddots & \vdots
    \\ &&&\\ -\sigma_{sn}^{G0} & -\sigma_{sn}^{G1} & \dots &
    (\sigma^G-\sigma_{sn}^{GG})
  \end{bmatrix}\:.
\end{equation}
It is also useful to combine the group flux moments and sources into a
single vector for more compact notation:
\begin{equation}
  \mathbf{\Phi_n} = (\phi^0_n\ \phi^1_n\ \cdots \phi^G_n )^T\:,
  \label{eq:group_flux_vector}
\end{equation}
\begin{equation}
  \mathbf{q} = (q^0\ q^1\ \cdots q^G )^T\:.
  \label{eq:group_source_vector}
\end{equation}
Next, we apply the $SP_N$ approximation to
Eq~(\ref{eq:multigroup_pn_equations}) in identical fashion to the
monoenergetic case. This gives:
\begin{multline}
  -\nabla \cdot \Bigg[\frac{n}{2n+1}\mathbf{\Sigma_{n-1}}^{-1} \nabla
    \Big(\frac{n-1}{2n-1} \mathbf{\Phi_{n-2}} +
    \frac{n}{2n-1}\mathbf{\Phi_n} \Big) \\+
    \frac{n+1}{2n+1}\mathbf{\Sigma_{n+1}}^{-1} \nabla
    \Big(\frac{n+1}{2n+3}\mathbf{\Phi_n} +
    \frac{n+2}{2n+3}\mathbf{\Phi_{n+2}}\Big) \Bigg] \\+
  \mathbf{\Sigma_n} \mathbf{\Phi_n} = \mathbf{q}
  \delta_{n0}\ \ \ \ \ \ \ \ \ n = 0,2,4,\cdots,N\:.
  \label{eq:multigroup_spn_equations}
\end{multline}
This adds more complexity than the monoenergetic formulation in that
all unknowns in this group of equations are now vector quantities and
scattering relationships are contained in matrices rather than a
scalar quantity. Because of this, the same sequence of variable
changes and algebra can be used to build a set of matrix equations,
this time in a block formulation:
\begin{equation}
  -\nabla \cdot \mathbb{D}_n \nabla \mathbb{U}_n + \sum_{m=1}^4
  \mathbb{A}_{nm} \mathbb{U}_m = \mathbb{Q}_n\:,
  \label{eq:spn_multigroup_system}
\end{equation}
where the definition of all quantities are the same with internal
scalar values replaced by the group-vector values. In addition,
$\mathbb{A}$ is now a block matrix of $N_g \times N_g$ sub-matrices
generated from the moment scattering matrices:
\begin{equation}
  \mathbf{A} =
  {\tiny \begin{bmatrix}
    (\mathbf{\Sigma}_0) &
    (-\frac{2}{3}\mathbf{\Sigma}_0) &
    (\frac{8}{15}\mathbf{\Sigma}_0) &
    (-\frac{16}{35}\mathbf{\Sigma}_0) \\
    %%
    &&&\\
    %%
    (-\frac{2}{3}\mathbf{\Sigma}_0) &
    (\frac{4}{9}\mathbf{\Sigma}_0 + \frac{5}{9}\mathbf{\Sigma}_2) &
    (-\frac{16}{45}\mathbf{\Sigma}_0 - \frac{4}{9}\mathbf{\Sigma}_2) &
    (\frac{32}{105}\mathbf{\Sigma}_0 + \frac{8}{21}\mathbf{\Sigma}_2) \\
    %%
    &&&\\
    %%
    (\frac{8}{15}\mathbf{\Sigma}_0) &
    (-\frac{16}{45}\mathbf{\Sigma}_0 - \frac{4}{9}\mathbf{\Sigma}_2) &
    (\frac{64}{225}\mathbf{\Sigma}_0 + \frac{16}{45}\mathbf{\Sigma}_2 + \frac{9}{25}\mathbf{\Sigma}_4) &
    (-\frac{128}{525}\mathbf{\Sigma}_0 - \frac{32}{105}\mathbf{\Sigma}_2 - \frac{54}{175}\mathbf{\Sigma}_4)
    \\
    %%
    &&&\\
    %%
    (-\frac{16}{35}\mathbf{\Sigma}_0) &
    (\frac{32}{105}\mathbf{\Sigma}_0 + \frac{8}{21}\mathbf{\Sigma}_2) &
    (-\frac{128}{525}\mathbf{\Sigma}_0 - \frac{32}{105}\mathbf{\Sigma}_2 - \frac{54}{175}\mathbf{\Sigma}_4)
    &
    (\frac{256}{1225}\mathbf{\Sigma}_0 + \frac{64}{245}\mathbf{\Sigma}_2 +
    \frac{324}{1225}\mathbf{\Sigma}_4 + \frac{13}{49}\mathbf{\Sigma}_6)
  \end{bmatrix}}\:.
  \label{eq:A_block_matrix}
\end{equation}
