\chapter{Finite Volume Discretization for the $SP_N$ Equations\ }
\label{chap:spn_spatial_discretization}
The moment-discretized form of the $SP_N$ equations given by
Eq~(\ref{eq:spn_matrix}) for the monoenergetic case has yet to have
the spatial components of the solution and the derivative operators
discretized. To achieve this, we choose a conservation law approach,
using Eq~(\ref{eq:spn_matrix}) as the balance equation, to yield a
finite volume calculation \citep{leveque_finite_2002}. To do this, we
first define a control volume about point $(i,j,k)$ in the domain as a
cell in a Cartesian mesh as shown in Figure~\ref{fig:mesh_cell}).
\begin{figure}[h!]
  \begin{center}
    \scalebox{1.5}{
    \input{backmatter/mesh_cell.pdftex_t} }
  \end{center}
  \caption{\textbf{Cartesian mesh cell used for the spatial
      discretization of the $SP_N$ equations.} \textit{The cell is
      centered about $(i,j,k)$ and has a volume $V_{ijk} = \Delta_i
      \Delta_j \Delta_k$.}}
  \label{fig:mesh_cell}
\end{figure}

To begin, we first define the neutron current in the $n^th$
pseudo-moment, $u_n$ using Fick's law as:
\begin{equation}
  J_n = -D_n \nabla u_n\:,
  \label{eq:neutron_current}
\end{equation}
where $J_n$ is the moment current and $D_n$ is the moment diffusion
coefficient. Next, we substitue Eq~(\ref{eq:neutron_current}) into
Eq~(\ref{eq:spn_matrix}) to get the balance equation:
\begin{equation}
    \nabla \cdot J_n + \sum_{m=1}^4 A_{nm} u_m = q_n\ \ \ \ \ \ \ n =
    1,2,3,4\:,
  \label{eq:neutron_balance}
\end{equation}
with the current term accounting for losses, $q_n$ accounting for
neutron generation, and $\sum_{m=1}^4 A_{nm} u_m$ accounting for
moment-to-moment transport. To enforce conservation, we integrate
moments over the control volume of cell $ijk$ in Figure~\ref{fig:mesh_cell}):
\begin{equation}
    \int_{ijk} \Bigg[\nabla \cdot J_n + \sum_{m=1}^4 A_{nm} u_m - q_n
      \Bigg] dV = 0 \ \ \ \ \ \ \ n = 1,2,3,4\:.
  \label{eq:neutron_conservation}
\end{equation}
If for a given quantity $X$, we define the cell-averaged quantity
$X_{ijk}$ as:
\begin{equation}
  X_{ijk} = \frac{\int_{ijk} X dV }{V_{ijk}}\:,
  \label{eq:cell_averaged_quant}
\end{equation}
then Eq~(\ref{eq:neutron_conservation}) can be written as:
\begin{equation}
  \int_{ijk} \nabla \cdot J_n dV + \sum_{m=1}^4 A_{nm,ijk} u_{m,ijk} = q_{n,ijk}\:,
  \label{eq:neutron_conservation_2}
\end{equation}
with $n$ now implied to be 1,2,3 or 4 and the moment matrix elements,
moments, and sources now defined as volume averaged terms in the mesh
cell. Using the divergence theorem, we can rewrite the remaining
integral in terms of a discrete sum of area-weighted averages over the
faces of the cell:
\begin{equation}
  \int_{ijk} \nabla \cdot J_n dV = \int_{\partial} J_n \cdot \hat{n}
  dA = \sum_{f=1}^6 J_{n,f} \cdot \hat{n}_f A_f\:,
\label{eq:face_currents}
\end{equation}
where $f$ is the face index, $\hat{n}_f$ is the unit normal vector for
that face and $A_f$ is the area of the face. Inserting this into
Eq~(\ref{eq:neutron_conservation_2}) and expanding the face current
summation provides a discrete form of the $SP_N$ equations:
\begin{multline}
(J_{n,i+1/2} - J_{n,i-1/2}) \Delta_j \Delta_k + (J_{n,j+1/2} -
  J_{n,j-1/2}) \Delta_i \Delta_k +\\ (J_{n,k+1/2} - J_{n,k-1/2})
  \Delta_i \Delta_j + \sum_{m=1}^4 A_{nm,ijk} u_{m,ijk} V_{ijk} =
  q_{n,ijk} V_{ijk}\:,
  \label{eq:discrete_current}
\end{multline}
For the discretization we desire only cell-centered quanities and
therefore we aim to elminate all $J_{n,l \pm 1/2}$ terms from the
discretization where $l = \{i,j,k\}$. To do this, we enforce
continuity of the moment flux and the moment currents across cell
boundaries such that $J_{n,l \pm 1/2}^+ = J_{n,l \pm
  1/2}^-$. Considering $J_{n,l+1/2}$, using
Eq~(\ref{eq:neutron_current}) we then have:
\begin{equation}
  J_{n,l+1/2} = -2 D_{n,l+1/2}\frac{u_{n,l+1} - u_{n,l}}{\Delta_{l+1}+
    \Delta_{l}}\:,
  \label{eq:face_current}
\end{equation}
\begin{equation}
  J_{n,l+1/2}^+ = -2 D_{n,l+1}\frac{u_{n,l+1} - u_{n,l+1/2}}{\Delta_{l+1}}\:,
  \label{eq:face_current_plus}
\end{equation}
\begin{equation}
  J_{n,l+1/2}^- = -2 D_{n,l}\frac{u_{n,l+1/2} - u_{n,l}}{\Delta_{l}}\:.
  \label{eq:face_current_minus}
\end{equation}
To eliminate $D_{n,l+1/2}$ in Eq~(\ref{eq:face_current}), we equate
Eqs~(\ref{eq:face_current_plus}) and (\ref{eq:face_current_minus}) to
enforce continuity of the neutron current:
\begin{equation}
  -2 D_{n,l+1}\frac{u_{n,l+1} - u_{n,l+1/2}}{\Delta_{l+1}} = -2
  D_{n,l}\frac{u_{n,l+1/2} - u_{n,l}}{\Delta_{l}}\:.
  \label{eq:face_current_equality}
\end{equation}
Solving for $u_{n,l+1/2}$ gives:
\begin{equation}
  u_{n,l+1/2} = \frac{\Delta_l D_{n,l+1} u_{n,l+1} + \Delta_{l+1}
    D_{n,l} u_{n,l}}{\Delta_l D_{n,l+1} + \Delta_{l+1} D_{n,l}}\:.
  \label{eq:face_flux}
\end{equation}
Next, we equate Eq~(\ref{eq:face_current}) to
Eq~(\ref{eq:face_current_plus}):
\begin{equation}
-2 D_{n,l+1/2}\frac{u_{n,l+1} - u_{n,l}}{\Delta_{l+1} + \Delta_{l}} =
-2 D_{n,l+1}\frac{u_{n,l+1} - u_{n,l+1/2}}{\Delta_{l+1}}\:,
  \label{eq:face_current_solve}
\end{equation}
and rearrange:
\begin{equation}
  \Delta_{l+1} D_{n,l+1/2} (u_{n,l+1} - u_{n,l}) = (\Delta_{l+1} +
  \Delta_l) D_{n,l+1}u_{n,l+1} - (\Delta_{l+1} + \Delta_l)
  D_{n,l+1}u_{n,l+1/2}\:.
  \label{eq:face_current_solve_2}
\end{equation}
Inserting Eq~(\ref{eq:face_flux}) into
Eq~(\ref{eq:face_current_solve_2}) gives:
\begin{multline}
  \Delta_{l+1} D_{n,l+1/2} (\Delta_l D_{n,l+1} + \Delta_{l+1} D_{n,l})
  (u_{n,l+1} - u_{n,l}) = \\(\Delta_{l+1} + \Delta_l) D_{n,l+1}
  (\Delta_l D_{n,l+1} + \Delta_{l+1} D_{n,l}) u_{n,l+1} -\\
  (\Delta_{l+1} + \Delta_l) D_{n,l+1}(\Delta_l D_{n,l+1} u_{n,l+1} +
  \Delta_{l+1} D_{n,l} u_{n,l})\:.
  \label{eq:face_current_solve_3}
\end{multline}
Expanding gives 5 terms:
\begin{multline}
  \Delta_{l+1} D_{n,l+1/2} (\Delta_l D_{n,l+1} + \Delta_{l+1} D_{n,l})
  (u_{n,l+1} - u_{n,l}) = \\(\Delta_{l+1} + \Delta_l) D_{n,l+1}
  \Delta_l D_{n,l+1} u_{n,l+1} + (\Delta_{l+1} + \Delta_l) D_{n,l+1}
  \Delta_{l+1} D_{n,l} u_{n,l+1} -\\ (\Delta_{l+1} + \Delta_l)
  D_{n,l+1} \Delta_l D_{n,l+1} u_{n,l+1} - (\Delta_{l+1} + \Delta_l)
  D_{n,l+1} \Delta_{l+1} D_{n,l} u_{n,l}\:,
  \label{eq:face_current_solve_4}
\end{multline}
two of which cancel, giving:
\begin{multline}
  \Delta_{l+1} D_{n,l+1/2} (\Delta_l D_{n,l+1} + \Delta_{l+1} D_{n,l})
  (u_{n,l+1} - u_{n,l}) = \\(\Delta_{l+1} + \Delta_l) \Delta_{l+1}
  D_{n,l+1} D_{n,l} ( u_{n,l+1} - u_{n,l} ) \:.
  \label{eq:face_current_solve_5}
\end{multline}
Solving Eq~(\ref{eq:face_current_solve_5}) for $D_{n,l+1/2}$ then
gives the face diffusion coefficient in terms of cell-centered
coefficients:
\begin{equation}
  D_{n,l+1/2} = \frac{\Delta_{l+1} D_{n,l+1} D_{n,l} + \Delta_l
    D_{n,l+1} D_{n,l}}{\Delta_l D_{n,l+1} + \Delta_{l+1} D_{n,l}} \:.
  \label{eq:face_current_solve_6}
\end{equation}
With this diffusion coefficient, we can then finish the derivation of
the cell-face currents by inserting Eq~(\ref{eq:face_current_solve_6})
into Eq~(\ref{eq:face_current}) giving:
\begin{equation}
  J_{n,l+1/2} = -2 \frac{D_{n,l+1} D_{n,l}}{\Delta_l D_{n,l+1} +
    \Delta_{l+1} D_{n,l}} (u_{n,l+1} - u_{n,l})\:.
  \label{eq:face_current_solve_7}
\end{equation}
Analagously, for cell face in the opposite direction, we have:
\begin{equation}
  J_{n,l-1/2} = -2 \frac{D_{n,l} D_{n,l-1}}{\Delta_{l-1} D_{n,l} +
    \Delta_{l} D_{n,l-1}} (u_{n,l} - u_{n,l-1})\:.
  \label{eq:face_current_solve_8}
\end{equation}

Before inserting the derived face fluxes into
Eq~(\ref{eq:discrete_current}), we divide by the cell volume to get:
\begin{multline}
(J_{n,i+1/2} - J_{n,i-1/2}) \frac{1}{\Delta_i} + (J_{n,j+1/2} -
  J_{n,j-1/2}) \frac{1}{\Delta_j} +\\ (J_{n,k+1/2} - J_{n,k-1/2})
  \frac{1}{\Delta_k} + \sum_{m=1}^4 A_{nm,ijk} u_{m,ijk} =
  q_{n,ijk}\:,
  \label{eq:discrete_current_2}
\end{multline}
Based on the face currents computed, we identify the following
coefficients that will appear in the moment equations:
\begin{equation}
  C_{n,l+1} = \frac{2}{\Delta_l}\Bigg( \frac{D_{n,l+1}
    D_{n,l}}{\Delta_l D_{n,l+1} + \Delta_{l+1} D_{n,l}}\Bigg)\:,
  \label{eq:moment_coeff_plus}
\end{equation}
\begin{equation}
  C_{n,l-1} = \frac{2}{\Delta_l}\Bigg( \frac{D_{n,l}
    D_{n,l-1}}{\Delta_{l-1} D_{n,l} + \Delta_{l} D_{n,l-1}}\Bigg)\:.
  \label{eq:moment_coeff_minus}
\end{equation}
Inserting these coefficients into Eqs~(\ref{eq:face_current_solve_7})
and (\ref{eq:face_current_solve_8}) gives:
\begin{equation}
  J_{n,l+1/2} = - \Delta_l C_{n_l+1} (u_{n,l+1} - u_{n,l})\:.
  \label{eq:face_current_solve_9}
\end{equation}
\begin{equation}
  J_{n,l-1/2} = - \Delta_l C_{n_l-1} (u_{n,l} - u_{n,l-1})\:.
  \label{eq:face_current_solve_10}
\end{equation}
Applying Eqs~(\ref{eq:face_current_solve_9}) and
(\ref{eq:face_current_solve_10}) Eq~(\ref{eq:discrete_current_2})
gives the final spatially discretized $SP_N$ equations:
\begin{multline}
  - C_{n,i+1} u_{n_i+1} - C_{n,i-1} u_{n_i-1} - C_{n,j+1} u_{n_j+1} -
  C_{n,j-1} u_{n_j-1} - C_{n,k+1} u_{n_k+1} - C_{n,k-1} u_{n_k-1} +\\
  \sum_{m=1}^4 \Big[A_{nm,ijk} +
    (C_{n,i+1}+C_{n,j+1}+C_{n,k+1}+C_{n,i-1}+C_{n,j-1}+C_{n,k-1})\delta_{nm}\Big]
  u_{m,ijk} = q_{n,ijk}\:.
  \label{eq:discrete_space_spn}
\end{multline}

On the boundaries of the domain, special treatment of the face
currents must be made as their are no adjacent cells with which to
close the system. Starting with Eq~(\ref{eq:spn_bnd_matrix}), we again
insert the neutron current given by Eq~(\ref{eq:neutron_current}):
\begin{equation}
  \mathbf{n} \cdot J_n + \sum_{m=1}^4 B_{nm} u_m = s_n\:.
  \label{eq:boundary_current_system}
\end{equation}
On the low boundary we then have:
\begin{equation}
  J_{n,1/2} = s_{n,1} - \sum_{m=1}^4 B_{nm,1/2} u_{m,1/2}\:,
  \label{eq:low_bnd_current}
\end{equation}
or in terms of moments:
\begin{equation}
  J_{n,1/2} = -2 D_{n,1}\frac{u_{n,1} - u_{n,1/2}}{\Delta_1}\:.
  \label{eq:low_bnd_current_2}
\end{equation}
Equating Eqs~(\ref{eq:low_bnd_current}) and
(\ref{eq:low_bnd_current_2}) gives an extra equation for the boundary
moments:
\begin{equation}
  s_{n,1} = \sum_{m=1}^4 B_{nm,1/2} u_{m,1/2} -2 D_{n,1}\frac{u_{n,1}
    - u_{n,1/2}}{\Delta_1}\:,
  \label{eq:low_bnd_current_3}
\end{equation}
or
\begin{equation}
  s_{n,1} = \sum_{m=1}^4 \Big[B_{nm,1/2} + \frac{2
      D_{n,1}}{\Delta_1}\delta_{mn} \Big] u_{m,1/2} -2
  \frac{D_{n,1}}{\Delta_1} u_{n,1}\:.
  \label{eq:low_bnd_current_4}
\end{equation}
To use these boundary fluxes to close the system, we insert
Eq~(\ref{eq:low_bnd_current_2}) into Eq~(\ref{eq:discrete_current_2})
and then add Eq~(\ref{eq:low_bnd_current_4}) to the system to
eliminate the boundary currents introduced by
Eq~(\ref{eq:low_bnd_current_2}).

For the multigroup $SP_N$ equations, the spatial discretization is
identical, this time with the multigroup moment vectors where now:
\begin{multline}
  - \mathbb{C}_{n,i+1} \mathbb{U}_{n_i+1} - \mathbb{C}_{n,i-1}
  \mathbb{U}_{n_i-1} - \mathbb{C}_{n,j+1} \mathbb{U}_{n_j+1} -
  \mathbb{C}_{n,j-1} \mathbb{U}_{n_j-1} - \mathbb{C}_{n,k+1}
  \mathbb{U}_{n_k+1} - \mathbb{C}_{n,k-1} \mathbb{U}_{n_k-1}
  +\\ \sum_{m=1}^4 \Big[\mathbb{A}_{nm,ijk} +
    (\mathbb{C}_{n,i+1}+\mathbb{C}_{n,j+1}+\mathbb{C}_{n,k+1}+
    \mathbb{C}_{n,i-1}+\mathbb{C}_{n,j-1}+\mathbb{C}_{n,k-1})\delta_{nm}\Big]
  \mathbb{U}_{m,ijk} = \mathbb{Q}_{n,ijk}\:,
  \label{eq:discrete_mg_space_spn}
\end{multline}
and the coefficients are:
\begin{equation}
  \mathbb{C}_{n,l+1} = \frac{2}{\Delta_l}\Bigg( \frac{\mathbb{D}_{n,l+1}
    \mathbb{D}_{n,l}}{\Delta_l \mathbb{D}_{n,l+1} + \Delta_{l+1} \mathbb{D}_{n,l}}\Bigg)\:,
  \label{eq:mg_coeff_plus}
\end{equation}
\begin{equation}
  \mathbb{C}_{n,l-1} = \frac{2}{\Delta_l}\Bigg( \frac{\mathbb{D}_{n,l}
    \mathbb{D}_{n,l-1}}{\Delta_{l-1} \mathbb{D}_{n,l} + \Delta_{l}
    \mathbb{D}_{n,l-1}}\Bigg)\:.
  \label{eq:mg_coeff_minus}
\end{equation}
The boundary conditions are analagously formulated with the multigroup
matrices/vectors.
