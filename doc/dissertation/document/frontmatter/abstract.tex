Predictive modeling and simulation is an important and developing
component of modern nuclear reactor analysis. New high performance
computing hardware and algorithms that leverage this hardware are
needed to help meet the goals and requirements of engineering
analysis. Current practices in neutron transport and fluid flow
leverage methods based on subspace iterations to achieve good
convergence properties and quality parallel scaling on large
machines. However, the next generation of machines is expected to
possess both high levels of concurrency and low memory availability
per core compared to today's machines. Therefore, we seek new
algorithms to solve these problems in reactor analysis that have the
potential to perform well on forthcoming hardware with the goal of
improving both iterative performance and parallel scalability.

This work researches and develops Monte Carlo Synthetic Acceleration
(MCSA) methods as a new class of solution techniques for discrete
neutron transport and fluid flow problems. Monte Carlo Synthetic
Acceleration methods use a traditional Monte Carlo process to
approximate the solution to the discrete problem as a means of
accelerating traditional fixed-point methods. To apply these methods
to neutronics and fluid flow and determine the feasibility of these
methods on modern hardware, three complementary research and
development exercises are performed.

First, solutions to the $SP_N$ discretization of the linear Boltzmann
neutron transport equation are obtained using Monte Carlo Synthetic
Acceleration with a difficult criticality calculation for a light
water reactor fuel assembly used as the driving problem. It was
discovered that basic Jacobi-based preconditioning as used in previous
work was not sufficient to achieve convergence due to the dominating
effect of the light water moderator. To enable MCSA as a solution
technique a group of modern preconditioning strategies are
researched. The new preconditioning techniques, however, create
unwanted growth in both memory consumption and computation time for
these problems. Monte Carlo Synthetic Acceleration was verified to
produce the correct solution for the fuel assembly criticality problem
when compared to conventional Krylov methods and demonstrated improved
iterative performance over GMRES by converging in fewer iterations
when using the same preconditioning.

Second, solutions to the full Navier-Stokes equations were obtained by
developing the Forward-Automated Newton-MCSA (FANM) method for
nonlinear systems based on Newton's method. Three difficult fluid
benchmark problems in both convective and driven flow regimes were
used to drive the research and development of the method. It was found
for all benchmarks that FANM produced identical solutions to modern
Newton-Krylov methods. For each benchmark, performance comparisons
between FANM and Newton-Krylov were also conducted. For 8 out of 12
benchmark cases, it was found that FANM had better iterative
performance than the Newton-Krylov method by converging the nonlinear
residual in fewer linear solver iterations with the same
preconditioning.

Third, a new domain decomposed algorithm to parallelize MCSA aimed at
leveraging leadership-class computing facilities was developed by
utilizing parallel strategies from the radiation transport
community. The new algorithm utilizes the Multiple-Set
Overlapping-Domain strategy in an attempt to reduce parallel overhead
and add a natural element of replication to the algorithm. A large
number of parallel scaling studies using a neutron diffusion problem
were performed on the Titan Cray XK7 machine at Oak Ridge National
Laboratory with up to 65,356 cores used in the strong scaling studies
and 131,072 cores used in the weak scaling studies. It was found that
for the current implementation of MCSA, both weak and strong scaling
were superior to production implementations of Krylov methods. In
addition, the replication in the algorithm was able to boost parallel
efficiencies by up to 20\% at large core counts in the strong scaling
studies. By using the principle of superposition, replicating the
Monte Carlo sequence was also demonstrated to enhance time to
solution. Finally, it was discovered that MCSA exhibits the most
scalable behavior when it is formulated as a stochastic realization of
an additive Schwarz method.

Each of these research and development activities meet the goals of
this work by producing methods that in many cases were demonstrated to
converge in fewer iterations than conventional methods and exhibit
better parallel efficiencies at high levels of concurrency. All the
components of this work are required to enable future simulation of
nonlinear transport phenomena on the next generation of hardware. To
enable high performance simulations with FANM, we require the MCSA
techniques developed in the research on the neutron transport problem
and in turn effectively leveraging MCSA on leadership-class hardware
requires the new parallel algorithm developed by this work. With the
further development of these methods based on this work, future
nuclear reactor simulations may be enabled that have an improved time
to solution and more efficiently utilize the available computational
resources.
