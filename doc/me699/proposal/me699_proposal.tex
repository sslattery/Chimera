%        File: me699_proposal.tex
%
\documentclass[letterpaper,12pt]{article}
\usepackage[top=1.0in,bottom=1.0in,left=1.25in,right=1.25in]{geometry}
\usepackage{verbatim}
\usepackage{amsthm} 
\usepackage{tmadd,tmath}
\usepackage[mathcal]{euscript}
\usepackage{amssymb}
\usepackage{graphicx}
\usepackage{longtable}
\usepackage{amsfonts}
\usepackage{amsmath}
\usepackage[usenames]{color}
\usepackage[
naturalnames = true, 
colorlinks = true, 
linkcolor = black,
anchorcolor = black,
citecolor = black,
menucolor = black,
urlcolor = blue
]{hyperref}

%%---------------------------------------------------------------------------%%
\author{Stuart R. Slattery
\\ \href{mailto:sslattery@wisc.edu}{\texttt{sslattery@wisc.edu}}
}

\date{\today} \title{ME699 Proposal: Fully Implicit Solutions to the
  Steady and Incompressible Navier-Stokes Equations}
\begin{document}
\maketitle

%%---------------------------------------------------------------------------%%
\section{Proposal}
Fully implicit formulations of the Navier-Stokes equations are
becoming increasingly popular due to their robustness over
conventional segregated methods that arises from a fully consistent
problem formulation. Solutions with inexact Newton methods have
proliferated due to advancements in both hardware and numerical
methods for solving sparse linear sytems and preconditioning
\cite{shadid_1997,mchugh_1994,gropp_2001,knoll_2004}. To study the
details of how fully implicit solutions are implemented in practice,
and independent study is posed to develop a piece of software aimed at
generating fully implicit solutions to the steady and incompressible
Navier-Stokes equations.

\subsection{Model Problem}
We choose the steady incompressible Navier-Stokes equations as model
problems in residual form to solve for the state vector
$\{\ve{u},P,T\}$ where $\ve{u}$ is the fluid velocity, $P$ is the
fluid pressure, and $T$ is the fluid temperature. As formulated in
\cite{shadid_1997}, we consider vector forms of momentum transport:

\begin{equation}
  \rho \ve{u} \cdot \nabla \ve{u} - \nabla \cdot \ve{T} - \rho \ve{g}
  = \ve{0} \:,
  \label{eq:momentum_transport}
\end{equation}

continuity:

\begin{equation}
  \nabla \cdot \ve{u} = \ve{0} \:,
  \label{eq:continuity}
\end{equation}

and energy transport with the effects of viscous dissipation neglected
for low Mach number flow:

\begin{equation}
  \rho C_p \ve{u} \cdot \nabla T + \nabla \ve{q} = \ve{0} \:,
  \label{eq:energy_transport}
\end{equation}

where $\rho$ is the fluid density, $\ve{g}$ the gravity vector and
$C_p$ the specific heat of the fluid at constant pressure. To close
the system we define the stress tensor:

\begin{equation}
  \ve{T} = -P \ve{I} + \mu [ \nabla \ve{u} \nabla \ve{u}^{T} ] \:,
  \label{eq:stress_tensor}
\end{equation}

and the heat flux:

\begin{equation}
  \ve{q} = -\kappa \nabla T
  \label{eq:heat_flux}
\end{equation}

where $\ve{I}$ is the identity matrix, $\mu$ is the dynamic
viscosity of the fluid, and $\kappa$ is the thermal conductivity of the
fluid. These fluid properties will be assumed constant and boundary
conditions will be chosen as a combination of Neumann and Dirichlet
conditions depending on the problem chosen.

\subsection{Discretization}
Reference \cite{shadid_1997} uses a pressure stabilized
Petrov-Galerkin finite element formulation of these equations
\cite{hughes_1986}. Depending on the software infrastructure
developed, these may or may not be as easy to formulate as the finite
volume discretizations in \cite{mchugh_1994} due to the availabilty of
finite element discretization libraries, finite element assembly
engines, and unstructured mesh databases in the Trilinos scientific
computing libraries \cite{trilinos_2005}. It will likely be easiest to
start with the finite volume approach.

\subsection{Solution Method}
We will choose an inexact Newton method for the nonlinear scheme to
solve these equations \cite{dembo_1982}. Of particular interest will
be investigating how the forcing term affects convergence of the
nonlinear resdiual \cite{eisenstat_1996}. In addition, if convergence
issues are noted, globalization methods may be investigated to improve
the Newton correction \cite{pawlowski_2006}. The linear solver for the
Newton method will be GMRES \cite{saad_1986} as its convergence
properties for these types of systems are well documented and its long
recurrence relation has been shown to improve robustness and preserve
a monotonically decreasing residuals in nonlinear solves. We will seek
to solve these equations using parallel methods provided by well
established framework libraries in the Trilinos suite of scientific
computing libraries \cite{trilinos_2005}.

\subsection{Benchmark Problems}
To verify the implementation, three widely referenced two-dimensional
benchmark problems will be assessed using the resulting software. The
natural convection cavity problem \cite{davis_1983}, the driven cavity
problem \cite{ghia_1982}, and flow over a backwards step
\cite{gartling_1990}.

%%---------------------------------------------------------------------------%%
\pagebreak
\bibliographystyle{ieeetr}
\bibliography{references}
\end{document}


