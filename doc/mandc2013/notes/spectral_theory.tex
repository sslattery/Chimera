%        File: latexdoc.tex
%     Created: Friday March 4 12:24:15 2011
% Last Change: Friday March 4 12:24:20 2011
%
\documentclass[letterpaper,12pt]{article}
\usepackage[top=1.0in,bottom=1.0in,left=1.25in,right=1.25in]{geometry}
\usepackage{verbatim}
\usepackage{amssymb}
\usepackage{graphicx}
\usepackage{longtable}
\usepackage{amsfonts}
\usepackage{amsmath}
\usepackage{amsthm}
\usepackage[mathcal]{euscript}
\usepackage{tabularx}
\usepackage{cite}
\usepackage{c++}
\usepackage{tmadd,tmath}
\usepackage[usenames]{color}
\usepackage[
naturalnames = true, 
colorlinks = true, 
linkcolor = black,
anchorcolor = black,
citecolor = black,
menucolor = black,
urlcolor = blue
]{hyperref}
\usepackage{listings}
\usepackage{textcomp}
\definecolor{listinggray}{gray}{0.9}
\definecolor{lbcolor}{rgb}{0.9,0.9,0.9}
\lstset{
  backgroundcolor=\color{lbcolor},
  tabsize=4,
  rulecolor=,
  language=c++,
  basicstyle=\scriptsize,
  upquote=true,
  aboveskip={1.5\baselineskip},
  columns=fixed,
  showstringspaces=false,
  extendedchars=true,
  breaklines=true,
  prebreak =
  \raisebox{0ex}[0ex][0ex]{\ensuremath{\hookleftarrow}},
  frame=single,
  showtabs=false,
  showspaces=false,
  showstringspaces=false,
  identifierstyle=\ttfamily,
  keywordstyle=\color[rgb]{0,0,1},
  commentstyle=\color[rgb]{0.133,0.545,0.133},
  stringstyle=\color[rgb]{0.627,0.126,0.941},
}

%%---------------------------------------------------------------------------%%
\author{Stuart R. Slattery
\\ \href{mailto:sslattery@wisc.edu}{\texttt{sslattery@wisc.edu}}
}

\date{\today}
\title{A Spectral Analysis of the Domain Decomposed Adjoint
  Neumann-Ulam Method}
\begin{document}
\maketitle

%%---------------------------------------------------------------------------%%
\begin{abstract}

  The domain decomposed behavior of the adjoint Neumann-Ulam method
  for solving linear operator equations is analyzed using the spectral
  properties of the linear operator. Relationships for the average
  length of the adjoint random walks and the fraction of histories
  leaking from a domain are made with respect to the eigenvalues of
  the linear operator. The one-speed, two-dimensional neutron
  diffusion equation is used as a model problem to test the spectral
  theory for symmetric operators. These relationships will serve as
  guidelines for selecting an appropriate parallel algorithm strategy
  and provide a basis for performance models for future parallel
  implementations of the adjoint Neumann-Ulam method.

\end{abstract}

%%---------------------------------------------------------------------------%%
\section{The Adjoint Neumann-Ulam Method}
We seek solutions of the general linear problem in the following form:
\begin{equation}
  \ve{A} \ve{x} = \ve{b}\:,
  \label{eq:linear_problem}
\end{equation}
where $\ve{A} \in \mathbb{R}^{N \times N}$ is a linear operator such
that $\ve{A} : \mathbb{R}^{N} \rightarrow \mathbb{R}^{N}$, $\ve{x} \in
\mathbb{R}^N$ is the solution vector, and $\ve{b} \in \mathbb{R}^N$ is
the forcing term. We choose the adjoint method Monte Carlo method to
invert the linear operator. We begin by defining the linear system
adjoint to Eq~(\ref{eq:linear_problem}):
\begin{equation}
  \ve{A}^T \ve{y} = \ve{d}\:,
  \label{eq:adjoint_linear_problem}
\end{equation}
where $\ve{y}$ and $\ve{d}$ are the adjoint solution and source
respectively and $\ve{A}^T$ is the adjoint operator. We can split this
equation by defining the following inner product equivalence
\cite{spanier_monte_1969}:
\begin{equation}
  \langle \ve{A}^T \ve{x}, \ve{y} \rangle = \langle \ve{x}, \ve{A}
  \ve{y} \rangle\:.
  \label{eq:adjoint_operator_product}
\end{equation}
With this statement we can then define the split equation:
\begin{equation}
  \ve{y} = \ve{H}^T \ve{y} + \ve{d}\:.
  \label{eq:adjoint_split_system}
\end{equation}
For convergence, we require that the spectral radius of $\ve{H}$ must
remain less than 1 as $\ve{H}^T$ contains the same eigenvalues and
therefore has the same spectral radius. From this definition it
follows that:
\begin{equation}
  \langle \ve{x}, \ve{d} \rangle = \langle \ve{y}, \ve{b} \rangle\:.
  \label{eq:adjoint_vector_relation}
\end{equation}
Using these definitions, we can derive an estimator from the adjoint
method that will also give the solution vector, $\ve{x}$. As with the
direct method, we can acquire the adjoint solution by forming the
Neumann series by writing Eq~(\ref{eq:adjoint_split_system}) as:
\begin{equation}
  \ve{y} = (\ve{I} - \ve{H}^T)^{-1} \ve{d}\:,
  \label{eq:adjoint_split_system_2}
\end{equation}
which in turn yields the Neumann series using the adjoint operator:
\begin{equation}
  \ve{y} = \sum_{k=0}^{\infty} (\ve{H}^T)^k\ve{d}\:.
  \label{eq:adjoint_neumann_series}
\end{equation}
We expand this summation to again yield a series of transitions that
can be approximated by a Monte Carlo random walk sequence, this time
forming the Neumann series in reverse order:
\begin{equation}
  y_i = \sum_{k=0}^{\infty}\sum_{i_1}^{N}\sum_{i_2}^{N}\ldots
  \sum_{i_k}^{N}h_{i_k,i_{k-1}}\ldots h_{i_2,i_1} h_{i_1,i} d_{i_k}\:.
  \label{eq:adjoint_neumann_solution}
\end{equation}
We can readily build an estimator for the adjoint solution from this
series expansion, but we instead desire the direct solution. We
achieve this by using Eq~(\ref{eq:adjoint_vector_relation}) as a
constraint. Here we have 2 unknowns, $\ve{y}$ and $\ve{d}$ and
therefore we require a second constraint to close the system. As a
second constraint we select:
\begin{equation}
  \ve{d} = \boldsymbol{\delta}_i\:,
  \label{eq:adjoint_second_constraint}
\end{equation}
where the $k^{th}$ component of the vector $\boldsymbol{\delta}_i$ is
the Dirac delta function $\delta_{i_k,i}$. If we apply
Eq~(\ref{eq:adjoint_second_constraint}) to our first constraint
Eq~(\ref{eq:adjoint_vector_relation}), we get the following convenient
outcome:
\begin{equation}
  \langle \ve{y}, \ve{b} \rangle = \langle \ve{x},
  \boldsymbol{\delta}_i \rangle = x_i \:,
  \label{eq:inner_product_constraint}
\end{equation}
meaning that if we compute the inner product of the direct source and
the adjoint solution using a delta function source, we recover the
direct solution we are looking for 

In terms of particle transport, this adjoint method is equivalent to a
traditional forward method. As a result of using the adjoint system,
we modify our probabilities and weights using the \textit{adjoint
  Neumann-Ulam decomposition} of $\ve{H}$:
\begin{equation}
  \ve{H}^{T} = \ve{P} \circ \ve{W}\:,
  \label{eq:adjoint_neumann_ulam}
\end{equation}
where now we are forming the decomposition with respect to the
transpose of $\ve{H}$. We then follow the same procedure as the direct
method for forming the probability and weight matrices in the
decomposition. Using the adjoint form, probabilities should instead be
column-scaled:
\begin{equation}
  p_{ij} = \frac{|h_{ji}|}{\sum_j |h_{ji}|}\:,
  \label{eq:adjoint_probability}
\end{equation}
such that we expect to select a new state $j$ from the current state
in the random walk $j$ by sampling column-wise. Per
Eq~(\ref{eq:adjoint_neumann_ulam}), the transition weight is then
defined as:
\begin{equation}
  w_{ij} = \frac{h_{ji}}{p_{ij}}\:.
  \label{eq:adjoint_weight}
\end{equation}
Using the decomposition we can then define an expectation value for
the adjoint method. Using our result from
Eq~(\ref{eq:inner_product_constraint}) generated by applying the
adjoint constraints, the contribution to the solution in state $i$
from a particular random walk permutation is then:
\begin{equation}
  X_{\nu} = \sum_{m=0}^k W_{m} \delta_{i,i_m}\:,
  \label{eq:adjoint_permutation_contribution}
\end{equation}
where the Kronecker delta indicates that the tally contributes only in
the current state and $b_{i_0}$ will be the sampled source starting
weight. Note here that the estimator in
Eq~(\ref{eq:adjoint_permutation_contribution}) does not have a
dependency on the source state as in
Eq~(\ref{eq:adjoint_permutation_contribution}), providing a remedy for
the situation in the direct method where we must start a random walk
in each source state for every permutation such that we may compute a
contribution for that state. In the adjoint method, we instead tally
in all states and those of lesser importance will not be visited as
frequently by the random walk. Finally, the expectation value using
all permutations is:
\begin{equation}
  E\{X\} = \sum_{\nu} P_{\nu} X_{\nu}\:
  \label{eq:adjoint_expectation_value}
\end{equation}
which, if expanded in the same way as the direct method, directly
recovers the exact solution:
\begin{equation}
  \begin{split}
    E\{X_j\} &=\sum_{k=0}^{\infty}\sum_{i_1}^{N}\sum_{i_2}^{N}\ldots
    \sum_{i_k}^{N} b_{i_0} h_{i_0,i_1}h_{i_1,i_2}\ldots h_{i_{k-1},i_k}
    \delta_{i_k,j} \\ &= x_{j}\:,
  \end{split}
  \label{eq:adjoint_expectation_expansion}
\end{equation}
therefore also providing an unbiased Monte Carlo estimate of the
solution.

Like the direct method, we also desire a criteria for random walk
termination for problems where only an approximate solution is
necessary. For the adjoint method, we utilize a \textit{relative
  weight cutoff}:
\begin{equation}
  W_f = W_c b_{i_0}\:,
  \label{eq:relative_weight_cutoff}
\end{equation}
where $W_c$ is defined as in the direct method. The adjoint random
walk will then be terminated after $m$ steps if $W_m < W_f$ as tally
contributions become increasingly small.

%%---------------------------------------------------------------------------%%
\section{Domain Decomposition}

%%---------------------------------------------------------------------------%%
\section{Model Problem}
For our numerical experiments, we choose the one-speed,
two-dimensional neutron diffusion equation as the model problem
\cite{duderstadt_nuclear_1976}: 
\begin{equation}
  -\boldsymbol{\nabla} \cdot D \boldsymbol{\nabla} \phi + \Sigma_a
  \phi = S\:,
  \label{eq:diffusion_eq}
\end{equation}
where $\phi$ is the neutron flux, $\Sigma_a$ is the absorption cross
section, $S$ is the source of neutrons. In addition, $D$ is the diffusion
coefficient defined as:
\begin{equation}
  D = \frac{1}{3 ( \Sigma_t - \bar{\mu}\Sigma_s )}\:,
  \label{eq:diffusion_coeff}
\end{equation}
where $\Sigma_s$ is the scattering cross section, $\Sigma_t = \Sigma_a
+ \Sigma_s$ is the total cross section, and $\bar{\mu}$ is the cosine
of the average scattering angle. For simplicity, we will take
$\bar{\mu} = 0$ for our analysis giving $D=(3 \Sigma_t)^{-1}$. In
addition, to further simplify we will assume a homogenous domain such
that the cross sections remain constant throughout. Doing this permits
us to rewrite Eq~(\ref{eq:diffusion_eq}) as:
\begin{equation}
  -D \boldsymbol{\nabla}^2 \phi + \Sigma_a \phi = S\:.
  \label{eq:diffusion_eq_simple}
\end{equation}

We choose a finite difference scheme on a square Cartesian grid to
discretize the problem. For the Laplacian, we choose the 9-point
stencil shown in Figure~\ref{fig:stencil} \cite{leveque_finite_2007}:
\begin{multline}
  \nabla^2_9\phi = \frac{1}{6\Delta^2}[4 \phi_{i-1,j} + 4 \phi_{i+1,j}
    + 4 \phi_{i,j-1} + 4 \phi_{i,j+1} + \phi_{i-1,j-1}\\ +
    \phi_{i-1,j+1} + \phi_{i+1,j-1} + \phi_{i+1,j+1} - 20
    \phi_{i,j}]\:.
  \label{eq:nine_point_stencil}
\end{multline}
\begin{figure}[htpb!]
  \begin{center}
    \scalebox{1.25}{\input{stencil.pdftex_t}}
  \end{center}
  \caption{\textbf{Nine-point Laplacian stencil.}}
  \label{fig:stencil}
\end{figure}
We then have the following linear system to solve:
\begin{multline}
  -\frac{1}{6\Delta^2}[4 \phi_{i-1,j} + 4 \phi_{i+1,j}
    + 4 \phi_{i,j-1} + 4 \phi_{i,j+1} + \phi_{i-1,j-1}\\ +
    \phi_{i-1,j+1} + \phi_{i+1,j-1} + \phi_{i+1,j+1} - 20
    \phi_{i,j}] + \Sigma_a \phi_{i,j} = s_{i,j}\:,
  \label{eq:fd_system}
\end{multline}
and in operator form:
\begin{equation}
  \ve{D}\boldsymbol{\phi}=\ve{S}\:,
  \label{eq:operator_system}
\end{equation}
where $\ve{D}$ is the diffusion operator, $\ve{S}$ is the source in
vector form and $\boldsymbol{\phi}$ is the vector of unkown fluxes. We
note here that the diffusion operator is symmetric.

To close the system a non-reentrant current condition is applied to
all global boundaries of the domain. For our problems, we choose the
formulation of Duderstadt by assuming the flux is zero at some ghost
point beyond the grid. Consider for example the equations on the $i=0$
boundary of the domain:
\begin{multline}
  -\frac{1}{6\Delta^2}[4 \phi_{-1,j} + 4 \phi_{1,j}
    + 4 \phi_{0,j-1} + 4 \phi_{0,j+1} + \phi_{-1,j-1}\\ +
    \phi_{-1,j+1} + \phi_{1,j-1} + \phi_{1,j+1} - 20
    \phi_{0,j}] + \Sigma_a \phi_{0,j} = s_{0,j}\:.
  \label{eq:x_min_bnd}
\end{multline}
Here we note some terms where $i=-1$ and therefore are representative
of grid points beyond the boundary of the domain. We set the flux at
these points to be zero, giving a valid set of equations for the $i=0$
boundary:
\begin{multline}
  -\frac{1}{6\Delta^2}[4 \phi_{1,j} + 4 \phi_{0,j-1} + 4 \phi_{0,j+1}
    \\ + \phi_{-1,j+1} + \phi_{1,j-1} + \phi_{1,j+1} - 20 \phi_{0,j}]
  + \Sigma_a \phi_{0,j} = s_{0,j}\:.
  \label{eq:x_min_bnd}
\end{multline}
We repeat this procedure for the other boundaries of the domain.

%%---------------------------------------------------------------------------%%
\section{Spectral Analysis}

%%---------------------------------------------------------------------------%%
\section{Random Walk Lengths}

%%---------------------------------------------------------------------------%%
\section{Domain Leakage}

%%---------------------------------------------------------------------------%%
\section{Conclusion}

%%---------------------------------------------------------------------------%%
\pagebreak
\bibliographystyle{ieeetr}
\bibliography{references}
\end{document}


