%%---------------------------------------------------------------------------%%
%% direct_vs_adjoint.tex
%% Stuart Slattery
%% Wed May 23 11:25:38 2012
%% Copyright (C) 2008-2010 Oak Ridge National Laboratory, UT-Battelle, LLC.
%%---------------------------------------------------------------------------%%
\documentclass[note]{TechNote}
\usepackage[centertags]{amsmath}
\usepackage{amssymb,amsthm}
\usepackage[mathcal]{euscript}
\usepackage{tabularx}
\usepackage{cite}
\usepackage{c++}
\usepackage{tmadd,tmath}

%%---------------------------------------------------------------------------%%
%% DEFINE SPECIFIC ENVIRONMENTS HERE
%%---------------------------------------------------------------------------%%
%\newcommand{\elfit}{\ensuremath{\operatorname{Im}(-1/\epsilon(\vq,\omega)}}
%\newcommand{\vOmega}{\ensuremath{\ve{\Omega}}}
%\newcommand{\hOmega}{\ensuremath{\hat{\ve{\Omega}}}}

%%---------------------------------------------------------------------------%%
%% BEGIN DOCUMENT
%%---------------------------------------------------------------------------%%
\begin{document}

%%---------------------------------------------------------------------------%%
%% OPTIONS FOR NOTE
%%---------------------------------------------------------------------------%%

\refno{RNSD-01}
\subject{Stochastic Linear Solver Comparison for Monte Carlo Synthetic
Acceleration Method}

%-------HEADING
\TIname{Reactor and Nuclear Systems Division}
\groupname{Radiation Transport Group}
\from{uy7}
\date{\today}
%-------HEADING

\audience{\email{uy7}{uy7@ornl.gov}}

%%---------------------------------------------------------------------------%%
%% BEGIN NOTE
%%---------------------------------------------------------------------------%%

\opening

\begin{abstract}
  The Monte Carlo Synthetic Acceleration (MCSA) method utilizes a
  stochastic linear solver in its solution scheme. Two stochastic
  solvers are available in the form of a direct and an adjoint
  formulation. This work implements the direct and adjoint solvers
  within an MCSA method and compares their speed and convergengence
  properties to determine which, if either, is most suitable for use
  with MCSA. 
\end{abstract}

%%---------------------------------------------------------------------------%%
%% OPTIONAL TOC

% \memotoc

%%---------------------------------------------------------------------------%%
\section{Introduction}

%%---------------------------------------------------------------------------%%
\section{Model Problem}

%%---------------------------------------------------------------------------%%
\section{Timing Study Results}

%%---------------------------------------------------------------------------%%
\section{Convergence Study Results}

%%---------------------------------------------------------------------------%%
\section{Conclusion}

%%---------------------------------------------------------------------------%%
%% BIBLIOGRAPHY STUFF

% \bibliographystyle{notes}
% \bibliography{direct_vs_adjoint}

\closing
\caution
\end{document}

%%---------------------------------------------------------------------------%%
%% end of direct_vs_adjoint.tex
%%---------------------------------------------------------------------------%%
