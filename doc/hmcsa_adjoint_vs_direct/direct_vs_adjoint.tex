%%---------------------------------------------------------------------------%%
%% direct_vs_adjoint.tex
%% Stuart Slattery
%% Wed May 23 11:25:38 2012
%% Copyright (C) 2008-2010 Oak Ridge National Laboratory, UT-Battelle, LLC.
%%---------------------------------------------------------------------------%%
\documentclass[note]{TechNote}
\usepackage[centertags]{amsmath}
\usepackage{amssymb,amsthm}
\usepackage[mathcal]{euscript}
\usepackage{tabularx}
\usepackage{cite}
\usepackage{c++}
\usepackage{tmadd,tmath}

%%---------------------------------------------------------------------------%%
%% DEFINE SPECIFIC ENVIRONMENTS HERE
%%---------------------------------------------------------------------------%%
%\newcommand{\elfit}{\ensuremath{\operatorname{Im}(-1/\epsilon(\vq,\omega)}}
%\newcommand{\vOmega}{\ensuremath{\ve{\Omega}}}
%\newcommand{\hOmega}{\ensuremath{\hat{\ve{\Omega}}}}

%%---------------------------------------------------------------------------%%
%% BEGIN DOCUMENT
%%---------------------------------------------------------------------------%%
\begin{document}

%%---------------------------------------------------------------------------%%
%% OPTIONS FOR NOTE
%%---------------------------------------------------------------------------%%

\refno{RNSD-01}
\subject{Stochastic Linear Solver Comparison for Monte Carlo Synthetic
Acceleration Method}

%-------HEADING
\TIname{Reactor and Nuclear Systems Division}
\groupname{Radiation Transport Group}
\from{Stuart R. Slattery}
\date{\today}
%-------HEADING

\audience{\email{Stuart Slattery}{uy7@ornl.gov},
          \email{Thomas Evans}{tme@ornl.gov},
          \email{Greg Davidson}{gqe@ornl.gov}}

%%---------------------------------------------------------------------------%%
%% BEGIN NOTE
%%---------------------------------------------------------------------------%%

\opening

\begin{abstract}
  The Monte Carlo Synthetic Acceleration (MCSA) method utilizes a
  stochastic linear solver in its solution scheme. Two stochastic
  solvers are available in the form of a direct and an adjoint
  formulation. This work implements the direct and adjoint solvers
  within an MCSA solver and compares their speed and convergengence
  properties to determine which, if either, is most suitable for use
  with MCSA. The studies show that the adjoint solver requires
  significantly less CPU time for convergence than the direct
  method. Therefore, it is recommended that the adjoint solver be used
  as the linear solver within an MCSA solver implementation.
\end{abstract}

%%---------------------------------------------------------------------------%%
%% OPTIONAL TOC

% \memotoc

%%---------------------------------------------------------------------------%%
\section{Introduction}
The Monte Carlo Synthetic Acceleration (MCSA) method developed by
Evans et. al. \cite{Evans_2009} provides an acceleration to standard
stochastic linear solver methods that are bound by the central limit
theorem, instead giving exponential convergence. To do this, we must
first consider stochastic solutions for linear systems. For the
following linear system

\begin{equation}
  A x = b
  \label{eq:linear_system}
\end{equation}

we can rewrite using an iteration matrix formulation as follows.

\begin{equation}
  x = (I-A) x + b
  \label{eq:iteration_form}
\end{equation}

We will define $H=I-A$ to be the iteration matrix. Using this
definition we can then rewrite the original linear system using the
iteration matrix.

\begin{equation}
  x = (I-H)^{-1} b
  \label{eq:iteration_form2}
\end{equation}

If the spectral radius of the iteration matrix is less than unity,
then we can rewrite the inverted operator as a Von Neumann series.

\begin{equation}
  (I-H)^{-1} = \sum_k H^k
  \label{eq:von_neumann_series}
\end{equation}

We then expect the following series solution for x to converge if the
spectral radius of the iteration matrix is less than unity.

\begin{equation}
  x = b + H b + H^2 b + H^3 b + ...
  \label{eq:series_solution}
\end{equation}

Using this form, we can approximate the terms in the series with a
random walk sequence as outlined in \cite{Hammersley_1964}. There are
two primary ways of achieving this; through a direct sequence and an
adjoint sequence. For the direct sequence, analagous (and
inconveniently so) to an adjoint stochastic transport calculation,
each term in the source is simulated with an equivalent number of
random walk histories. Each step in the random walk sequence will
contribute to the solution in the starting source state. For the
adjoint sequence, analagous to a forward stochastic transport
calculation, the source term is sampled with a set number of random
walk histories, meaning that regions of larger source will be sampled
as a starting state more regularly. In this case, each step in the
random walk sequence will contribute to the state in which it
currently resides, not the current source state. It is this set of
primary differences between the two solution methods that will dictate
solver performance with MCSA.

With this knowledge, for the linear system in
Eq. \ref{eq:linear_system} we can then define the following MCSA
solution sequence

\begin{equation}
  x^{k+1/2} = (I-A)x^k + b
  \label{eq:mcsa_step1}
\end{equation}
\begin{equation}
  r = b - A x^{k+1/2}
  \label{eq:mcsa_step2}
\end{equation}
\begin{equation}
  A \delta x = r
  \label{eq:mcsa_step3}
\end{equation}
\begin{equation}
  x^{k+1} = x^{k+1/2} + \delta x
  \label{eq:mcsa_step4}
\end{equation}

where k is the iteration index. In brief, in Eq. \ref{eq:mcsa_step1}
we compute an initial relaxation for $x$ with a Richardson iteration,
in Eq. \ref{eq:mcsa_step2} we compute the residual, in
Eq. \ref{eq:mcsa_step3} we compute the solution update via a
stochastic linear solve and finally we update the solution in
Eq. \ref{eq:mcsa_step4}. Of interest to this work is the stochastic
linear solution needed to compute the solution update in Eq.
\ref{eq:mcsa_step3} which we will choose to use either the direct or
adjoint method. It is important to note that the stochastic solver is
not computing $x$, rather it is computing an error term using the
residual that we will apply to $x$.

%%---------------------------------------------------------------------------%%
\section{Model Problem}
To study the behaviour of the MCSA method with both direct and adjoint
solutions, we choose the two-dimensional time-dependent heat equation
as a simple model problem.

\begin{equation}
  \frac{\partial u}{\partial t} = \alpha \nabla^2 u
  \label{eq:heat_equation}
\end{equation}

where $\alpha$ is a constant coefficient. For all comparisons, a
single time step is computed with implicit Euler time integration. The
Laplacian is differenced with a second-order five point stencil

\begin{equation}
  \nabla^2_5 = \frac{1}{h^2}[u_{i-1,j} + u_{i+1,j} + u_{i,j-1} +
  u_{i,j+1} - 4 u_{i,j}]
  \label{eq:five_point_stencil}
\end{equation}

and a fourth-order nine point stencil both assuming a grid size of $h$
in both the $i$ and $j$ directions.

\begin{equation}
  \nabla^2_9 = \frac{1}{6h^2}[4 u_{i-1,j} + 4 u_{i+1,j} + 4 u_{i,j-1}
    + 4 u_{i,j+1} + u_{i-1,j-1} + u_{i-1,j+1} + u_{i+1,j-1} +
    u_{i+1,j+1} - 20 u_{i,j}]
  \label{eq:nine_point_stencil}
\end{equation}

For a single time step solution, we then have the following linear
system to be solved with the MCSA method.

\begin{equation}
  A u^{n+1} = u^n
  \label{eq:heat_eq_lin_sys}
\end{equation}

Both the stencils will be used to vary the size and density of the
sparse linear system in Eq. \ref{eq:heat_eq_lin_sys}.

%%---------------------------------------------------------------------------%%
\section{Timing Study Results}
To assess both the CPU time and number of iterations required to
converge to a solution, a problem of constant grid size and $\alpha$
was used with varying values of problem size, fixing the spectral
radius of the system at a constant value for each variation. Both the
five point and nine point stencils were used with both the direct and
adjoint solvers. For each case, 50 histories were simulated per cycle
with a weight cutoff of $1.0E-4$ for the stochastic solver and a
convergence tolerance of $1.0E-8$ for the MCSA solver. All
computations were completed on a 2.66 GHz Intel Core i7 machine with 8
GB 1067 MHz DDR3 memory.

Fig. \ref{fig:cpu_time} gives the CPU time needed for each case to
converge in seconds and Fig. \ref{fig:iterations} gives the number of
iterations needed for each case to converge.



%%---------------------------------------------------------------------------%%
\section{Convergence Study Results}

%%---------------------------------------------------------------------------%%
\section{Conclusion}

%%---------------------------------------------------------------------------%%
%% BIBLIOGRAPHY STUFF

% \bibliographystyle{notes}
% \bibliography{direct_vs_adjoint}

\closing
\caution
\end{document}

%%---------------------------------------------------------------------------%%
%% end of direct_vs_adjoint.tex
%%---------------------------------------------------------------------------%%
