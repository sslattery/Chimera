%        File: thesis_proposal.tex
%
\documentclass[letterpaper,12pt]{article}
\usepackage[top=1.0in,bottom=1.0in,left=1.25in,right=1.25in]{geometry}
\usepackage{verbatim}
\usepackage{amssymb}
\usepackage{graphicx}
\usepackage{longtable}
\usepackage{amsfonts}
\usepackage{amsmath}
\usepackage[usenames]{color}
\usepackage[
naturalnames = true, 
colorlinks = true, 
linkcolor = black,
anchorcolor = black,
citecolor = black,
menucolor = black,
urlcolor = blue
]{hyperref}

%%---------------------------------------------------------------------------%%
\author{Stuart R. Slattery
\\ \href{mailto:sslattery@wisc.edu}{\texttt{sslattery@wisc.edu}}
}

\date{\today}
\title{Thesis Proposal:\\
  A Massively Parallel Stochastic Method for Nonlinear Problems}
\begin{document}
\maketitle

%%---------------------------------------------------------------------------%%
\section{Introduction}
In many fields of engineering and physics, nonlinear problems are a
primary focus of study. Recent focus on multiple physics systems (and
in particular nuclear systems) adds a new level of complexity to
common nonlinear systems as solution strategies change when they are
coupled to other problems. Furthermore, a desire for predictive
simulations to enhance the safety and performance of engineered
systems creates a need for extremely high fidelity computations to be
performed for these coupled systems as a means to capture effects not
modeled by coarser methods. In order to achieve this high fidelity,
state-of-the-art computing facilities must be leveraged in a way that
is both efficient and considerate of hardware-related issues. As
scientific computing moves towards exascale facilities with machines
of $O(1,000,000)$ cores already coming online, new algorithms to solve
these complex problems must be developed to leverage this new
hardware. Issues such as resiliency to node failure and scaling to
large numbers of cores will be pertinent to robust algorithms aimed at
this new hardware. Considering these issues, this thesis proposes a
novel stochastic method to advance solution techniques for nonlinear
problems with a focus on coupled physics systems.

For some time, the particle transport community has been utilizing
Monte Carlo methods for the solution of transport problems
\cite{Lewis_1993}. The partial differential equation (PDE) community
has focused on various deterministic methods for solutions to linear
problems \cite{Saad_2003}. In between these two areas are a not widely
known small group of stochastic methods for solving sparse linear
systems \cite{Hammersley_1964, Halton_1962, Halton_1994}. In recent
years, these methods have been further developed for transport
problems in the form of Monte Carlo Synthetic-Acceleration (MCSA)
\cite{Evans_2003, Evans_2009} but have yet to be applied to more
general sparse linear systems. Compared to other methods in these
regime, MCSA offers three attractive qualities; (1) the linear problem
operator need not be symmetric or positive-definite, thereby reducing
preconditioning complexity, (2) the stochastic nature of the
solution method provides a natural solution to the issue of resiliency,
and (3) is amenable to parallelization using modern methods developed by
the transport community \cite{Wagner_2011}. The developement of MCSA
as a general linear solver and the development of a parallel MCSA
method will be new and unique features of this work. Resiliency will
not be addressed.

In addition to linear solver advancements, nonlinear solvers may also
benefit from a general and parallel MCSA scheme. In the engineering
community, nonlinear problems are often addressed by linearizing the
problem and using traditionaly iterative or direct methods
\cite{Tannehill_1997}. In the mathematics community, various Newton
methods have been popular \cite{Kelley_1995}. Recently, Jacobian-Free
Newton-Krylov (JFNK) schemes \cite{Knoll_2004} have been utilized in
multiple physics architectures and advanced single physics codes
\cite{Gaston_2009}. The benefits of JFNK schemes are that the Jacobian
is never formed, simplifying the implementation, and a Krylov solver
is leveraged (typically GMRES or Conjugate Gradient), providing
excellent convergence properties for well-conditioned and well-scaled
systems. However, there are two potential drawbacks to these methods
for high fidelity predictive simulations: (1) the Jacobian is
approximated by a first-order differencing method on the order of
machine precision such this error can grow beyond that of those in a
fine-grained system and (2) for systems that are not symmetric
positive-definite (which will be the case for most multiphysics
systems and certainly for most preconditioned systems) the Krylov
subspace generated by the GMRES solver may become prohibitively
large. To address these issues, this thesis proposes a new and novel
method for nonlinear systems based on the MCSA method.

The Forward-Automated Newton-MCSA (FANM) method is proposed as new
nonlinear solution method. The key features of FANM are: full Jacobian
generation using modern Forward Automated Differencing (FAD) methods
\cite{Bartlett_2006}, and MCSA as the inner linear solver. This method
has several attractive properties. First, the first-order
approximation to the Jacobian used in JFNK type methods is eliminated
by generating the Jacobian explicitly with the model equations through
FAD. Second, the Jacobian need not be explicitly formed by the user
but is instead automated through FAD; this eleminates the complexity
of hand-coding derivatives and has also been demonstrated to be more
efficient computationally than evaluating differenced derivatives
\cite{Bartlett_2006}. Third, unlike GMRES, MCSA does not build a
subspace during iterations. Although the Jacobian must be explicitly
formed to use MCSA, for problems that take more than a few GMRES
iterations to converge the size of the Krylov subspace will grow
beyond that of the Jacobian. Finally, using MCSA for the linear solve
provides its benefits for preconditioning, potential resiliency, and
parallelism.

%%---------------------------------------------------------------------------%%
\section{Deliverables}

This work will deliver the following items:

\begin{itemize}
\item Develop a massively parallel strategy for the MCSA method for
  sparse linear systems.
\item Implement a massively parallel strategy for the MCSA method for
  sparse linear systems.
\item Fully assess the performance of parallel MCSA on leadership class
  computing systems in order to determine its feasibility using
  benchmark problems.
\item Compare the performance of MCSA to other iterative methods for
  sparse linear systems including GMRES.
\item Develop the FANM method for nonlinear systems.
\item Implement the FANM method for nonlinear systems.
\item Fully assess the performance of the FANM method on leadership
  class computing systems using model problems.
\item Compare the performance of MCSA to other nonlinear methods
  including JFNK.
\end{itemize}

This work will not address the resiliency of these algorithms.

%%---------------------------------------------------------------------------%%
\section{Dissertation Outline}

Based on the above work, the dissertation can be outlined in five
distinct parts:

\begin{enumerate}
\item Iterative methods for sparse linear systems (fixed-point,
  subspace, and stochastic methods).
\item MCSA and Multiple Set Overlapping Domain Decomposition
  (development and implementation).
\item Iterative methods for nonlinear systems (Newton methods).
\item FANM (developement and implementation).
\item Application to model problems (conduction, Navier-Stokes,
  conjugate heat transfer).
\end{enumerate}

%%---------------------------------------------------------------------------%%
\section{Committee}

\begin{itemize}
\item Paul Wilson (advisor): Associate Professor, Department of
  Engineering Physics.
\item Greg Moses: Professor, Department of Engineering Physics.
\item Carl Sovinec: Profssor, Department of Engineering Physics.
\item Chris Rutland: Professor, Deptartment of Mechanical Engineering.
\item Tom Evans: Distinguished R\&D Staff, Reactor and Nuclear Systems
  Division, Oak Ridge National Laboratory.

\end{itemize}

%%---------------------------------------------------------------------------%%
\section{Timeline}

%%---------------------------------------------------------------------------%%
\pagebreak
\bibliographystyle{ieeetr}
\bibliography{references}
\end{document}


