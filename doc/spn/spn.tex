%        File: spn.tex
\documentclass[letterpaper]{article}
\usepackage[top=1.0in,bottom=1.0in,left=1.0in,right=1.0in]{geometry}
\usepackage{verbatim}
\usepackage{amssymb}
\usepackage{graphicx}
\usepackage{longtable}
\usepackage{amsfonts}
\usepackage{amsmath}
\usepackage[usenames]{color}
\usepackage[
naturalnames = true, 
colorlinks = true, 
linkcolor = black,
anchorcolor = black,
citecolor = black,
menucolor = black,
urlcolor = blue
]{hyperref}
\usepackage{listings}
\usepackage{textcomp}
\definecolor{listinggray}{gray}{0.9}
\definecolor{lbcolor}{rgb}{0.9,0.9,0.9}
\lstset{
  backgroundcolor=\color{lbcolor},
  tabsize=4,
  rulecolor=,
  language=c++,
  basicstyle=\scriptsize,
  upquote=true,
  aboveskip={1.5\baselineskip},
  columns=fixed,
  showstringspaces=false,
  extendedchars=true,
  breaklines=true,
  prebreak =
  \raisebox{0ex}[0ex][0ex]{\ensuremath{\hookleftarrow}},
  frame=single,
  showtabs=false,
  showspaces=false,
  showstringspaces=false,
  identifierstyle=\ttfamily,
  keywordstyle=\color[rgb]{0,0,1},
  commentstyle=\color[rgb]{0.133,0.545,0.133},
  stringstyle=\color[rgb]{0.627,0.126,0.941},
}

\author{Stuart R. Slattery
  \\ \href{mailto:sslattery@wisc.edu}{\texttt{sslattery@wisc.edu}}
}

\date{\today}
\title{Monte Carlo Solutions for the $SP_N$ Equations}
\begin{document}
\maketitle

%%---------------------------------------------------------------------------%%
\section{Introduction}
\label{sec:introduction}

%%---------------------------------------------------------------------------%%
\section{Derivation of the $SP_N$ Equations}
\label{sec:derivation}
In this section, we derive the $SP_N$ equations. We begin by stating
the general time-independent neutron transport equation followed by a
derivation of the $P_N$ equations in planar geometry for multiple
energy groups. From these equations, we then apply a set of
approximations to yield the multi-dimensional, multi-group $SP_N$
equations for fixed source problems.

\subsection{The Neutron Transport Equation}
\label{subsec:transport_eq}
As a starting point we define the time-independent neutron transport
equation \cite{lewis_computational_1993}:
\begin{equation}
  \hat{\Omega} \cdot \vec{\nabla} \psi(\vec{r},\hat{\Omega},E) +
  \sigma(\vec{r},E) \psi(\vec{r},\hat{\Omega},E) = \int dE' \int
  d\Omega' \sigma_s(\vec{r},E' \rightarrow E,\hat{\Omega}' \rightarrow
  \hat{\Omega}) \psi(\vec{r},\hat{\Omega}',E') +
  q(\vec{r},\hat{\Omega},E)\:,
  \label{eq:general_transport}
\end{equation}
with the variables defined as:
\begin{itemize}
\item $\vec{r}$ - neutron spatial position
\item $\hat{\Omega}$ - neutron streaming direction with radial
  component $\mu$ and azimuthal component $\omega$
\item $\hat{\Omega}' \cdot \hat{\Omega} = \mu_0$ is the angle of
  scattering
\item $E$ - neutron energy
\item $\psi(\vec{r},\hat{\Omega},E)$ - angular flux
\item $\sigma(\vec{r},E)$ - total interaction cross section
\item $\sigma_s(\vec{r},E' \rightarrow E,\hat{\Omega}'$ - probability
  of scattering from direction $\hat{\Omega}'$ into an angular domain
  $d\hat{\Omega}'$ about the direction $\hat{\Omega}$ and from energy
  $E'$ to an energy domain $dE'$ about energy $E$
\item $q(\vec{r},\hat{\Omega},E)$ - external source of neutrons.
\end{itemize}
For this work, it is sufficient to formulate
Eq~(\ref{eq:general_transport}) in 1-dimensonal Cartesian geometry:
\begin{equation}
  \mu \frac{\partial}{\partial x} \psi(x,\mu,E) + \sigma(x,E)
  \psi(x,\mu,E) = \int dE' \int d\Omega' \sigma_s(x,E' \rightarrow
  E,\hat{\Omega}' \rightarrow \hat{\Omega})
  \psi(\vec{r},\hat{\Omega}',E') + \frac{q(x,E)}{4 \pi}\:,
  \label{eq:cart_1d_transport}
\end{equation}
where the angular component of the solution is no longer dependent on
the azimuthal direction of travel and an isotropic source of neutrons
is assumed.

\subsection{Derivation of the $P_N$ Equations}
\label{subsec:pn_equations}
Next, we derive the $P_N$ equations, a simplified form of the general
transport equation where Legendre polynomials are used to expand the
angular flux and scattering cross section variables as a means of
capturing the angular structure of the solution. Before deriving this
form of the transport equation, we briefly discuss a few properties of
Legendre polynomials that we will find useful in the derivation.

\subsubsection{Legendre Polynomials}
\label{subsubsec:legendre_polys}
The Legendre polynomials are an orthogonal set of functions that
are solutions to Legendre's differential equation. They have the
following form \cite{lewis_computational_1993}:
\begin{equation}
  P_l(\mu) = \frac{1}{2^l l!}\frac{d^l}{d \mu^l}(\mu^2-1)^l\:.
  \label{eq:general_legendre_poly}
\end{equation}
These functions have several nice properties including orthogonality:
\begin{equation}
  \int_{-1}^{1} P_l(\mu) P_{l'}(\mu) d\mu = \frac{1}{2l+1}\delta_{l l'}\:,
  \label{eq:legendre_orthog}
\end{equation}
a recurrence relation:
\begin{equation}
  \mu P_l(\mu) = \frac{1}{2l+1}[(l+1)P_{l+1}(\mu) + l P_{l-1}(\mu)]\:,
  \label{eq:legendre_recurrence}
\end{equation}
and an addition theorem:
\begin{equation}
  P_l(\hat{\Omega} \cdot \hat{\Omega}') = \frac{1}{2n+1}\sum_{m=-l}^l
  Y_{lm}(\hat{\Omega})Y^*_{lm}(\hat{\Omega}')\:,
  \label{eq:legendre_addition}
\end{equation}
where the functions $Y_{lm}(\hat{\Omega})$ are the spherical
harmonics. We can form the addition theorem in this way because the
spherical harmonics are in fact just harmonic multiples of the
Legendre polynomials:
\begin{equation}
  Y_{lm}(\hat{\Omega}) =
  \sqrt{\frac{(2l+1)(l-m)!}{(l+m)!}}P^m_l(\mu)e^{i \omega m}\:,
  \label{eq:spherical_harmonic}
\end{equation}
where $\omega$ is the azimuthal component of the streaming
direction. We can reduce Eq~(\ref{eq:legendre_addition}) for the
planar geometry we are studying by ignoring the azimuthal components
of the addition theorem. As shown in Eq~(\ref{eq:spherical_harmonic}),
the azimuthal dependence is given by the harmonic component, $e^{i
  \omega m}$, and therefore we choose to ignore all terms in
Eq.~(\ref{eq:legendre_addition}) where $m \neq 0$. This gives:
\begin{equation}
  P_l(\hat{\Omega} \cdot \hat{\Omega}') = \frac{1}{2n+1}
  Y_{l0}(\hat{\Omega})Y^*_{l0}(\hat{\Omega}')\:.
  \label{eq:legendre_addition_2}
\end{equation}
Per Eq~(\ref{eq:spherical_harmonic}) we have:
\begin{equation}
  Y_{l0}(\hat{\Omega}) = \sqrt{2l+1}P_l^0(\mu)\:,
  \label{eq:harmonic_0}
\end{equation}
where $P_l^0(\mu) = P_l(\mu)$ is the $0^{th}$ associated Legendre
function. Finally, we can reduce the addition theorem from
Eq~(\ref{eq:legendre_addition_2}) with Eq~(\ref{eq:harmonic_0}) to a
simple product for planar geometry:
\begin{equation}
  P_l(\hat{\Omega} \cdot \hat{\Omega}') = P_l(\mu)P_l(\mu')\:.
  \label{eq:legendre_addition_3}
\end{equation}

\subsubsection{Planar $P_N$ Equations}
\label{subsubsec:planar_pn}
With the Legendre polynomial properties defined above, we can proceed
by deriving the $P_N$ equations for planar geometry. For these
equations, we will start by assuming a monoenergtic field of neutrons
such that we are solving the following reduced form of the transport
equation: 
\begin{equation}
  \mu \frac{\partial}{\partial x} \psi(x,\mu) + \sigma(x) \psi(x,\mu)
  = \int d\Omega' \sigma_s(x,\hat{\Omega}' \rightarrow \hat{\Omega})
  \psi(\vec{r},\hat{\Omega}') + \frac{q(x)}{4 \pi}\:.
  \label{eq:mono_transport}
\end{equation}
The $P_N$ equations introduce the approximation that the angular
dependence of the scattering cross section, $sigma_s$, and the angular
flux $\psi$, can be discretized by expanding them in Legendre
polynomials as follows:
\begin{equation}
  \psi(x,\mu) = \sum_{n=0}^\infty (2n+1)P_n(\mu)\phi_n(x)\:,
  \label{eq:flux_expansion}
\end{equation}
\begin{equation}
  \sigma_{sm}(x) = \sum_{m=0}^\infty (2m+1)P_m(\mu)\sigma_s(x)\:,
  \label{eq:scattering_expansion}
\end{equation}
where we have supressed the $2\pi$ generated by integrating away the
azimuthal angular component and $\phi_n(x)$ in
Eq~(\ref{eq:flux_expansion}) is referred to as the $n^{th}$ Legendre
moment of the neutron flux and is given by:
\begin{equation}
  \phi_n(x) = \int_{-1}^1 P_n(\mu)\psi(x,\mu)d\mu\:.
  \label{eq:legendre_moments}
\end{equation}
We first insert the expansions given by Eq~(\ref{eq:flux_expansion})
and (\ref{eq:scattering_expansion}) into the planar transport equation
given by Eq~(\ref{eq:mono_transport}):
\begin{equation}
  \frac{\partial}{\partial x}\Big[\sum_{n=0}^\infty (2n+1) \phi_n \mu
    P_n(\mu) \Big] + \sigma \sum_{n=0}^\infty (2n+1) \phi_n P_n(\mu) =
  \int_{-1}^1 \sum_{m=0}^\infty (2m+1) \sigma_{sm} P_m(\mu_0)
  \sum_{n=0}^\infty (2n+1) \phi_n P_n(\mu') d \mu' + q\:,
  \label{eq:pn_deriv_1}
\end{equation}
where the dependence on the spatial variable, $x$, has been
supressed. To arrive at the $P_N$ equations we multiply
Eq~(\ref{eq:pn_deriv_1}) by $P_m(\mu)$ and integrate over the angular
domain $\int_{-1}^1 d \mu$. We will look at each term in
Eq~(\ref{eq:pn_deriv_1}) individually.

\paragraph{Streaming Term}
We first apply the multiplication and integration as prescribed above:
\begin{equation}
  \frac{\partial}{\partial x}\Bigg[\sum_{n=0}^\infty (2n+1) \mu \phi_n
    P_n(\mu) \Bigg] \rightarrow \int_{-1}^1 \frac{\partial}{\partial
    x}\Bigg[\sum_{n=0}^\infty (2n+1) \phi_n \mu P_n(\mu) P_m(\mu) \Bigg]
  d \mu\:.
  \label{eq:pn_deriv_2}
\end{equation}
The $\mu P_n(\mu)$ term can be eliminated via the recurrence relation
given by Eq~(\ref{eq:legendre_recurrence}):
\begin{equation}
\int_{-1}^1 \frac{\partial}{\partial x}\Bigg[\sum_{n=0}^\infty (2n+1)
  \frac{\phi_n}{2n+1}[(n+1)P_{n+1}(\mu) + n P_{n-1}(\mu)] P_m(\mu)
  \Bigg] d\mu\:,
\label{eq:pn_deriv_3}
\end{equation}
which expands to:
\begin{equation}
  \sum_{n=0}^\infty \frac{\partial}{\partial
    x}\phi_n\Bigg[(n+1)\int_{-1}^1 P_{n+1}(\mu)P_m(\mu) d\mu + n
    \int_{-1}^1 P_{n-1}(\mu) P_m(\mu) d\mu \Bigg] \:.
  \label{eq:pn_deriv_4}
\end{equation}
This reveals the orthogonality relation given by
Eq~(\ref{eq:legendre_orthog}) that when inserted into
Eq~(\ref{eq:pn_deriv_3}):
\begin{equation}
  \sum_{n=0}^\infty \frac{\partial}{\partial
    x}\phi_n\Bigg[(n+1)\frac{1}{2n+1}\delta_{n,n+1} +
    n\frac{1}{2n+1}\delta_{n,n-1} \Bigg] \:.
  \label{eq:pn_deriv_5}
\end{equation}
We can then distribute the Legendre moment to arrive at the final form
of the streaming term:
\begin{equation}
  \sum_{n=0}^\infty \frac{\partial}{\partial x} \frac{1}{2n+1} \Big[
    (n+1) \phi_{n+1} + n \phi_{n-1} \Big] \:.
  \label{eq:pn_deriv_6}
\end{equation}

\paragraph{Removal Term}
To reduce the removal term, the orthogonality relation is again
applied after the integration:
\begin{equation}
  \sigma \sum_{n=0}^\infty (2n+1) \phi_n P_n(\mu) \rightarrow
  \int_{-1}^1 \sigma \sum_{n=0}^\infty (2n+1) \phi_n P_n(\mu) P_m(\mu)
  d\mu\:,
  \label{eq:pn_deriv_7}
\end{equation}
\begin{equation}
  \sigma \sum_{n=0}^\infty (2n+1) \phi_n \frac{1}{2n+1}\:,
  \label{eq:pn_deriv_8}
\end{equation}
giving for the final removal term:
\begin{equation}
  \sum_{n=0}^\infty \sigma \phi_n \:.
  \label{eq:pn_deriv_9}
\end{equation}

\paragraph{Scattering Term}
For the scattering term:
\begin{multline}
  \int_{-1}^1 \sum_{m=0}^\infty (2m+1) \sigma_{sm} P_m(\mu_0)
  \sum_{n=0}^\infty (2n+1) \phi_n P_n(\mu') d\mu' \rightarrow\\
  \int_{-1}^1 \int_{-1}^1 \sum_{m=0}^\infty (2m+1) \sigma_{sm} P_m(\mu)
  P_m(\mu_0) \sum_{n=0}^\infty (2n+1) \phi_n P_n(\mu') d\mu' d\mu\:.
  \label{eq:pn_deriv_10}
\end{multline}
The addition thereom from Eq~(\ref{eq:legendre_addition_3}) is applied
to give:
\begin{equation}
  \int_{-1}^1 \int_{-1}^1 \sum_{m=0}^\infty (2m+1) \sigma_{sm} P_m(\mu)
  P_m(\mu)P_m(\mu') \sum_{n=0}^\infty (2n+1) \phi_n P_n(\mu') d\mu' d\mu\:,
  \label{eq:pn_deriv_11}
\end{equation}
which can be rearranged as:
\begin{equation}
  \sum_{m=0}^\infty \sum_{n=0}^\infty (2m+1) \sigma_{sm} (2n+1) \phi_n
  \int_{-1}^1 P_m(\mu') P_n(\mu') d\mu' \int_{-1}^1 P_m(\mu) P_m(\mu)
  d\mu\:.
  \label{eq:pn_deriv_12}
\end{equation}
Again we can apply orthogonality to eliminate the Legendre polynomials:
\begin{equation}
  \sum_{m=0}^\infty \sum_{n=0}^\infty (2m+1) \sigma_{sm} (2n+1) \phi_n
  \frac{1}{2n+1}\delta_{nm}\frac{1}{2m+1}\delta_{mm}\:,
  \label{eq:pn_deriv_13}
\end{equation}
which is reduced to:
\begin{equation}
  \sum_{n=0}^\infty \sigma_{sn} \phi_n\:,
  \label{eq:pn_deriv_14}
\end{equation}
with the dependence on the index $m$ eliminated.

\paragraph{Source Term}
The last term we are concerned with in Eq~(\ref{eq:pn_deriv_1}) is the
source term:
\begin{equation}
  q \rightarrow \int_{-1}^1 q P_n(\mu) d\mu \:.
  \label{eq:pn_deriv_15}
\end{equation}
We can leverage orthogonality by multiplying by $P_0(\mu) = 1$:
\begin{equation}
  \int_{-1}^1 q P_0 P_n(\mu) d\mu = \frac{q}{2*0+1}\delta_{n0}\:,
  \label{eq:pn_deriv_16}
\end{equation}
giving a final source term of:
\begin{equation}
  q\delta_{n0}\:.
  \label{eq:pn_deriv_17}
\end{equation}

Now that we have expanded all angular dependent terms in
Eq~(\ref{eq:pn_deriv_1}) and reduced them appropriately, we can
combine them to generate the $P_N$ equations:
\begin{equation}
    \sum_{n=0}^\infty \frac{\partial}{\partial x} \frac{1}{2n+1} \Big[
      (n+1) \phi_{n+1} + n \phi_{n-1} \Big] + \sum_{n=0}^\infty \sigma
    \phi_n = \sum_{n=0}^\infty \sigma_{sn} \phi_n + q\delta_{n0}\:.
  \label{eq:pn_deriv_18}
\end{equation}
More formally, the $P_N$ equations are written as:
\begin{equation}
   \frac{1}{2n+1} \frac{\partial}{\partial x}\Big[ (n+1) \phi_{n+1} + n
     \phi_{n-1} \Big] + \Sigma_n \phi_n = q\delta_{n0} \:,
  \label{eq:final_pn_equations}
\end{equation}
where $\Sigma_n = \sigma-\sigma_{sn}$ and the summations are truncated
at some level of approximation $N$ such that $n = 0,1,\dotsc,N$. This
yields a set of $N+1$ equations for $N+2$ flux moments. We therefore
require an additional equation to close the system. In accordance with
the series truncation as an approximation we choose the last moment in
the expansion to be zero:
\begin{equation}
  \phi_{N+1} = 0\:.
  \label{eq:pn_closure}
\end{equation}
As an example, we will construct the P5 equations from
Eq~(\ref{eq:final_pn_equations}) and the closure given by
Eq~(\ref{eq:pn_closure}):
\begin{subequations}
  \begin{gather}
   \frac{\partial}{\partial x}\phi_{1} + \Sigma_0 \phi_0 = q\:,\\ 
   \frac{1}{3} \frac{\partial}{\partial x}\Big[ 2
     \phi_{2} + \phi_{0} \Big] + \Sigma_1 \phi_1 = 0\:,\\
   \frac{1}{5} \frac{\partial}{\partial x}\Big[ 3 \phi_{3} + 2
     \phi_{1} \Big] + \Sigma_2 \phi_2 = 0 \:,\\
   \frac{1}{7} \frac{\partial}{\partial x}\Big[ 4 \phi_{4} + 3
     \phi_{2} \Big] + \Sigma_3 \phi_3 = 0 \:,\\
   \frac{1}{9} \frac{\partial}{\partial x}\Big[ 5 \phi_{5} + 4
     \phi_{3} \Big] + \Sigma_4 \phi_4 = 0 \:,\\
   \frac{1}{11} \frac{\partial}{\partial x} 5 \phi_{4} + \Sigma_5
   \phi_5 = 0 \:.
  \end{gather}
  \label{eq:p5_equations}
\end{subequations}
This gives us a set of 6 coupled equations for the six Legendre
moments requested defined over the entire spatial domain for a single
energy group. In practice, only odd-numbered $P_N$ orders are
generally used \cite{lewis_computational_1993}. This is due to the
fact that using odd $N$ yields an even number of $N+1$ equations which
can be split evenly on the left and right boundaries of the problem to
facilitate the description of boundary conditions. We will choose this
convention when deriving the boundary conditions.

\subsubsection{Boundary Conditions for the $P_N$ Equations}
\label{subsbusec:bcs_pn}
Per the analysis in \cite{lewis_computational_1993}, two types of
boundary conditions will be discussed: reflecting and Marshak. Marshak
boundary conditions can be used to specify both vacuum conditions and
isotropic source conditions on the boundary.

\paragraph{Reflecting Boundary Conditions}
In this case, the incoming flux should be equivalent to the outgoing
flux at the boundary point $x_b$:
\begin{equation}
  \psi(x_b,\mu) = \psi(x_b,-\mu)\:.
  \label{eq:reflecting_condition}
\end{equation}
Given the legendre expansion for the flux defined in
Eq~(\ref{eq:flux_expansion}) and the Legendre polynomial property that
$P_n(\mu) = (-1)^n P_n(-\mu)$, the condition specified by
Eq~(\ref{eq:reflecting_condition}) can be satisfied if
\begin{equation}
  \phi_n = 0,\ \forall \ \text{odd}\ n\:,
  \label{eq:reflecting_condition_odd}
\end{equation}
as all even $n$ yield $P_n(\mu) = P_n(-\mu)$ and therefore an
equivalent reflecting condition for the flux moments.

\paragraph{Marshak Boundary Conditions}
The Marshak conditions come directly from the Legendre moments of the
flux:
\begin{equation}
  \int_{\mu_b} P_i(\mu) \psi(\mu) d\mu = \int_{\mu_b} P_i(\mu)
  \psi_b(\mu) d\mu\ \ \ \ \ \ \ \ \ \ \text{for $i=1,3,...,N$}\:,
  \label{eq:general_marshak}
\end{equation}
where $\psi_b(\mu)$ is the prescribed angular flux on the boundary of
interest and $\mu_b$ the angular domain defined by the boundary. To
discretize this condition, we again insert the angular flux expansions
from Eq~(\ref{eq:flux_expansion}) for the fluxes defined in the
domain:
\begin{equation}
  \int_{\mu_b} P_i(\mu) \sum_{n=0}^N (2n+1) \phi_n P_n(\mu) d\mu =
  \int_{\mu_b} P_i(\mu) \psi_b(\mu) d\mu\:.
  \label{eq:marshak_expanded}
\end{equation}
The boundary flux, $\phi_b(\mu)$ is assumed to known and therefore
Eq~(\ref{eq:marshak_expanded}) defines a set of $(N+1)/2$ equations to
be solved on each boundary in the planar case, closing the system in
the spatial domain.

As an example of applying the Marshak conditions, consider the $P_3$
case with an isotropic boundary source $\phi_b$ on the left side of
the domain. In this case, the angular domain over which the boundary
flux is defined will be $\mu_b \in [0,1]$, giving the bounds of
integration. We first expand the summation for $i=1$:
\begin{equation}
  \int_0^1 \mu \Bigg[ \phi_0 + 3\phi_1\mu +
    \frac{5}{2}\phi_2(3\mu^2-1) + \frac{7}{2}\phi_3(5\mu^3-3\mu)
    \Bigg] d\mu = \int_0^1 \mu \phi_b d\mu\:,
  \label{eq:marshak_p1_deriv_1}
\end{equation}
and then $i=3$:
\begin{equation}
  \int_0^1 \frac{1}{2}(5\mu^3-3\mu) \Bigg[ \phi_0 + 3\phi_1\mu +
    \frac{5}{2}\phi_2(3\mu^2-1) + \frac{7}{2}\phi_3(5\mu^3-3\mu)
    \Bigg] d\mu = \int_0^1 \frac{1}{2}(5\mu^3-3\mu) \phi_b d\mu\:.
  \label{eq:marshak_p3_deriv_1}
\end{equation}
Expanding the polynomials in $\mu$ and carrying out the simple
integration then gives 2 equations for the left hand boundary:
\begin{equation}
  \phi_0 + 2\phi_1 + \frac{5}{4}\phi_2 = \phi_b\:,
  \label{eq:marshak_p1_deriv_2}
\end{equation}
\begin{equation}
  \phi_0 - 5\phi_1 - 8\phi_2 = \phi_b\:.
  \label{eq:marshak_p1_deriv_3}
\end{equation}
The right hand side boundary condition will yield 2 complementary
equations if Marshak conditions are used or 2 non-zero moments to be
solved for if reflected conditions are used. The formulation above
also holds for vacuum conditions where $\phi_b = 0$.

\subsection{Formulation of the $SP_N$ Equations}
\label{subsec:spn_equations}

\subsection{Formulation of the Multigroup $SP_N$ Equations}
\label{subsec:mg_spn_equations}

%%---------------------------------------------------------------------------%%
\section{Spectral Analysis of the $SP_N$ Equations}
\label{sec:spectral_analysis}

%%---------------------------------------------------------------------------%%
\section{Monte Carlo Solutions for the $SP_N$ Equations}
\label{sec:monte_carlo}

%%---------------------------------------------------------------------------%%
\section{Conclusion}
\label{sec:conclusion}

%%---------------------------------------------------------------------------%%
\pagebreak
\bibliographystyle{ieeetr} \bibliography{references}
\end{document}
