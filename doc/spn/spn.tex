%        File: spn.tex
\documentclass[letterpaper]{article}
\usepackage[top=1.0in,bottom=1.0in,left=1.0in,right=1.0in]{geometry}
\usepackage{verbatim}
\usepackage{amssymb}
\usepackage{graphicx}
\usepackage{longtable}
\usepackage{amsfonts}
\usepackage{amsmath}
\usepackage[usenames]{color}
\usepackage[
naturalnames = true, 
colorlinks = true, 
linkcolor = black,
anchorcolor = black,
citecolor = black,
menucolor = black,
urlcolor = blue
]{hyperref}
\usepackage{listings}
\usepackage{textcomp}
\definecolor{listinggray}{gray}{0.9}
\definecolor{lbcolor}{rgb}{0.9,0.9,0.9}
\lstset{
  backgroundcolor=\color{lbcolor},
  tabsize=4,
  rulecolor=,
  language=c++,
  basicstyle=\scriptsize,
  upquote=true,
  aboveskip={1.5\baselineskip},
  columns=fixed,
  showstringspaces=false,
  extendedchars=true,
  breaklines=true,
  prebreak =
  \raisebox{0ex}[0ex][0ex]{\ensuremath{\hookleftarrow}},
  frame=single,
  showtabs=false,
  showspaces=false,
  showstringspaces=false,
  identifierstyle=\ttfamily,
  keywordstyle=\color[rgb]{0,0,1},
  commentstyle=\color[rgb]{0.133,0.545,0.133},
  stringstyle=\color[rgb]{0.627,0.126,0.941},
}

\author{Stuart R. Slattery
  \\ \href{mailto:sslattery@wisc.edu}{\texttt{sslattery@wisc.edu}}
}

\date{\today}
\title{Monte Carlo Solutions for the $SP_N$ Equations}
\begin{document}
\maketitle

%%---------------------------------------------------------------------------%%
\section{Introduction}
\label{sec:introduction}

%%---------------------------------------------------------------------------%%
\section{Derivation of the $SP_N$ Equations}
\label{sec:derivation}
In this section, we derive the $SP_N$ equations. We begin by stating the
general time-independent neutron transport equation followed by a
derivation of the $P_N$ equations in planar geometry for multiple energy
groups. From these equations, we then apply a set of approximations to
yield the multi-dimensional, multi-group $SP_N$ equations.

\subsection{The Neutron Transport Equation}
\label{subsec:transport_eq}
As a starting point we define the time-independent neutron transport
equation \cite{lewis_computational_1993}:
\begin{equation}
  \hat{\Omega} \cdot \vec{\nabla} \psi(\vec{r},\hat{\Omega},E) +
  \sigma(\vec{r},E) \psi(\vec{r},\hat{\Omega},E) = \int dE' \int
  d\Omega' \sigma_s(\vec{r},E' \rightarrow E,\hat{\Omega}' \rightarrow
  \hat{\Omega}) \psi(\vec{r},\hat{\Omega}',E') +
  q(\vec{r},\hat{\Omega},E)\:,
  \label{eq:general_transport}
\end{equation}
with the variables defined as:
\begin{itemize}
\item $\vec{r}$ - neutron spatial position
\item $\hat{\Omega}$ - neutron streaming direction with radial
  component $\mu$ and azimuthal component $\omega$
\item $E$ - neutron energy
\item $\psi(\vec{r},\hat{\Omega},E)$ - angular flux
\item $\sigma(\vec{r},E)$ - total interaction cross section
\item $\sigma_s(\vec{r},E' \rightarrow E,\hat{\Omega}'$ - probability
  of scattering from direction $\hat{\Omega}'$ into an angular domain
  $d\hat{\Omega}'$ about the direction $\hat{\Omega}$ and from energy
  $E'$ to an energy domain $dE'$ about energy $E$
\item $q(\vec{r},\hat{\Omega},E)$ - external source of neutrons.
\end{itemize}
For this work, it is sufficient to formulate
Eq~(\ref{eq:general_transport}) in 1-dimensonal Cartesian geometry:
\begin{equation}
  \mu \frac{\partial}{\partial x} \psi(x,\mu,E) + \sigma(x,E)
  \psi(x,\mu,E) = \int dE' \int d\Omega' \sigma_s(x,E' \rightarrow
  E,\hat{\Omega}' \rightarrow \hat{\Omega})
  \psi(\vec{r},\hat{\Omega}',E') + \frac{q(x,E)}{4 \pi}\:,
  \label{eq:general_transport}
\end{equation}
where the angular component of the solution is no longer dependent on
the azimuthal direction of travel and an isotropic source of neutrons
is assumed.

\subsection{Derivation of the $P_N$ Equations}
\label{subsec:pn_equations}
Next, we derive the $P_N$ equations, a simplified form of the general
transport equation where Legendre polynomials are used to expand the
angular flux and scattering cross section variables as a means of
capturing the angular structure of the solution. Before deriving this
form of the transport equation, we briefly discuss a few properties of
Legendre polynomials that we will find useful in the derivation.

\subsubsection{Legendre Polynomials}
\label{subsubsec:legendre_polys}

\subsubsection{Planar $P_N$ Equations}
\label{subsubsec:planar_pn}

\subsubsection{Boundary Conditions for the $P_N$ Equations}

%%---------------------------------------------------------------------------%%
\section{Spectral Analysis of the $SP_N$ Equations}
\label{sec:spectral_analysis}

%%---------------------------------------------------------------------------%%
\section{Monte Carlo Solutions for the $SP_N$ Equations}
\label{sec:monte_carlo}

%%---------------------------------------------------------------------------%%
\section{Conclusion}
\label{sec:conclusion}

%%---------------------------------------------------------------------------%%
\pagebreak
\bibliographystyle{ieeetr}
\bibliography{references}
\end{document}


